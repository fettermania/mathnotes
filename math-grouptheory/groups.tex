\documentclass[11pt, oneside]{article}   	% use "amsart" instead of "article" for AMSLaTeX format
\usepackage{geometry}                		% See geometry.pdf to learn the layout options. There are lots.
\geometry{letterpaper}                   		% ... or a4paper or a5paper or ... 
\usepackage[parfill]{parskip}    		% Activate to begin paragraphs with an empty line rather than an indent
\usepackage{graphicx}				% Use pdf, png, jpg, or eps§ with pdflatex; use eps in DVI mode
								% TeX will automatically convert eps --> pdf in pdflatex		
\usepackage{amssymb}
\usepackage{amsmath}

\usepackage{verbatim}
\usepackage{tikz} 

\usepackage{syntonly}
% \syntaxonly <-- use this for checking syntax only
% \mbox {text} - keep together
% \fbox {text} - keep together and draw around

%\pagestyle{plain|headings|empty} % header and footer p.27
%SetFonts
%\include{filename}, \includeonly{filename1, filename2} , \input[fiename}

%SetFonts% 

\title{Brilliant: Group Theory}
\author{Dave Fetterman}
\date{2/23/22}							% Activate to display a given date or no date

\begin{document}
\maketitle
Note: Latex reference: http://tug.ctan.org/info/undergradmath/undergradmath.pdf
\section{Chapter 1.2}
\subsection{Page 1}
$R (R_1 (x) ) = A \rightarrow B, B \rightarrow A, C \rightarrow C$.
So reflection about CE.

\subsection{Page 2}
$R_2 (R_1 (x) ) = A \rightarrow B, B \rightarrow C, C \rightarrow A$.
So rotation clockwise ${120^\circ}$

\subsection{Page 5}
$ R \star R = H \star H = V \star V = I $ on the letter "I".


\subsection {Page 6 - 9}
Cayley table for rotating letter "I":

\begin{tabular}{|l|l|l|l|l|}
\hline
  & I & H & V & R \\ \hline
I & I & H & V & R \\ \hline
H & H & I & R & V \\ \hline
V & V & R & I & H \\ \hline
R & R & V & H & I \\ \hline
\end{tabular}

\emph{Note: check out https://www.tablesgenerator.com/ here.}


\subsection {Page 10}

\begin{itemize}
\item Klein four group: $ (+, [0, 1] \times [0, 1]) $ is equivalent to the "I" rotation.  

\item First coord could be: Does it rotate?
\item Second coord could be: Does it flip?
\end{itemize}

\section {Chapter 1.3}

Group Properties

\begin{itemize}
\item Some binary operation $( \cdot )$
\item Identity (not e.g., even integers)
\item Inverse (not e.g. multiplication modulo non-prime p)
\item Associativity (not e.g. an average $f(x,y) = (x+y)/2$)?
\end{itemize}

\section {Chapter 1.4}

Cube symmetries

One way  to think about it: 

\begin{itemize}
\item Corner \emph{A} maps to one of eight new corners
\item Each mapping has three orientations of that corner spin  (0 degrees, 120, 240)
\item Therefore 24
\end{itemize}

Another way:
\begin{itemize}
\item One identity = 1
\item Type I: Rotate around line joining two opposite face centers: 3 pairs * 3 non-identity spins = 9
\item Type II: Spin around line joining two opposite corners.  4 pairs * 2 non-identiy spins = 8
\item Type III: Spin 180 degrees around line from front upper edge to back lower edge.  Combo of a spin and a rotate.  6 pairs = 6.
\item Sum to 24.
\end{itemize}

Another way:
\begin{itemize}
\item There are four diagonals to a cube.
\item Their permutations are in 1:1 correspondence with the transformations possible.   (24)
\item Type I keeps none fixed.  90 degrees: Chain = $4! / 4$ = 6.   180 degrees: two pairs.  Select who $A$ matches = 3.
\item Type II rotates three, keeps one fixed = 8
\item Type III does one swap, keeps two fixed = ${4 \choose 2} = 6$
\end{itemize}

Note also: There are 24 reflection symmetries as well. (1:1 correspondence with rotations via "swap top center labels?")

\section {Chapter 2.1}

\subsection {Page 2-3}
The integers under multiplication are not a group, as they have no inverse.
The set of rationals with multiplication as the group operation is not a group as 0 has no inverse

\subsection {Page 5 - 7}

\begin{itemize}
\item Dihedral group $D_n$ has $2n$ elements, is not commutative, not cyclical.
\item If n is even, there is exactly one rotational symmetry $R\neq I$
 =I which commutes with all the other elements of $D_n$ (the 180 degree rotation)
\end{itemize}

\subsection {Page 8 - 9}

\begin{itemize}
\item Symmetric group $S_n$ is the set of permutations on n elements.
\item "in-shuffle" of a deck of four cards is "split in half, interleave top half with bottom half, top card second", or $\phi = (1, 2, 4, 3)$.  $\phi^4 = I$
\end{itemize}

\subsection {Page 10-11}

\begin{itemize}
\item Cyclic group $Z_n$ is the set of integers modulo $n$ under addition.  
\item Note that usually multiplication is the group operation, it usually uses "+".
\item Every element in $Z_n$ is its own inverse iff n is even.
\end{itemize}

\section {Chapter 2.2: More Group Examples}

\subsection {Page 1-2}

\begin{itemize}
\item Order of an element $g$ is smallest $k$ such that $g^k = e$.  Otherwise \emph{infinite order}
\end{itemize}

\subsection {Page 3}

Quaternion group $Q_8$ rules:

\begin{itemize}
\item $i^2 = j^2 = k^2 = ijk =  - 1$
\item Implies $ij = k, jk = i, ki = j$
\item implies $ji = -k, kj = -i, ik = -j$
\item So this is not only \emph{non-commutative} but \emph{anti-commutative}
\item $Q = {\pm 1, \pm i, \pm j, \pm k}$
\item So one element of order 1, one of order 2 (element -1), remaining six of these elements have order 4
\end{itemize}

\subsection {Page 4}

Note that  musical notes ($Z_{12}$) has only generators 1, 5, 7, 11.  
These corresponding to chromatic, circle of fourths (anti-fifths), circle of fifths, downwards chromatic scales!

\subsection {Page 55}

\begin{itemize}
\item $GL_n(\mathbb{R})$ is invertible n x n matrices in R.
\item $SL_n(\mathbb{R})$ is determinant 1 n x n matrices in R.
\item 
 
$A = \begin{pmatrix}
    -1      & 1 \\
    0       & 1 \\

\end{pmatrix}
$
has order 2,  $B = \begin{pmatrix}
    -1      & 0 \\
   0      & 1 \\

\end{pmatrix}$
has order 2,  but $AB = \begin{pmatrix}
    1       & 	1 \\
    0      & 1\\

\end{pmatrix}$ has infinite order!  Non-commutativity strikes.

\end{itemize}


\subsection {Page 6-11}

\begin{itemize}
\item \emph{isomorphism} is a bjiection preserving group operations.
\item Can think of it as a relabeling of the Cayley table.
\item Example given is Klein-four and symmetries of tall serif letter "I", or of a diamond/non-square rhombus.
\item $Z_{12}$ is isomorphic to rotational symmetries of a 12-gon.
\item $Q_8$ is isomorphic under matrix multiplication to 
$\Bigg\{\pm \begin{pmatrix}
    1      & 0 \\
    0       & 1 \\

\end{pmatrix}
$, 
$\pm \begin{pmatrix}
    i      & 0 \\
    0       & -i \\

\end{pmatrix}
$, 
$\pm \begin{pmatrix}
    0      & 1 \\
    -1       & 0 \\

\end{pmatrix}
$, 
$\pm \begin{pmatrix}
    0      & i \\
    i       & 0 \\

\end{pmatrix}
\Bigg\} \subset GL_2(\mathbb{R})$ 
\item $D_3$ is isomorphic to $S_3$ since any permutation is possible and no more.
\end{itemize}


\section {Chapter 2.3: Subgroups}

\subsection {Page 1 - 3}

\begin{itemize}
\item Subgroups are closure-bound subsets of groups.  
\item Easy test: $H \subset G$ if for every $h_1, h_2 \in H, h_1h_2 \in H$, and for any $h \in H, h^{-1} \in H$.
\end{itemize}

\subsection {Page 4}

\begin{itemize}
\item Cartesian product of groups G, H is also a group: $G \times H = (g, h) \cdot (g', h') = (gg', hh'), g \in G, h \in H$  
\end{itemize}

\subsection {Lagrange's theorem}

Lemma: $H \subset G.  r,s \in G.  Hr = Hs \iff rs^{-1} \in H$.  Otherwise, $Hr, Hs$ have no element in common.

One direction: $rs^{-1} \in H \rightarrow Hr=Hs$
\begin{itemize} 
\item $rs^{-1}  = h \in H$ by supposition
\item $Hh = Hrs^{-1} = H$
\item $Hr = Hs$
\end{itemize}

Other direction: $ Hr=Hs \rightarrow rs^{-1} \in H$
\begin{itemize} 
\item $Hr = Hs$ by supposition
\item $Hrs^{-1} = H, so h_1rs^{-1} = h_2$ for some $h_1, h_2$.
\item $rs^{-1} = h_1^{-1}h_2 \in H$
\end{itemize}

Therefore, if Hr and Hs have some element in common, meaning $h_1r=h_2s$, then $rs^{-1} = h_1^{-1}h_2 \in H$.  
So, by the first direction above, $Hr = Hs$.

\emph{Lagrange construction}:
\begin{itemize}
\item Take $r_1 \in G$, so $Hr_1 = H$.  
\item If $H \neq G$, take $r_2 \in G - Hr_1$ to create $Hr_2$.
\item Repeat.  We wiil thus create disjoint $Hr_1, Hr_2, ... $ of the same size.  
\end{itemize}

\subsection {My take on Lagrange}

\begin{itemize}
\item If $t \in Hr since t = h_1r$ and $t \in Hs since t=h_2s$, then $r = h_1^{-1}h_2s \in Hs$ and likewise for s, so $Hr = Hs$.  So every element is in both or neither.
\item Therefore "H" is a partition relation on the elements of G.
\item Size of $Hr$ equals size of H for obvious group reasons.
\item Every element g of G is in some coset $Hg$.  
\item Therefore G is partitioned into cosets of equal size, which is size of H.
\item Therefore size of subgroup H divides size of group G
\end{itemize}

\begin{comment}

* Cosets form equivalence classes
** Hr and Hs share either one element (identity) or all elements.
*** Suppose t in Hr and Hs, and t is not identity.
*** Then t = h_1*r = h_2 * s
*** Suppose r, s in H.  Then equal.
*** Suppose 
* Cosets are the same size
* Size of cosets divide group size
* Therefore size of He = H divides group size
\end{comment}

\subsection {Page 7-12}

\begin {itemize} 
\item Note that if H and K are subgroups, so is $H \cap K$.
\item $Z_6$ has subgroups $Z_6, {0, 2, 4}, {0, 3}, {0}$, all divisors of 6 in this case.
\item $Z_p$, p prime, has only subgroups $Z_p, {0}$
\item $Z_p \times Z_p$ has p + 3 subgroups 
  \begin{itemize}
  \item $Z_p \times Z_p$
  \item Generator (0,0)
  \item Generator (0,1)
  \item All generators $(1, n), n \in [0, p-1]$ . p of those.
  \end {itemize} 
\item Another way to think about  $Z_p \times Z_p$: Outside of (0,0), the remaining $p^2-1$ elements each have order p.  They are generate a group of size p, minus the identity.  So $(p^2-1)/(p-1) + 2 = p+3.$
\item Subgroup count of $Z_4 \times Z_2$: a counting exercise, based on generators. 
\begin{itemize}
\item Look at all cyclic groups of each of the elements.
\item (0,0) generates 1 group
\item Order 2: Three elements, which generate three distinct cyclic subgroups
\item Order 4: Four elements, which generate two distinct subgroups
\item Order 8: $Z_4 \times Z_2$, non-cyclic
\item And there's one distict $Z_2 \times Z_2$ group.
\item \emph{Note: Is there a good (even recursive) formula for this?}
\end {itemize} 

\end {itemize} 

\section {Chapter 2.4: Abelian Groups}


\subsection{Page 1-3}
\begin{itemize}
\item Theorem: $Z_a \times Z_b$ is isomorphic to $Z_{ab}$ iff a and b are relatively prime.
\item DF Proof: If a and b are relatively prime, (1,1) is of order $ab$.  If $a$ and $b$ share factor $c$, then $Z_{ab}$ has an element of order $ab$, but $Z_a \times Z_b$ will have cycled by $a * b / c$.
\item So decompose e.g. $Z_{12}$ into $Z_4 \times Z_3$, for example.
\end{itemize}

\subsection{Page 4-6}
\begin{itemize}
\item Theorem: Every finite abelian group is isomorphic to a direct product of cyclic groups.
\item Therefore, the number of these groups of order n is the product of the partitions of each of its prime factors' powers.
\item Therefore, the number of abelian groups of size $24 = 3* 2^3 = p(3) * p(1) = 3 * 1 = 3, Z_3  \times Z_8, Z_3  \times Z_4 \times Z_2, Z_3  \times Z_2 \times Z_2 \ times Z_2$
\item Therefore, the number of abelian groups of size $2310 = 2 * 3 *5 * 7 * 11$ is one.
\end{itemize}

\subsection{Page 7-11: $Z_n^*$ or $U(n)$}
\begin{itemize}
\item Group $Z_n^*$: elements of $Z_n$ relatively prime to n, under multiplication. 
\item $|Z_n^*| = \phi(n)$ , the totient function.
\item This is a group even if n not prime because there is $ax + bn = 1$  if $x, n$ are relatively prime.
\item $Z_8^* = \{1,3,5,7\}$ is isomorphic to $Z_2  \times Z_2$ since every element squared is 1.
\item $Z_10^* = \{1,3,7,9\}$ is isomorphic to $Z_4$ since it is generated by 3.
\item $Z_15^* = \{1,2,4,7,8,11,13,14\}$ is isomorphic to $Z_4 x Z_2$ by counting element orders.
\item Note: Primitive roots of n are those that generate $Z_n^*$.  There are primitive roots mod n if and only if $n = 1,2,4,p^k, 2p^k$.
\item TODO: read https://brilliant.org/wiki/primitive-roots/ and why these are the only solutions. Also, look up \emph{Legendre symbol}
\end{itemize}

\section {Chapter 2.5: Homomorphisms}
\subsection(\subsection{Page 1 - X}
\begin{itemize}
\item Homomorphism $\phi: \phi(a) *' \phi(b) = \phi(a * b)$.  Note that  * and *' are different  operations.
\item This means, "translate each via the function, then combine" yields the same result as "combine first, then translate".  So structure is preserved.
\item Note this is like isomorphism, except homeomorphism can squash some items to zero.
\item Also, this can change to an entirely separate domain, e.g. $det(AB) = det(A)det(B)$
\item Easy to prove homomorphism preserves identities and inverses.
\item Order of transformed element $\phi(g)$ divides order of g, since $g^k = e and \phi(g)^k = \phi(e)$, but consider - we could map everything to the identity! 
\item
\end{itemize}

\subsection(\subsection{Page 7- 10: Counting homomorphisms}
\begin{itemize}
\item Main idea: Knowing where we send identity determines entire homomorphism for a cyclic group.
\item Homomorphism count for $Z_4 \rightarrow Z_{10}$: There are 10 places to send identity, but recall that $\phi(1)$ has to have order 4 since $\phi(1 + 1 + 1 + 1) = \phi(0) = 0$.  Therefore, $\phi(1)$ has to be 0 or 5.  So 2 possibilities.
\item Homomorphism count for $Z_{99} \rightarrow Z_{100}$: Since $\phi(99) = 0 and \phi(1) \times 100$, it must divide both. Therefore, $\phi(1) = 1$, and only one possibility.
\item  Homomorphism count for $Z_{99} \rightarrow Z_{99}$: 99, since $99 \cdot \phi(1) = 0$, so $\phi(1)$ can go anywhere.
\item  Homomorphism count for $D_3 \rightarrow Z_3$: 1, since $D_3$ has 3 elements of order 2, 2 of order 3, 1 of order 1.  Only mapping everything to $0$ works.
\end{itemize}


\subsection{Page 11: Counting automorphisms}
\begin{itemize}
\item Automorphism is isomorphism from group to itself.
\item Count of automorphisms of $Z_8$: If 1 maps to an order-8 element, we're isomorphic.  There are four: ${1,3,5,7}$
\item $Aut(Z_8)$ is isomorphic to $Z_2 \times Z_2$, since $\phi_3(1)^2 = \phi_5(1)^2 = \phi_7(1)^2 = 1$, where $\phi_a$ maps a to 1.  Three elements of order 2 means it's the Klein 4 group.
\item Count of \emph{automorphisms} (meaning, we need all the elements in the codomain) of $Z_2 \times Z_2 \times Z_2$: Think of $\phi((1,0,0)), \phi((0,1,0)), \phi((0,0,1))$ as the basis for the group.  There are seven choices for the first, six for the next, and \emph{four} for the third.  
\item The above group is $(\phi(e_1) | \phi(e_2) | \phi(e_3)) = GL(\mathbb{F}_2)$, invertible matrices of 3x3.
\end{itemize}


\section {Chapter 2.6: Quotient Groups}
\subsection{Aside: Complex multiplication}
\begin{itemize}
\item Complex modulus (size) of $a+bi$ is defined as $root(a^2+b^2)$
\item Complex multiplication: Angles add, moduli multiply
\item One proof of moduli: $(a+bi)(c+di) = (ac - bd) + (ad+bc)i$
and $\sqrt{a^2+b^2}\sqrt{c^2+d^2} = \sqrt{a^2c^2 + b^2d^2 - 2abcd + ad^2+bc^2 + 2adbc}$
\item One proof of angles: Convert to $r_1(cos(a) + sin(a))r_2(cos(b)+sin(b))$ and multiply
\item More visual proof: Think of $c_1(a+bi) = c_1a + i(c_1b)$. $a$ scales original vector, and $bi$ rotates by 90 degrees and scales.
\end{itemize}

\subsection{Page 1-6}
\begin{itemize}
\item $S^1$, is defined as the group of complex numbers with modulus 1.
\item The coset $zS^1$ is any complex number multiplied by $S^1$ , which is a circle about the origin.  $z=2$ and $z=2i$ would be in the same coset.  These cosets are members of $C^*$ with the same modulus (length).
\item These are disjoint cosets that fill out $\mathbb{C}^*$ (don't include the zero, since no inverse).
\item If you consider $H=x+iy$, $x > 0, y=0$ (positive reals) then the cosets are rays from the origin.  Any $zH$ is just the different sizes of that (say, unit) vector.  These cosets are members of $C^*$ with the same angle.
\item \textbf{quotient group} of $\mathbb{C}^*$ by $S^1$:
	\begin{itemize}
	\item Members are cosets
	\item Multiplying is defined as $aH \times bH = abH, H \in S^1, a,b \in \mathbb{C}^*$ 
	\item $S^1$ is therefore the identity.
	\item This group is isomorphic to $R^+$ under multiplication (or really, like H).
	\item "A ray of angle A and a ray of angle B multiply to a ray of angle AB, forget about the size".
	\item This is like collapsing out the divisor, in this case, $S^1$.
	\item size $|G / H| = |G| / |H|$ since cosets are equally sized.
	\item \textbf{Gotcha}: Only works (meaning, $g_1, g_1' \in C_1, g_2, g_2' \in C_2$ implies $g_1g_2$ in same coset as $g_1'g_2'$ ) if H is \textbf{normal} in G.  
	\item Note: Normal means $xH = Hx$, so that makes sense that $g_1Cg_2C = g_1g_2C*C=g_1g_2C$
	\end{itemize}
\item So $\mathbb{C}^* / H$ is all the rays with the same modulus, or $S^1$.  
\item	"A ray of size X and a ray of size Y multuply to a ray of size XY, and forget about the angles".
\item So $\mathbb{C}^* / S^1 = H, mathbb{C}^* \ H = S^1$!
\end{itemize}


\subsection{Page 7-12}
\begin{itemize}
\item Another example: $\mathbb{Z} / 10\mathbb{Z} = \mathbb{Z_{10}}$ under addition.  Forget about the non-unit digits!
\item Another example: $\mathbb{Q} / \mathbb{Z}$ is $\overline{q} = q + \mathbb{Z}$, so $\overline{1/2} + \overline{2/3} = \overline{1/6}$
\item Another example: if $N$ is the \textbf{center} (omni-commuter subgroup) of $D_4$, then N is two elements $I, R_{180}$.  Forgetting about those we have cosets $(I, R_{180})N, (R_{90}, R_{270})N, (D_1, D_2)N, (V,H)N$.  All non-identity are degree 2, so isomorphic to $Z_2 \times Z_2$
\item Another example: $Z_{13}^*$ with multiplication mod 13.  $N = {1,12}$ is a normal subgroup.  $Z_{13}^* / N$ is "forget about the +/- 1 of it and think of these as 1 through 6.  
\item Another example: \textbf{commutator subgroup} [a,b] is generated by all $aba^{-1}b^{-1}$ for all $a, b \in G$.  So, group members are products of these guys, not necessarily all of that form.
\item This is just $e$ for an Abelian group.  Its size measures "how far" the group is from being Abelian.  
\item \textbf{Main idea} of quotients: "what do we force to the identity?"  If we say every $\overline{aba^{-1}b^{-1}} = \overline{1}$, then you can multiby by $ba$ to get $\overline{ab} =  \overline{ba}$.  So $G / [G,G]$ is necessarily Abelian.
 \end{itemize}


\section{Chapter 3.1: Number Theory}


\subsection{Page 1- 7}
\begin{itemize}
\item A Fermat's little theorem proof
 \begin{itemize}
  \item Take prime $p$, and $a$ not divisible by $p$.
   \item ${a, 2a, 3a ... , (p-1)a} \equiv {1, 2, 3, ... (p-1)} mod p$ since they're the same elements mod p.
   \item Take the product of each: $a^{p-1}(p-1)! \equiv (p-1)! mod p$
   \item Divide $(p-1)!$ out (there's an inverse mod p) and you get $a^{p-1} \equiv 1$ mod p
  \end{itemize}
 \item Another: Since the order of $a$ in $\mathbb{Z}_p^*$ is $p-1$, $a^{p-1} \equiv 1$ mod p.
 \item Note: Generalization of Fermat's little theorem using same group argument: $a^{\phi(n)} \equiv 1$ if $a$ and $n$ relatively prime.
 \end{itemize}
 
 \subsection{Page 8-11}
 \begin{itemize}
 \item Wilson's theorem: $1 * 2 * ... * (p-1) \equiv -1$ mod $p$.
 \item One proof: These all have inverses, except 1 and -1 mod p, which are self-inverting ($x^2 = 1$ solutions).
 \item This also proves that the product of all elements of a finite Abelian group \emph{which has a single element $g$ of order 2} is that element, $g$.
 \item A hard proof TODO. The powers of a \textbf{primitive root of p} yield all elements $a mod p$.  So $\mathbb{Z}_p^*$ is cyclical for any prime p.
\item One more proof: if k relatively prime to p-1, where p a prime > 2, then $1^k + 2^k + ... + (p-1)^k \equiv 0$ mod $p$, since each of these summands is a different member of the group, summing to $\frac{p(p-1)}{2}$
 \end{itemize}
 
 \section{Chapter 3.2: Games}
 \subsection{15 puzzle}
 I think this will go:
 - The board is a permutation of (1, 2, ... 15), read like a book, with a blank  somewhere in there ,immaterial.
 - Sliding the blank left or right doesn't change the order.
 - Sliding it up or down skips three backward or forward.
 
\textbf{Their proof}: Think of this as a series of swaps with $(j, 16)$, 16 being the blank tile.  To return to the bottom right corner, 16 must make an even number of moves.  So only even permutations allowed.  So (14,15) is not a viable swap, nor any of the odd permutations.  

\section{Chapter 3.3: Peg solitaire}

\begin{itemize}
\item Consider Klein four group: $xy = yx = z, yz = zy = x, xz = zx = y$.  
\item Label all pegs such that three consecutive are always, in some order: x, y, z
\item Invariant: product of all occupied spaces.  If x jumps over y to get to z, eliminating jumped peg, xy = z.
\item 11  x's, 11 z's, 10 y's yield xz = y as the product.
\end{itemize}

\section{Chapter 3.4: Rubix's Cube}

\begin{itemize}
\item Each element is the state $(S_{12}, S_8, (Z_2)^{12}, (Z_3)^8$), representing around a fixed set of centers: (middle selections, corner sleections, middle orientation, corner orientation).
\item Invariant: First and second perms for all F,B,D,U,L,R are odd, so first two args need same permutation parit
\item Invariant: (Not proven here): Sum of edge orientations (0,1) is zero, sum of corner orientations (0, 1, 2) is zero.
\item \textbf{Commutator}: $ghg^{-1}h^{-1}$ measure how entangled $g$ and $h$ are.  If they're commutatitive, it is $e$.
\item For Rubix's cube, commutators  $ghg^{-1}h^{-1}$ are great for only moving pieces where effects of $g$ and $h$ overlap.
\item $g$ and $h$ are \textbf{conjugates} if some x such that $h = x^{-1}gx$.   "h is same as g, just in a different location".
\item Conjugate interpretation: "h is move via x, operate with g, move back via x.  "\
\item For Rubix's cube - you can use conjugates to make whatever change to a different part of the cube (move it to the operating table, operate, move it back).
\end{itemize}

\section{Chapter 4.1: Normal Subgroups}

\begin{itemize}
\item Think: Every conjugacy $g^{-1}Hg$ moves a group to another subgroup.  Normal subgroups  $g^{-1}Ng = N$ are the ones \emph{that don't move}
\item Example of non-normal: Any one of the $n$ sets of $S_{n-1}$ among conjugates of $S_n$.  
\item Normal definition: Group N is normal if and only if
	\begin{itemize}
	\item $gN = Ng$ for all $g \in G$
	\item $gNg^{-1} = N$ for all $g \in G$ (equiv to above)
	\item $gng^{-1} \in N$ for all $g \in G$
	\end{itemize}
\item Trivial: Any subgroup of index 2 is normal.  G has two distinct cosets $N$, $gN$, but also $N$ and $Ng$ so $gN = Ng$.
\item Normal doesn't recursively nest.  
	\begin{itemize}
	\item If $G$ has normal subgroup $H$ and $H$ has normal subgroup $K$, $K$ is normal in $H$ too (those elements also "pass through K"
	\item However, $H$ can be normal in $G$ (e.g. ${I, R_{180}, F_v, F_h}$ in $D_4$,  $K$ can be normal in $H$ (e.g. ${I, V}$, but $K$ is not normal in $G: VR_{90} = D_{ul}, R_{90}V = D_{ur}$
	\end{itemize}
\item Normal examples in $GL_2(\mathbb{C})$: $SL_2(\mathbb{C})$ (determinant 1) and non-zero diags $zI_2$.  
\item Non-normal examples in $GL_2(\mathbb{C})$: $GL_2(\mathbb{R})$ and non-zero diags with different entries. Easy to throw some arbitrary ones in Wolfram Alpha and see messed up after conjugation.
\item G's Center $Z(G)$ are the omni-commuters.  Always normal.
\item G's Commutator group $[G,G]$: Product of any $aba^{-1}b^{-1}$ for $a, b \in G$. is normal, since $g[a,b]g^{-1} = [gag^{-1}, gbg^{-1}]$.

\end{itemize}

\end{document}