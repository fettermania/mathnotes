\documentclass[11pt, oneside]{article}   	% use "amsart" instead of "article" for AMSLaTeX format
\usepackage{geometry}                		% See geometry.pdf to learn the layout options. There are lots.
\geometry{letterpaper}                   		% ... or a4paper or a5paper or ... 
\usepackage[parfill]{parskip}    		% Activate to begin paragraphs with an empty line rather than an indent
\usepackage{graphicx}				% Use pdf, png, jpg, or eps§ with pdflatex; use eps in DVI mode
								% TeX will automatically convert eps --> pdf in pdflatex		
\usepackage{amssymb}
\usepackage{amsmath}

\usepackage{verbatim}
\usepackage{tikz} 

\usepackage{syntonly}
% \syntaxonly <-- use this for checking syntax only
% \mbox {text} - keep together
% \fbox {text} - keep together and draw around

%\pagestyle{plain|headings|empty} % header and footer p.27
%SetFonts
%\include{filename}, \includeonly{filename1, filename2} , \input[fiename}

%SetFonts% 

\title{Brilliant: Group Theory}
\author{Dave Fetterman}
\date{2/23/22}							% Activate to display a given date or no date

\begin{document}
\maketitle
Note: Latex reference: http://tug.ctan.org/info/undergradmath/undergradmath.pdf
\section{Chapter 1.2}
\subsection{Page 1}
$R (R_1 (x) ) = A \rightarrow B, B \rightarrow A, C \rightarrow C$.
So reflection about CE.

\subsection{Page 2}
$R_2 (R_1 (x) ) = A \rightarrow B, B \rightarrow C, C \rightarrow A$.
So rotation clockwise ${120^\circ}$

\subsection{Page 5}
$ R \star R = H \star H = V \star V = I $ on the letter "I".


\subsection {Page 6 - 9}
Cayley table for rotating letter "I":

\begin{tabular}{|l|l|l|l|l|}
\hline
  & I & H & V & R \\ \hline
I & I & H & V & R \\ \hline
H & H & I & R & V \\ \hline
V & V & R & I & H \\ \hline
R & R & V & H & I \\ \hline
\end{tabular}

\emph{Note: check out https://www.tablesgenerator.com/ here.}


\subsection {Page 10}

\begin{itemize}
\item Klein four group: $ (+, [0, 1] \times [0, 1]) $ is equivalent to the "I" rotation.  

\item First coord could be: Does it rotate?
\item Second coord could be: Does it flip?
\end{itemize}

\section {Chapter 1.3}

Group Properties

\begin{itemize}
\item Some binary operation $( \cdot )$
\item Identity (not e.g., even integers)
\item Inverse (not e.g. multiplication modulo non-prime p)
\item Associativity (not e.g. an average $f(x,y) = (x+y)/2$)?
\end{itemize}

\section {Chapter 1.4}

Cube symmetries

One way  to think about it: 

\begin{itemize}
\item Corner \emph{A} maps to one of eight new corners
\item Each mapping has three orientations of that corner spin  (0 degrees, 120, 240)
\item Therefore 24
\end{itemize}

Another way:
\begin{itemize}
\item One identity = 1
\item Type I: Rotate around line joining two opposite face centers: 3 pairs * 3 non-identity spins = 9
\item Type II: Spin around line joining two opposite corners.  4 pairs * 2 non-identiy spins = 8
\item Type III: Spin 180 degrees around line from front upper edge to back lower edge.  Combo of a spin and a rotate.  6 pairs = 6.
\item Sum to 24.
\end{itemize}

Another way:
\begin{itemize}
\item There are four diagonals to a cube.
\item Their permutations are in 1:1 correspondence with the transformations possible.   (24)
\item Type I keeps none fixed.  90 degrees: Chain = $4! / 4$ = 6.   180 degrees: two pairs.  Select who $A$ matches = 3.
\item Type II rotates three, keeps one fixed = 8
\item Type III does one swap, keeps two fixed = ${4 \choose 2} = 6$
\end{itemize}

Note also: There are 24 reflection symmetries as well. (1:1 correspondence with rotations via "swap top center labels?")

\section {Chapter 2.1}

\subsection {Page 2-3}
The integers under multiplication are not a group, as they have no inverse.
The set of rationals with multiplication as the group operation is not a group as 0 has no inverse

\subsection {Page 5 - 7}

\begin{itemize}
\item Dihedral group $D_n$ has $2n$ elements, is not commutative, not cyclical.
\item If n is even, there is exactly one rotational symmetry $R\neq I$ which commutes with all the other elements of $D_n$ (the 180 degree rotation)
\end{itemize}

\subsection {Page 8 - 9}

\begin{itemize}
\item Symmetric group $S_n$ is the set of permutations on n elements.
\item "in-shuffle" of a deck of four cards is "split in half, interleave top half with bottom half, top card second", or $\phi = (1, 2, 4, 3)$.  $\phi^4 = I$
\end{itemize}

\subsection {Page 10-11}

\begin{itemize}
\item Cyclic group $Z_n$ is the set of integers modulo $n$ under addition.  
\item Note that though usually multiplication is the default group operation, this usually uses "+".
\end{itemize}

\section {Chapter 2.2: More Group Examples}

\subsection {Page 1-2}

\begin{itemize}
\item \textbf{Order of an element} $g$ is smallest $k$ such that $g^k = e$.  Otherwise \textbf{infinite order}
\end{itemize}

\subsection {Page 3}

Quaternion group $Q_8$ rules:

\begin{itemize}
\item $i^2 = j^2 = k^2 = ijk =  - 1$
\item Implies $ij = k, jk = i, ki = j$
\item implies $ji = -k, kj = -i, ik = -j$
\item So this is not only \emph{non-commutative} but \emph{anti-commutative}
\item $Q = {\pm 1, \pm i, \pm j, \pm k}$
\item So one element of order 1, one of order 2 (element -1), remaining six of these elements have order 4
\end{itemize}

\subsection {Page 4}

Note that  musical notes ($Z_{12}$) has only generators 1, 5, 7, 11.  
These corresponding to chromatic, circle of fourths (anti-fifths), circle of fifths, downwards chromatic scales!

\subsection {Page 55}

\begin{itemize}
\item $GL_n(\mathbb{R})$ is invertible n x n matrices in R.
\item $SL_n(\mathbb{R})$ is determinant 1 n x n matrices in R.
\item 
 
$A = \begin{pmatrix}
    -1      & 1 \\
    0       & 1 \\

\end{pmatrix}
$
has order 2,  $B = \begin{pmatrix}
    -1      & 0 \\
   0      & 1 \\

\end{pmatrix}$
has order 2,  but $AB = \begin{pmatrix}
    1       & 	1 \\
    0      & 1\\

\end{pmatrix}$ has infinite order!  Non-commutativity strikes.

\end{itemize}


\subsection {Page 6-11}

\begin{itemize}
\item \textbf{isomorphism} is a bjiection preserving group operations.
\item Can think of it as a relabeling of the Cayley table.
\item Example given is Klein-four and symmetries of tall serif letter "I", or of a diamond/non-square rhombus.
\item $Z_{12}$ is isomorphic to rotational symmetries of a 12-gon.
\item $Q_8$ is isomorphic under matrix multiplication to 
$\Bigg\{\pm \begin{pmatrix}
    1      & 0 \\
    0       & 1 \\

\end{pmatrix}
$, 
$\pm \begin{pmatrix}
    i      & 0 \\
    0       & -i \\

\end{pmatrix}
$, 
$\pm \begin{pmatrix}
    0      & 1 \\
    -1       & 0 \\

\end{pmatrix}
$, 
$\pm \begin{pmatrix}
    0      & i \\
    i       & 0 \\

\end{pmatrix}
\Bigg\} \subset GL_2(\mathbb{R})$ 
\item $D_3$ is isomorphic to $S_3$ since any permutation is possible in $D_3$ and no more.
\end{itemize}


\section {Chapter 2.3: Subgroups}

\subsection {Page 1 - 3}

\begin{itemize}
\item Subgroups are closure-bound subsets of groups.  
\item Easy test: $H \subset G$ if for every $h_1, h_2 \in H, h_1h_2 \in H$, and for any $h \in H, h^{-1} \in H$.
\end{itemize}

\subsection {Page 4}

\begin{itemize}
\item Cartesian product of groups G, H is also a group: $G \times H = (g, h) \cdot (g', h') = (gg', hh'), g \in G, h \in H$.  Is this also called the \textbf{direct product}? 
\end{itemize}

\subsection {Lagrange's theorem}

Theorem: Order of every subgroup divides the containing group.

Lemma: $H \subset G.  r,s \in G.  Hr = Hs \iff rs^{-1} \in H$.  Otherwise, $Hr, Hs$ have no element in common.

One direction: $rs^{-1} \in H \rightarrow Hr=Hs$
\begin{itemize} 
\item $rs^{-1}  = h \in H$ by supposition
\item $Hh = Hrs^{-1} = H$
\item $Hr = Hs$
\end{itemize}

Other direction: $ Hr=Hs \rightarrow rs^{-1} \in H$
\begin{itemize} 
\item $Hr = Hs$ by supposition
\item $Hrs^{-1} = H, so h_1rs^{-1} = h_2$ for some $h_1, h_2$.
\item $rs^{-1} = h_1^{-1}h_2 \in H$
\end{itemize}

Therefore, if Hr and Hs have some element in common, meaning $h_1r=h_2s$, then $rs^{-1} = h_1^{-1}h_2 \in H$.  
So, by the first direction above, $Hr = Hs$.

\emph{Lagrange construction}:
\begin{itemize}
\item Take $r_1 \in G$, so $Hr_1 = H$.  
\item If $H \neq G$, take $r_2 \in G - Hr_1$ to create $Hr_2$.
\item Repeat.  We wiil thus create disjoint $Hr_1, Hr_2, ... $ of the same size.  
\end{itemize}

\subsection {My take on Lagrange}

\begin{itemize}
\item If $t \in Hr$ since $t = h_1r$ and $t \in Hs$ since $t=h_2s$, then $r = h_1^{-1}h_2s \in Hs$ and likewise for s, so $Hr = Hs$.  So every element is in both or neither.
\item Therefore $H(x) = Hx$ is a partition relation on the elements of G.
\item Size of $Hr$ equals size of H for obvious group reasons.
\item Every element g of G is in some coset $Hg$.  
\item Therefore G is partitioned into cosets of equal size, which is size of H.
\item Therefore size of subgroup H divides size of group G
\end{itemize}

\begin{comment}

* Cosets form equivalence classes
** Hr and Hs share either one element (identity) or all elements.
*** Suppose t in Hr and Hs, and t is not identity.
*** Then t = h_1*r = h_2 * s
*** Suppose r, s in H.  Then equal.
*** Suppose 
* Cosets are the same size
* Size of cosets divide group size
* Therefore size of He = H divides group size
\end{comment}

\subsection {Page 7-12}

\begin {itemize} 
\item Note that if H and K are subgroups, so is $H \cap K$.
\item $Z_6$ has subgroups $Z_6, {0, 2, 4}, {0, 3}, {0}$, all divisors of 6 in this case.
\item $Z_p$, p prime, has only subgroups $Z_p, {0}$
\item $Z_p \times Z_p$ has p + 3 subgroups 
  \begin{itemize}
  \item $Z_p \times Z_p$
  \item Generator (0,0)
  \item Generator (0,1)
  \item All generators $(1, n), n \in [0, p-1]$ . p of those.
  \end {itemize} 
\item Another way to think about  $Z_p \times Z_p$: Outside of (0,0), the remaining $p^2-1$ elements each have order p.  They are generate a group of size p, minus the identity.  So $(p^2-1)/(p-1) + 2 = p+3.$
\item Subgroup count of $Z_4 \times Z_2$: a counting exercise, based on generators. 
\begin{itemize}
\item Look at all cyclic groups of each of the elements.
\item (0,0) generates 1 group
\item Order 2: Three elements, which generate three distinct cyclic subgroups
\item Order 4: Four elements, which generate two distinct subgroups
\item Order 8: $Z_4 \times Z_2$, non-cyclic
\item And there's one distict $Z_2 \times Z_2$ group.
\item \emph{Note: Is there a good (even recursive) formula for this?}
\end {itemize} 

\end {itemize} 

\section {Chapter 2.4: Abelian Groups}


\subsection{Page 1-3}
\begin{itemize}
\item Theorem: $Z_a \times Z_b$ is isomorphic to $Z_{ab}$ iff a and b are relatively prime.
\item DF Proof: If a and b are relatively prime, (1,1) is of order $ab$.  If $a$ and $b$ share factor $c$, then $Z_{ab}$ has an element of order $ab$, but $Z_a \times Z_b$ will have cycled by $a * b / c$.
\item So decompose e.g. $Z_{12}$ into $Z_4 \times Z_3$, for example.
\end{itemize}

\subsection{Page 4-6}
\begin{itemize}
\item Theorem: Every finite abelian group is isomorphic to a direct product of cyclic groups.
\item Therefore, the number of these groups of order n is the product of the partitions of each of its prime factors' powers.
\item Therefore, the number of abelian groups of size $24 = 3* 2^3 = p(3) * p(1) = 3 * 1 = 3, Z_3  \times Z_8, Z_3  \times Z_4 \times Z_2, Z_3  \times Z_2 \times Z_2 \times Z_2$
\item Therefore, the number of abelian groups of size $2310 = 2 * 3 *5 * 7 * 11$ is one.
\end{itemize}

\subsection{Page 7-11: $Z_n^*$ or $U(n)$}
\begin{itemize}
\item Group $Z_n^*$: elements of $Z_n$ relatively prime to n, under multiplication. 
\item $|Z_n^*| = \phi(n)$ , the totient function.
\item This is a group even if n not prime because there is $ax + bn = 1$  if $x, n$ are relatively prime.
\item $Z_8^* = \{1,3,5,7\}$ is isomorphic to $Z_2  \times Z_2$ since every element squared is 1.
\item $Z_10^* = \{1,3,7,9\}$ is isomorphic to $Z_4$ since it is generated by 3.
\item $Z_15^* = \{1,2,4,7,8,11,13,14\}$ is isomorphic to $Z_4 x Z_2$ by counting element orders.
\item Note: Primitive roots of n are those that generate $Z_n^*$.  There are primitive roots mod n if and only if $n = 1,2,4,p^k, 2p^k$.
\item TODO: read https://brilliant.org/wiki/primitive-roots/ and why these are the only solutions. Also, look up \emph{Legendre symbol}
\end{itemize}

\section {Chapter 2.5: Homomorphisms}
\subsection{Page 1 - 6}
\begin{itemize}
\item Homomorphism $\phi: \phi(a) *' \phi(b) = \phi(a * b)$.  Note that  $*$ and $*'$ are different  operations.
\item This means, "translate each via the function, then combine" yields the same result as "combine first, then translate".  So structure is preserved.
\item Note this is like isomorphism, except homomorphism can squash some items to zero.
\item Also, this can change to an entirely separate domain, e.g. $det(AB) = det(A)det(B)$
\item Easy to prove homomorphism preserves identities and inverses.
\item Order of transformed element $\phi(g)$ divides order of g, since $g^k = e$ and $\phi(g)^k = \phi(e)$, but consider that $\phi(g)$ could hit $e$ at some divisor of $k$ - we could map everything to the identity and make that 1!
\end{itemize}

\subsection{Page 7- 10: Counting homomorphisms}
\begin{itemize}
\item Main idea: Knowing where we send identity determines entire homomorphism for a cyclic group.
\item Homomorphism count for $Z_4 \rightarrow Z_{10}$: There are 10 places to send identity, but recall that $\phi(1)$ has to have order 4 since $\phi(1 + 1 + 1 + 1) = \phi(0) = 0$.  Therefore, $\phi(1)$ has to be 0 or 5.  So 2 possibilities.
\item Homomorphism count for $Z_{99} \rightarrow Z_{100}$: Since $\phi(99) = 0$ and $\phi(1) \times 100 = 0$, and order of $\phi(1)$ must divide both, only one possibility: $\phi(1) = 1$,.
\item  Homomorphism count for $Z_{99} \rightarrow Z_{99}$: 99, since $99 \cdot \phi(1) = 0$, so $\phi(1)$ can go anywhere.
\item  Homomorphism count for $D_3 \rightarrow Z_3$: 1, since $D_3$ has 3 elements of order 2, 2 of order 3, 1 of order 1.  Only mapping everything to $0$ works.
\end{itemize}


\subsection{Page 11: Counting automorphisms}
\begin{itemize}
\item Automorphism is isomorphism from group to itself.
\item Count of automorphisms of $Z_8$: If 1 maps to an order-8 element, we're isomorphic.  There are four: ${1,3,5,7}$
\item $Aut(Z_8)$ is isomorphic to $Z_2 \times Z_2$, since $\phi_3(1)^2 = \phi_5(1)^2 = \phi_7(1)^2 = 1$, where $\phi_a$ maps a to 1.  Three elements of order 2 means it's the Klein 4 group.
\item Count of \emph{automorphisms} (meaning, we need all the elements in the codomain) of $Z_2 \times Z_2 \times Z_2$: Think of $\phi((1,0,0)), \phi((0,1,0)), \phi((0,0,1))$ as the basis for the group.  There are seven choices for the first, six for the next, and \emph{four} for the third.  
\item The above group is $(\phi(e_1) | \phi(e_2) | \phi(e_3)) = GL(\mathbb{F}_2)$, invertible matrices of 3x3.
\end{itemize}


\section {Chapter 2.6: Quotient Groups}
\subsection{Aside: Complex multiplication}
\begin{itemize}
\item Complex modulus (size) of $a+bi$ is defined as $root(a^2+b^2)$
\item Complex multiplication: Angles add, moduli multiply
\item One proof of moduli: $(a+bi)(c+di) = (ac - bd) + (ad+bc)i$
and $\sqrt{a^2+b^2}\sqrt{c^2+d^2} = \sqrt{a^2c^2 + b^2d^2 - 2abcd + ad^2+bc^2 + 2adbc}$
\item One proof of angles: Convert to $r_1(cos(a) + sin(a))r_2(cos(b)+sin(b))$ and multiply
\item More visual proof: Think of $c_1(a+bi) = c_1a + i(c_1b)$. $a$ scales original vector, and $bi$ rotates by 90 degrees and scales.
\end{itemize}

\subsection{Page 1-6}
\begin{itemize}
\item $S^1$, is defined as the group of complex numbers with modulus 1.
\item The coset $zS^1$ is any complex number multiplied by $S^1$ , which is a circle about the origin.  $z=2$ and $z=2i$ would be in the same coset.  These cosets are members of $C^*$ with the same modulus (length).
\item These are disjoint cosets that fill out $\mathbb{C}^*$ (don't include the zero, since no inverse).
\item If you consider $H=x+iy$, $x > 0, y=0$ (positive reals) then the cosets are rays from the origin.  Any $zH$ is just the different sizes of that (say, unit) vector.  These cosets are members of $C^*$ with the same angle.
\item \textbf{quotient group} of $\mathbb{C}^*$ by $S^1$:
	\begin{itemize}
	\item Members are cosets
	\item Multiplying is defined as $aH \times bH = abH, H \in S^1, a,b \in \mathbb{C}^*$ 
	\item $S^1$ is therefore the identity.
	\item This group is isomorphic to $R^+$ under multiplication (or really, like H).
	\item "A ray of angle A and a ray of angle B multiply to a ray of angle AB, forget about the size".
	\item This is like collapsing out the divisor, in this case, $S^1$.
	\item size $|G / H| = |G| / |H|$ since cosets are equally sized.
	\item \textbf{Gotcha}: Only works (meaning, $g_1, g_1' \in C_1, g_2, g_2' \in C_2$ implies $g_1g_2$ in same coset as $g_1'g_2'$ ) if H is \textbf{normal} in G.  
	\item Note: Normal means $xH = Hx$, so that makes sense that $g_1Cg_2C = g_1g_2C*C=g_1g_2C$
	\end{itemize}
\item So $\mathbb{C}^* / H$ is all the rays with the same modulus, or $S^1$.  
\item	"A ray of size X and a ray of size Y multuply to a ray of size XY, and forget about the angles".
\item So $\mathbb{C}^* / S^1 = H$ and $\mathbb{C}^* / H = S^1$!
\end{itemize}


\subsection{Page 7-12}
\begin{itemize}
\item Another example: $\mathbb{Z} / 10\mathbb{Z} = \mathbb{Z}_{10}$ under addition.  Forget about the non-unit digits!
\item Another example: $\mathbb{Q} / \mathbb{Z}$ is $\overline{q} = q + \mathbb{Z}$, so $\overline{1/2} + \overline{2/3} = \overline{1/6}$
\item Another example: if $N$ is the \textbf{center} (omni-commuter subgroup) of $D_4$, then N is two elements $I, R_{180}$.  Forgetting about those we have cosets $(I, R_{180})N, (R_{90}, R_{270})N, (D_1, D_2)N, (V,H)N$.  All non-identity are degree 2, so isomorphic to $Z_2 \times Z_2$
\item Another example: $Z_{13}^*$ with multiplication mod 13.  $N = {1,12}$ is a normal subgroup.  $Z_{13}^* / N$ is "forget about the +/- 1 of it and think of these as 1 through 6.  
\item Another example: \textbf{commutator subgroup} [a,b] is generated by all $aba^{-1}b^{-1}$ for all $a, b \in G$.  Note: group members are products of these guys, not necessarily all of that form.  This is just $e$ for an Abelian group.  Its size measures "how far" the group is from being Abelian.  
\item \textbf{Main idea} of quotients: "what do we force to the identity?"  If we say every $\overline{aba^{-1}b^{-1}} = \overline{1}$, then you can multiply by $ba$ to get $\overline{ab} =  \overline{ba}$.  So $G / [G,G]$ is necessarily Abelian.
 \end{itemize}


\section{Chapter 3.1: Number Theory}


\subsection{Page 1- 7}
\begin{itemize}
\item A Fermat's little theorem proof
 \begin{itemize}
  \item Take prime $p$, and $a$ not divisible by $p$.
   \item ${a, 2a, 3a ... , (p-1)a} \equiv {1, 2, 3, ... (p-1)}$ mod $p$ since they're the same elements mod p.
   \item Take the product of each: $a^{p-1}(p-1)! \equiv (p-1)!$ mod $p$
   \item Divide $(p-1)!$ out (there's an inverse mod p) and you get $a^{p-1} \equiv 1$ mod p
  \end{itemize}
 \item Another: Since the order of $a$ in $\mathbb{Z}_p^*$ is $p-1$, $a^{p-1} \equiv 1$ mod p.
 \item Note: Generalization of Fermat's little theorem using same group argument: $a^{\phi(n)} \equiv 1$ if $a$ and $n$ relatively prime.
 \end{itemize}
 
 \subsection{Page 8-11}
 \begin{itemize}
 \item Wilson's theorem: $1 * 2 * ... * (p-1) \equiv -1$ mod $p$.
 \item One proof: These all have inverses, except 1 and -1 mod p, which are self-inverting ($x^2 = 1$ solutions).
 \item This also proves that the product of all elements of a finite Abelian group \emph{which has a single element $g$ of order 2} is that element, $g$.
 \item A hard proof TODO. The powers of a \textbf{primitive root of p} yield all elements $a$ mod $p$.  So $\mathbb{Z}_p^*$ is cyclical for any prime p.
\item One more proof: if k relatively prime to p-1, where p a prime $> 2$, then $1^k + 2^k + ... + (p-1)^k \equiv 0$ mod $p$, since each of these summands is a different member of the group, summing to $\frac{p(p-1)}{2}$
 \end{itemize}
 
 \section{Chapter 3.2: Games}
 \subsection{15 puzzle}
 I think this will go:
 - The board is a permutation of (1, 2, ... 15), read like a book, with a blank  somewhere in there ,immaterial.
 - Sliding the blank left or right doesn't change the order.
 - Sliding it up or down skips three backward or forward.
 
\textbf{Their proof}: Think of this as a series of swaps with $(j, 16)$, 16 being the blank tile.  To return to the bottom right corner, 16 must make an even number of moves.  So only even permutations allowed.  So (14,15) is not a viable swap, nor any of the odd permutations.  

\section{Chapter 3.3: Peg solitaire}

\begin{itemize}
\item Consider Klein four group: $xy = yx = z, yz = zy = x, xz = zx = y$.  
\item Label all pegs such that three consecutive are always, in some order: x, y, z
\item Invariant: product of all occupied spaces.  If x jumps over y to get to z, eliminating jumped peg, xy = z.
\item 11  x's, 11 z's, 10 y's yield xz = y as the product.
\end{itemize}

\section{Chapter 3.4: Rubix's Cube}

\begin{itemize}
\item Each element is the state $(S_{12}, S_8, (Z_2)^{12}, (Z_3)^8$), representing around a fixed set of centers: (middle selections, corner sleections, middle orientation, corner orientation).
\item Invariant: First and second perms for all F,B,D,U,L,R are odd, so first two args need same permutation parit
\item Invariant: (Not proven here): Sum of edge orientations (0,1) is zero, sum of corner orientations (0, 1, 2) is zero.
\item \textbf{Commutator}: $ghg^{-1}h^{-1}$ measure how entangled $g$ and $h$ are.  If they're commutatitive, it is $e$.
\item For Rubix's cube, commutators  $ghg^{-1}h^{-1}$ are great for only moving pieces where effects of $g$ and $h$ overlap.
\item $g$ and $h$ are \textbf{conjugates} if some x such that $h = x^{-1}gx$.   "h is same as g, just in a different location".
\item Conjugate interpretation: "h is move via x, operate with g, move back via x.  "\
\item For Rubix's cube - you can use conjugates to make whatever change to a different part of the cube (move it to the operating table, operate, move it back).
\end{itemize}

\section{Chapter 4.1: Normal Subgroups}

\subsection{Normal definition}

\begin{itemize}
\item \textbf{Normal subgroup intuition}: Every conjugacy $g^{-1}Hg$ moves a group to another subgroup.  Normal subgroups  $g^{-1}Ng = N$ are the ones \emph{that don't move} when you conjugate them.
\item Example of non-normal: Any one of the $n$ sets of $S_{n-1}$ among conjugates of $S_n$.     Move it, mess with it, move it back - it's broken free by then.
\item Normal definition: Group N is normal if and only if (all equivalent):
	\begin{itemize}
	\item $gN = Ng$ for all $g \in G$
	\item $gNg^{-1} = N$ for all $g \in G$ (equiv to above)
	\item $gng^{-1} \in N$ for all $g \in G$
	\end{itemize}
\item \emph{Theorem: Any subgroup of index 2 is normal}.  Proof: G has two distinct cosets $N$, $gN$, but also $N$ and $Ng$ so $gN = Ng$.
\item Normal doesn't recursively nest.  
	\begin{itemize}
	\item If $G$ has normal subgroup $H$ and $H$ has normal subgroup $K$, $K$ is normal in $H$ too (those elements also "pass through K)"
	\item However, $H$ can be normal in $G$ (e.g. $(I, R_{180}, F_v, F_h)$ in $D_4$,  $K$ can be normal in $H$ (e.g. ${I, V}$, but $K$ is not normal in $G: VR_{90} = D_{ul}, R_{90}V = D_{ur}$
	\end{itemize}
\item Normal examples in $GL_2(\mathbb{C})$: $SL_2(\mathbb{C})$ (determinant 1) and non-zero diags $zI_2$.  
\item Non-normal examples in $GL_2(\mathbb{C})$: $GL_2(\mathbb{R})$ and non-zero diags with different entries. Easy to throw some arbitrary ones in Wolfram Alpha and see everything messed up after conjugation.
\item G's \textbf{Center}: $Z(G)$ are the omni-commuters.  Always normal.
\item G's \textbf{Commutator group $[G,G]$}: Product of any $aba^{-1}b^{-1}$ for $a, b \in G$. is normal, since $g[a,b]g^{-1} = [gag^{-1}, gbg^{-1}]$.
\end{itemize}

\subsection{Normal properties and examples}
\begin{itemize}
\item $S_3$ has three normal subgroups: two trivial ones, and $([], [123], [321])$ since it's of index 2.
\item $Q_8$ has four non-trivial subgroups, all normal: those generated by $I, j$, or $k$, all of order 4, index 2.  $-1$ also generates an order 2 group, but it's the center.
\item Definition: Product $HK = {hk: h \in H, k \in K}$. 
\item Property: If $H \cap K$ = $\{1\}$, and $H, K$ are finite, $|HK| = |H| \cdot |K|$.  Why?  
  $h_1k_1 = h_2k_2 \Longrightarrow h_2^{-1}h_1 = k_1^{-1}k_2$, proving they're both $e$ since left is in $H$, right in $K$.
\item Property : If $H, K$ subgrops of G, then $HK$ is a subgroup too if $H$ or $K$ is normal, otherwise not always.  Why?   
  \begin{itemize}
  \item Assume H is normal.
  \item Identity: $e_he_k = e$ is in there.
  \item Inverse: If $hk \in HK$, then $k^{-1}h^{-1} = k^{-1}h^{-1}k^1 * k^{-1}$ is in H, K due to H's normality.
  \item Closure: $h_1k_1 * h_2k_2  = h_1k_1h_2(k_1^{-1}k_1)k_2 =   h_1 (k_1h_2k_1^{-1}) k_1k_2 =  h_1h_3 * k_1k_2$ for some $h_3$
  \end{itemize}
\item Property: If $H, K$ are normal subgroups of G, $HK$ is normal.  Maybe not otherwise (e.g. take $H = \{1\}, G$ a non-normal subgroup).  Why? More tricks.  $ghkg^{-1} = gh(g^{-1}g)kg^{-1} =(ghg^{-1})(gkg^{-1}) = h'k'$ for some other $h' \in H, k' \in K$.

\item \textbf{Centralizer} of G's subgroup H is a subgroup of G which commutes with all H: $C_G(H) = \{g \in G: gh = hg$ for all $h \in H\}$.  This is G if and only if G is Abelian (almost definitional).  May not contain H.
\item \textbf{Normalizer} of G's subgroup H is a subgroup of G which makes H normal: $N_G(H) = \{g \in G: gH= Hg\}$.  This is G if and only if H is normal in G (almost definitional).  Largest subgroup of G where H is normal.
\item Centralizer is  a normal subgroup of normalizer with two different proofs:
\begin{itemize}
\item With $n \in N_G(H), c \in C_G(s)$, show that $ncn^{-1}$ commutes with members of H, so it's in $C_G$, therefore normal.  $hn$ is some $nh'$, and same for $n^{-1}$, so $ncn^{-1}h = nch'n^{-1} = nh'cn^{-1} = h'ncn^{-1}$ so $ncn^{-1}$ passed through h, is therefore in the centralizer, and so $C_G(H)$ is normal.
\item Using First isomorphism theorem (later): 
  \begin{itemize}
  \item $N_G(H)$ is the big "dividend" group, $C_G(H)$ is the "divisor", and Aut(H) the "quotient" (codomain of the homomorphism)
  \item The homomorphism $\phi: N_G(H) \rightarrow Aut(H)$ is $g \rightarrow \phi_g(x) = gxg^{-1}$.
  \item The kernel of this homomorphism is that which maps to $I \in Aut(H)$.
  \item The kernel is the centralizer, since $\phi_c(x) = cxc^{-1} = cc^{-1}x = x$, identity.
  \item Therefore, $N_G(H) / Ker(\phi) = N_G(H) / C_G(H) \rightarrow Aut(H).$ so $C_G(H)$ must be normal!
  \item (Kernels of homomorphisms always normal (DSF Proof): If $\phi: G \rightarrow H$ is a homormophism, and $g \in G, k \in Ker(\phi)$, then $gkg^{-1} \in K$ since $\phi(gkg^{-1}) = \phi(g)\phi(k)\phi(g^{-1}) = \phi(g)\phi(g^{-1}) = e$.  So $K$ is normal in $G$.
  \end{itemize}
\end{itemize}

\end{itemize}

\section{Chapter 4.2: Isomorphism theorems}
\begin {itemize}
\item Example of intuitive isomorphism: $M_2(\mathbb{Z})/N \cong (\mathbb{Z}_2)^4$, where N is the subgroup with even entries.   How?  Can \emph{either list all cosets} or construct a homomorphism $\phi
   \begin{pmatrix}
    a     & b \\
   c     & d \\
  \end{pmatrix}
    = (a ($mod$2), b ($mod$2), c ($mod$2), d$ mod $ 2)).$

\subsection{First Isomorphism Theorem and example}
  \begin{itemize} 
  \item $G = GL_2(\mathbb{R})$, invertible 2x2 real matrices
  \item $N = SL_2(\mathbb{R})$ is subgroup of $G$ with determinant 1.
  \item $\varphi$ is $det$, since $det(AB) = det(A)det(B)$.
  \item $G / N \cong \mathbb{R}^*$ intuitively, since for any matrix, you can divide by the determinant scalar, and find the representative in the group N.   Can think of N as the kernel of the homomorphism - it doesn't matter, it's mapped to identitty.
  \item \textbf{First isomorphism theorem}: given surjective homomorphism $\varphi: G \mapsto H$ with kernel $Ker(\varphi) = \{g \in G |\varphi(g) = e_H\}$, then $G / Ker(\varphi) \cong H$. 
  \item Another example in the above, if 
    $a= \begin{pmatrix} 0 & 1 \\ 2 & 3 \\   \end{pmatrix},
     b= \begin{pmatrix} 1 & 2 \\ 3 & 4 \\   \end{pmatrix},$ then $(aN)(bN)$ is some $cN$, where $det(c) = 4$, like $2I$
  \end{itemize}
  \end{itemize}

\subsection{Third Isomorphism theorem}
\begin{itemize}
\item Theorem: If G / N is abelian, then every subgroup H of G containing N is normal in G.
  \begin{itemize}
\item $H / N \subset G / N$, and so $H / N$ is abelian too.
\item Abelian means $ghN = hgN$
\item This also shows there is some $n$ such that $gh = hgn$.
\item But since N is normal in G, $gn = n'g \rightarrow hgn = (hn')g$, and $hn' \in H$, therefore $gh = (hn')g$, and H is normal in G.
\end{itemize}
\item Actual theorem says subgroups of G containing N correspond to subgroups of G/N. 
\item Also, $\frac{G/N}{H/N} \cong \frac{G}{H}$
\end{itemize}

\subsection{Second Isomorphism theorem}
\begin{itemize}
\item Actual theorem says: if $H$ is a subgroup of $G$, and $N$ is a normal subgroup of $G$, then $\frac{H}{H \cap N} \cong \frac{HN}{N}$
\item In particular, if $H \cap N = \{1\}$, then $\frac{HN}{N} \cong H$.  
\item Why?  
\item $HN$ contains both H and N since N is normal.
\item Therefore $(HN)/N$ is a group.
\item $\varphi(h) = hN$ is a surjective homomorphism to $(HN)/N$
\item The kernel is anything in N, which would be $H \cap N$.
\item Result follows from first isomorphism theorem.
\end{itemize}

\subsection{Examples using the first isomorphism theorem}
\begin{itemize}
\item: Typically, in order to identify $G/N \cong K$, find the surjective homomorphism $G \rightarrow K$ where $Ker(\varphi) = N$.
\item Example: $G = \mathbb{Z} \times \mathbb{Z}$ with addition, N = group generated by $(1,0)$.  $G/N \cong Z$ intuitively, since you're forgetting the first coordinate.  To make it formal : $\varphi((x,y) = y)$.  
\item Harder example: $G = \mathbb{Z} \times \mathbb{Z}$ with addition, H = group generated by $(2,3)$, or $(2a, 3a)$.  $G/H \cong Z$, actually, since $\phi((x,y) = 3x - 2y)$ is surjective (think of $\phi((a, a)) = a$ and its kernel is H.
\item Harder example: $G = \mathbb{Z} \times \mathbb{Z}$ with addition, H = group generated by $(2,4)$, or $(2a, 4a)$.  $G/H \cong Z \times Z_2$, actually, since $\phi((x,y) = 2x  - y, x (mod 2))$ is surjective and its kernel is H.
\item TODO: Get a better intuition here.  Is this group like, how far away from this null space line am I?
\end{itemize}

\section{4.3a: Interlude; Group actions}

\subsection{Group Actions}
Reference: https://brilliant.org/wiki/group-actions
\begin{itemize}
\item \textbf{group action} on group $G$, set $X$, is function $f: G \times X \rightarrow X$.  It's often written $f(g,x) = g \cdot x$. which has some groupy properties.
  \begin{itemize}
  \item $f(e_G, x) = x$ for all $x \in X$, or $e_G \cdot x = x$
  \item $f(g, f(h, x)) = f(gh, x)$ for all $x \in X$ ,or $g \cdot (h \cdot x) = (gh) \cdot x$.
  \item Canonical Example: if G is $S_n$, and $X = \{1, 2, ... n\}$.  
  \end{itemize}

\item \textbf{fixed point of a group element} $g \in G$ is $x \in X$ such that $g \cdot x = x$.  So, $f = g(x)$ is the (very straightforward) mapping, $g$ is the function, and $x$ would be a point that doesn't change.
\item For point $x$, \textbf{stabilizer of the point} is called $G_x$, and is the set of $g \in G$ that map $x$ as a fixed point: $g(x) = x$.of a of element $g \in G$ is $x \in X$ such that $g \cdot x = x$.  So, it's the \emph{subgroup} that makes $x$ totally stable.
\item \textbf{fixed point} of element $g \in G$ is $x \in X$ such that $g \cdot x = x$.  So, $f = g(x)$ is the (very straightforwardx) mapping, $g$ is the function, and $x$ would be a point that doesn't chagne.
\item \textbf{orbit of element $x \in X$} is how far $x$ reaches, the set of $y \in X$ such that there's a $g \cdot x = y$.
\item Example: So if $G = \mathbb{Z}_2 = {e, g}, X = \mathbb{Z}$, and the action is $e \cdot x = x, g \cdot x = -x$, then
  \begin{itemize}
  \item Fixed points of $e$ are all of them, of $g$ is 0.
  \item Stabilizers of $x$ are $e$ for all, $e,g$ for 0.
  \item Orbit of 0 is $\{0\}$, orbit of every other $n$ is $\{n, -n\}$
  \item \emph{orbits} are an equivalency relation!  So they partition $X$.
  \end{itemize}
 \item Action is \textbf{transitive} if there is only one orbit in the relation (sounds like a regular group): for any $x, y \in X$, there is a $g$ such that $g \cdot x = y$.
 \item Action is \textbf{faithful} If only $e_G$ if the only omni-stabilizer element is $e_G$.  Intersection of all $G_x$ is $e_G$.
 \item Another way to think about faithful: Think of $G$ as a homomorophism to $Sym(X)$, permutations of the group.  Faithful actions are injective / have a trivial kernel.
\item Examples of actions 
  \begin {itemize}
  \item Every group acts on itself by left multiplication.  It is transitive and faithful (since the Cayley table is a latin square).  One orbit.
  \item Every group acts on itself by conjugation $g \cdot x = g x g^{-1}$. Orbits are the conjugacy classes.  The \textbf{centralizer} $C_G(x)$ is the stabilizer of $x$.
  \item If $H$ is a subgroup of $G$, then cosets $G / H$ and left multiplication are a group action.  They are a transitive action since there is one orbit: you can always get from $gH$ to $kH$ by $(kg^{-1})H$.
  \item \textbf{Core Group}: The group $\bigcap_{g \in G} gHg^{-1}$ of G's subgroup H is the largest normal subgroup of H.  Proof:
  \begin{itemize}
  \item It's contained in H since $hHh^{-1} \in H$, and it's an intersection.
  \item It's normal because $kCore_G(H)k^{-1} = k (\bigcap_{g \in G} gHg^{-1}) k^{-1}  = \bigcap_{g \in G} kgHg^{-1}k^{-1} = Core_G(H)$ since every $kg$ is just a $g$ but permuted. 
  \item It's the largest normal one since if there were another normal subgroup $N' \in H$, then $gn'g^{-1} \in N'$ and $gn'g^{-1} = h$ for some $h \in H$, so $n' = g^{-1}hg$, and therefore $n'$ is somewhere in the core group.
  \item Therefore, the core group is the kernel of the map $G \rightarrow Sym(G/H)$, since those map to H.  So if H doesn't contain any trivial subgroups, it's faithful, and is called \textbf{simple}
  \end{itemize}	  
  \item Another group action: $PGL_2(\mathbb{C})$ = projective linear group of 2x2 matrices on the complex plane (plus infinity), sending $\begin{pmatrix} a & b \\ c & d\\ \end{pmatrix} \cdot z = \frac{az+b}{cz+d}$.  
  
  \end{itemize}
  

\item \textbf{Orbit stabilizer theorem}: If $G$ is finite, and $x \in G$ has a stabilizer $G_x$ and orbit orb(x), then $|G| = |G_x| |orb(x)|$.  Proof:
\begin{itemize}
\item Since stabilizer is a subgroup, the count of distinct cosets (index) times the subgroup is the size by Lagrange. 
\item Consider homomorphism $\phi$ from $G / G_x \rightarrow orb(x) = gG_x \rightarrow g \cdot x$
\item And the set $aG_x$ and $bG_x$ are equal under $\phi$  iff $a(x) = b(x)$, since $b^{-1}aG_x= G_x$, implying $b^{-1}a \in G_x \rightarrow b^{-1}a(x) = x \rightarrow a(x) = b(x)$.  
\item Also, this map is onto since every element $y \in orb(x)$, meaning some $g \cdot x = y$ is in that $gG_x$ .
\item Example: \emph{symmetric group}: $S_n: G_x \cong S_{n-1}! \rightarrow |G_x| = (n-1)!$ .  $orb(x) = n$.  So $|G| = n!$.
\item Example: \emph{cube symmetries}: Vertex is $x$, rotation of adjacent vertices is $G_x$. $|G_x| |O_x| = 3* 8 = 24$.  Can also do with edges and faces.  Turns out cube symmetries $\cong S_4$
\end{itemize}
\end{itemize}


\section{Aside: Conjugacy classes}
\subsection{Conjugacy classes defined}
\begin{itemize}
\item Note: It's easy to take a group to another group by conjugation group action $\phi(g, H): ghg^{-1}$.  Though the whole group gets mapped to another group, the elements inside get mapped to \textbf{conjugacy classes}.
\item Within group G, elements h, h' in conjugacy class H have some $g \in G$ such that $h' = ghg^{-1}$ (and therefore,  $g^{-1}h'g = h$.  So, it's an equivalence relation, thus a partition.
\item Note: if the group G is abelian, $h' = ghg^{-1} \rightarrow gg^{-1} h = h$, so all conjugacy classes there are of size one ($ghg^{-1} = gg^{-1}h = h)$
\item Each one of these classes corresponds to the orbit of that element $h$ under conjugation.  
\item Why useful?  They can be used to show structure (and thus classify, look at isomorphisms, etc.) of groups.
\item Example: In $GL_n(\mathbb{R}), A = PBP^{-1}$ is matrix similarity.  $B$ represents $A$ under a change of bases.
\item Example: In $S_3$, there are three conjugacy classes: $\{()\}, \{(abc), (bac)\}, \{(ab), (bc), (ac)\}$.  Easy to think about with permutations - this is just relabeling the members going in, doing the permutation, then reversing the labels.
\item Example: $\mathbb{Z} / 5\mathbb{Z} = \mathbb{Z}_5$ which is Abelian, so 5 classes (one for each element).
\end{itemize}

\subsection{$A_5$ example and the class equation}
\begin{itemize}

\item Examples: In $A_5$, types are repped by $(), (12)(34), (12)(23) = (123)$, and $ (12)(23)(34)(45) = (12345)$.  
\item \textbf{Gotcha}: Note that There are $5! / 5 = 24$ 5-cycles, and that subgroup order has to divide 60, so there must be two conjugacy classes of 5-cycles.  Makes sense that $(12345)$ and $(21345)$ can't be same class, since there's nowhere to "stash" during the relabeling.
\item Theorem: Sum of conjugacy class orbits is size of group, or, for arbitrary class reps $g_1...g_k, |G| = \sum_{i=1}^k [G:C_G(g_i)]$.  Note that $|Z(G)|$ is all of the reps and classes of size one.  Why does this work?  Orbit-stabilizer says that $C_G(g_i)$ is the stabilizer, and the conjugacy classes form a partition.
\item \textbf{Class Equation}: Just writing down the size of the equivalence classes.  In $A_5$, this would be $60 = 1 + 15 + 20 + 12 + 12$ (second set of 5-cycles).
\item Note that any \emph{normal} subgroup in $A_5$ has to be union of those since conjugation by $A_5$ elements maps to the whole conjugacy class.  BUT - there are no sums that divide 60.  So $A_5$ has no normal subgroups, so it is simple!
\end{itemize}

\section{4.3 Conjugacy class section}
\begin{itemize}
\item Reminder of orbit-stabilizer theorem: Number of conjugates of $g$ is the index (more generally, orbit, here under  conjugation group action) of the centralizer (more generally, stabilizer, here under conjugation group action) of $g$.
\item Example: If $|G| = 60, g \neq 1, g^5 = 1$ then size of conjugacy class is at most 12 since at least $e, g, g^2, g^3, g^4$ are in its centralizer $C_g(G)$.
\item Example: Class equation of $Q_8$:
  \begin{itemize}
  \item $\{1\}$ commute with everything = 8 / 8 = 1
  \item $\{-1\}$ commutes with everything = 8 / 8 = 1
  \item $i$ commutes with $1, -1, i, -i$ = 8 / 4 = 2.  Same for j, k,
  \item Thus, 8 = 1 + 1 + 2 + 2 + 2.
  \end{itemize}
\item Example: Class equation of $D_5$, with $\sigma$ as a clockwise rotation, $\tau$ as a flip:
  \begin{itemize}
  \item $\{e\}$ commutes with everything = 10 / 10 = 1
  \item $\{\sigma\}$ commutes with only rotations = 10 / 5  = 2, and conjugates to $\tau\sigma\tau = \sigma^4$.
  \item $\{\sigma^2\}$ commutes only with rotations = 10 / 5 = 2, and conjugates to $\tau\sigma^2\tau = \sigma^3$.
  \item $\{tau\}$ commutes only with $\tau, e$, and all five flipped actions $\sigma^k\tau$ are conjugate. = 5.
  \item Thus, 10 = 1 + 2 + 2 + 5
  \end{itemize}
\item Weird \textbf{gotcha}: Note that the abelian $\mathbb{Z}_5$ has 5 conjugacy classes, and $D_5$ has four, and there's an injective homomorphism $z \rightarrow \sigma^z$.  So even though $Z_5$ is a subgroup of $D_5$, it has more conjugacy classes.
\item $D_n$ has $n$ reflections.  If $n$ is odd, there is only one conjugacy class of reflections, since $(\sigma^i\tau)\tau = \sigma^i$ and $(\tau\sigma^i)\tau = \tau\tau\sigma^{-i}$, so if the paranthesized items are equal (i.e. if $\sigma^i$ commutes with $\tau$), then $\sigma^i = \sigma^{-i}$.  $i=0$ works, but only in even groups does $i=\frac{n}{2}$ work.  Therefore centralizer has size 2 for n odd, 4 for n even, and for these n/2 elements, there is one conjugacy class if n odd, 2 if even.
\item \emph{Theorem}: If there's a homomorphism $\pi: G \rightarrow K$, then count of  conjugacy classes $c(G) \geq c(K)$.  Homomorphism maps conjugacy classes to conjugacy classes $\phi(ghg^{-1}) = \phi(g)\phi(h)\phi(g)^{-1}$, so if there's a nonzero kernel with $k$ in it, $\pi(1) = \pi(k)$, but 1 and k could be different conjugacy classes in the domain.

\end{itemize}

\section{4.4 Permutations / Symmetric group}
\begin{itemize}
\item Note: Every group of size $n$ is a subgroup of $S_n$, since element $g$ induces a permutation on the elements by multiplication.  I suppose then the group is the permutations of each $g$!  Repping under $S_n$ is called the \textbf{regular representation}.
\item Conjugation is interesting in $S_n$; If $\sigma = (123), \alpha = (13524)$, then $\sigma(13524)\sigma^{-1} = (\sigma(1)(\sigma(3)\sigma(5)\sigma(2)\sigma(4))$.  Why?    (Proof)
\begin {itemize}
\item Say $\alpha = (a_1 a_2 ... a_n)$
\item $\sigma^{-1}\sigma a_1 = a_1$
\item $\alpha(a_i) = (a_{i+1 mod n})$ 
\item So for any $a_i, \sigma \alpha \sigma^{-1}(\sigma (a_i)) = \sigma (a_{i+1 mod k})$
\item So the $\sigma \alpha \sigma^{-1}$ operation on $\sigma(a_i)$ is just like taking $\sigma(a_i)$ and mapping it to the next $\sigma(a_{i+1 mod k})$.
\end {itemize}
\item $S_6$ has 11 conjugacy classes corresponding to partitions: $(), (1 2), (1 2 3), (1 2)(3 4), (1 2 3 4), (1 2 3 4 5), (1 2 3)(4 5), (1 2 3 4 5 6), (1 2 3 4)(5 6), (1 2 3)(4 5 6), (1 2)(3 4)(5 6)$.  Think of missing elements x, y, like $(x)(y)...$.
\item How many permutations fix 1 in $S_n$ ? Clearly this is just $|S_{n-1}| = (n-1)!$
\item Summing total fixed counts $\sum_{\sigma \in S_n} F(\sigma)$ of every permutation is then $n*(n-1)! = n!$
\item Random note: $A_4 \ncong D_6$ since $A_4$ since there's an element of order 6 in $D_6$, none in $A_4$.
\item Tetrahedon rotations group: Isomorphic to $A_4$.  all rotations of form (1)(234) = (23)(34), (2)(13)(34), etc.  Four places to map a vertex, and three spin locations = order 12 (or orbit-stablizer: three rotations in vertex centralizer, four places to go with vertex in orbit).
\end{itemize}

\section{Aside: Legendre symbol}

https://brilliant.org/wiki/legendre-symbol/

\begin{itemize}
\item $a$ is a \textbf{quadratic residue} mod $m$ if $x^2 \equiv a$ mod $m$ has at least one $x$ solution.  So, I suppose that 1 is always a quadratic residue.    $a$ and $m$ need to be coprime.
\item If $p$ is an odd prime, $a$ is an integer, Legendre symbol $\Big(\frac{a}{p}\Big)$ is:
 \begin{itemize}
 \item 0 if $a \equiv 0$ mod p
 \item 1 if $a$ is a quadratic residue mod p and $a \neq 0$ mod $p$.
 \item -1 if $a$ is a non-residue 
 \end{itemize}
 \item Sum of quadratic resides of a prime is 0.  Why?  There are no 0's, and every residue is repped twice, once by $a$ and once by $p-a$.  So half are non-residues, half are double-residues.  Why do they pair this way?  $a^2 \equiv b^2 mod p \Rightarrow a^2 - b^2 \equiv 0 mod p \Rightarrow (a-b)(a+b) \equiv 0 mod p \Rightarrow p | (a-b)(a+b) \Rightarrow a+b \equiv 0$ or $a-b \equiv 0$.
 \item Property:  \textbf{Euler's criterion}: If $p$ is an odd prime, $a$ is not divisible by $p$, then $a^\frac{p-1}{2} = \Big(\frac{a}{p}\Big)$ (mod $p$).  This follows from: (1) If $a = x^2$, then $x^{p-1}$ mod $p \equiv 1$ by Fermats, so take square root.  (2) If not, then because $Z_p$ is a group, every element $x$ has a pal $x^{-1}a$ that multiplies to a.  Product of these is  $a^\frac{p-1}{2} = (p-1)! = -1$ by Wilson's Theorem.
\item Property:  If $a \equiv b$ (mod $p), \Big(\frac{a}{p}\Big) =   \Big(\frac{b}{p}\Big)$.  Just reduce mod p.
\item Property  $ \Big(\frac{ab}{p}\Big) =   \Big(\frac{a}{p}\Big)   \Big(\frac{b}{p}\Big)$.  Follows from Euler's criterion ands exponents.
\item Property:   $\Big(\frac{-1}{p}\Big) = (-1)^\frac{p-1}{2}$, by Euler's criterion, so it is 1 iff $p \equiv 1$ mod 4.
\item Property:   $\Big(\frac{2}{p}\Big) = (-1)^\frac{p^2-1}{8}$, by \textbf{TODO} something called quadratic reciprocity.
\item Property:  If $p, q$ distinct odd primes, then $\Big(\frac{q}{p}\Big) \Big(\frac{p}{q}\Big)  = (-1)^\frac{p-1}{2}\frac{q-1}{2}$, by \textbf{TODO} something called quadratic reciprocity.


\end{itemize}


\section{4.5 Signs of Permutations}
\begin{itemize}
\item Note: Cycle structure of $\sigma_3(x) = 3x (mod 11): 3^5 = 1, and 2*3^5=2$.  Observe these two disjoint 5-cycles.
\item If $a^k = 1 mod p$ and is the smallest to do so, cycle structure of $\sigma_a(x) = ax mod p$ is all disjoint k-cycles.  \emph{Proof}: (1) $\sigma_a$ will have no fixed points, as $ax = x$ means $a = 1$ (mod $p$).  (2) $\sigma_{a^k} = \sigma_a^k = $ identity.  And (3) if $j < k$, $j$ can't be identity.   So $\sigma_a$ is the product of $\frac{p-1}{k}$ disjoint k-cycles.
\item Also implies that $\sigma_a$ is odd if and only if k is an even number (thus odd cycle) and $\frac{p-1}{k}$ is odd.
\item \textbf{Theorem}: $\Big(\frac{a}{p}\Big)  = -1$ iff k is even, and $\frac{p-1}{k}$ is odd, or $sgn(\sigma_a) = \Big(\frac{a}{p}\Big)$. Why? Suppose for some $a$, the primitive root $g$ taken to $x$ is $a: g^x = a$.  The order of $g^x$ is $\frac{p-1}{gcd(p-1,x)}$.  Then flip the denoms: $\frac{p-1}{k} = gcd(p-1,x)$, which is odd iff $x$ is odd, or NOT A SQUARE.  Therrefore, $sgn(\sigma_a) = \Big(\frac{a}{p}\Big) $!
\item An \textbf{inverison} in a permutation is where a pair$a < b$, $\sigma(a) > \sigma(b)$.  
\item Number of inversions in $\sigma_2$ is straightforward, as for prime p, $\sigma(1, 2, 3 ... p-1/2, p+1/2 .... p-1) \rightarrow (2, 4, 6  ... p-1, 1, .... p-2)$ ends up as $ 1 + 2 + ... + \frac{p-1}{2} = \frac{1}{2} \frac{p-1}{2}\frac{p+1}{2} = \frac{p^2-1}{8}$
\item The sign of a permutation is also $(-1)^r$, where r is number of inversions.
\item Putting all this together yields $\Big(\frac{2}{p}\Big)  = sgn(\sigma_2)$ by \textbf{theorem} above, $ = (-1)^r = (-1)^\frac{p^2-1}{8}$, property 5 in the last section.
\end{itemize}



\section{5.1 Group actions}
\subsection{Orbit-stabilizer}
\begin{itemize}
\item Canonical: Group $S_n$ acts on elements $X = {1, 2, 3 .. n}$.  $G \times X \rightarrow X$
\item Also canonical: Group acts on its own elements with left-multiplication, always.  $G \times G \rightarrow G$.
\item \textbf{orbit $O_x$} of element $x$ is all the places $x$ could go.  Note that in a group there is only one orbit (called \textbf{transitive}).
\item \textbf{Orbit-stabilizer theorem} says, for any $x \in G$, $|G| = |O_x| |G_x|$.
\item item \textbf{stabilizer} $G_x$ of element $x$ are the elements mapping $x$ to itself.  Note that in a group this is necessarily $G_x = {e}$ since $g \cdot x = x \rightarrow g \cdot x \cdot x^{-1} = x \cdot x^{-1} \rightarrow g = e$.
\item Example: $2n = |D_n|$, and since every \emph{vertex} element $x$ can be rotated to any other (orbit is size X), stabilizer must be of size 2 (identity, 180 flip)
\item Example: Rotations of a dodecahedron: Think of the faces - there are five rotations that fix the face, and the face can go to 12 different spots,so size of the group is 60. Turns out, also isomorphic to $A_5$.
\end{itemize}

\subsection{Action of $GL_2(F)$ on $F^2$}
\begin{itemize}
\item Action is on left-multiply: $A \cdot \begin{pmatrix} x \\ y \\ \end{pmatrix} = A \begin{pmatrix} x \\  y \\ \end{pmatrix}$
\item How many orbits in $\mathbb{R}^2$ under this action? 
\item  \emph{One orbit}: The point $\begin{pmatrix} 0 \\  0 \\ \end{pmatrix}$ can only map to itself, and no non-zero determ can map to it ($A \begin{pmatrix} x \\  y \\ \end{pmatrix} = \begin{pmatrix} 0 \\  0 \\ \end{pmatrix} \rightarrow \begin{pmatrix} x \\  y \\ \end{pmatrix} =  A^{-1} \begin{pmatrix} 0 \\  0 \\ \end{pmatrix}$, which only works for zero x,y or a zero-deteminant matrix.
\item \emph{The other orbit}: There's some invertible $A$ to match any $\begin{pmatrix} x \\  y \\ \end{pmatrix} = A \begin{pmatrix} 1 \\ 0 \\ \end{pmatrix} $, either $\begin{pmatrix} a & 1 \\ b & 0 \end{pmatrix}$ if $y \neq 0$ or $\begin{pmatrix} a & 0 \\ 0 & a\end{pmatrix}$ if $y = 0$. 
\item Example: $GL_2(\mathbb{Z_p})$ acts on $\mathbb{Z_p}^2$, just on integers modulo prime $p$p.
\item Orbit of $\begin{pmatrix} 1 \\ 0 \\ \end{pmatrix} $ is every non-zero element, so size $p^2-1$.  Stabilizer is anything $\begin{pmatrix} 1 & b \\ 0 & d \end{pmatrix}$ with $d \neq 0$, so $p^2 - p$ elements.
\end{itemize}


\subsection{General group action properties}
\begin{itemize}
\item Action is \textbf{regular} if $x, y \in X$ have exactly one  $g \in G$ so $g \cdot x = y$.  So, this means
  \begin{itemize}
  \item There's one orbit, since any $x$ can get to any $y$.
  \item Every element's stabilizer is just the identity (uniqueness).
  \item $|G| = |X|$ since $|G| = |O_x| |G_X| = |X| * 1$
  \item Really, any such regular action is isomorphic to $(G , G)$ by left-multiplication.
  \end{itemize}
\item If $x, y$ in the same orbit in G ($g \cdot x = y$ for some $g \in G$) for finite G, then $|G_x| = |G_y|$. Why?  First, because of the orbit-stabilizer theorem (same orbit size, same group size).  But also the "conjugating" bijection $f(h) = ghg^{-1}$, since $f(y) = ghg^{-1}(y) = gh(x)$ (since $h \in G_x$), $ = gx = y$.  Can reverse it too.
\end{itemize}



\end{document}