\documentclass[11pt, oneside]{article}   	% use "amsart" instead of "article" for AMSLaTeX format
\usepackage{geometry}                		% See geometry.pdf to learn the layout options. There are lots.
\geometry{letterpaper}                   		% ... or a4paper or a5paper or ... 
\usepackage[parfill]{parskip}    		% Activate to begin paragraphs with an empty line rather than an indent
\usepackage{graphicx}				% Use pdf, png, jpg, or eps§ with pdflatex; use eps in DVI mode
								% TeX will automatically convert eps --\rangle pdf in pdflatex		
\usepackage{amssymb}
\usepackage{amsmath}

\usepackage{verbatim}
\usepackage{tikz} 

\usepackage{syntonly}
% \syntaxonly \langle -- use this for checking syntax only
% \mbox {text} - keep together
% \fbox {text} - keep together and draw around

%\pagestyle{plain|headings|empty} % header and footer p.27
%SetFonts
%\include{filename}, \includeonly{filename1, filename2} , \input[fiename}

%SetFonts% 

\title{Brilliant: Vector Calculus}
\author{Dave Fetterman}
\date{6/21/22}							% Activate to display a given date or no date

\begin{document}
\maketitle
Note: Latex reference: http://tug.ctan.org/info/undergradmath/undergradmath.pdf
\section{Chapter 2.1: Calculus of Motion}

Consider vectors of motion against $t$ of the form $ \overrightarrow{x}(t) = \langle x(t), y(t), \ldots \rangle$.
\begin{itemize}
\item A \textbf{line} through $p = (a, b, c)$ parallel to $\overrightarrow{v} = \langle v_x, v_y, v_z\rangle$ is $\overrightarrow{x}(t) = \overrightarrow{p} + t \overrightarrow{v}$ 
\item \textbf{velocity} is characterized completely by $\overrightarrow{v}(t) = \overrightarrow{x}'(t) = \langle x'(t), y'(t), z'(t)\rangle$.
\item The \textbf{speed} of an object along that line versus $t$ is the length of $v$ ($\|v\|$)  
\item Therefore, the speed of an object along line 
$$ \langle x(t), y(t), z(t)\rangle = \langle 0, 2, -3\rangle + t\langle 1,-2,2\rangle$$ is $$\sqrt{1^2+(-2)^2 + 2^2} = 3$$
\item Note that $\overrightarrow{v}$ need not be constant.  The speed of $$\overrightarrow{x}(t) =  \overrightarrow{p} + 3 \sin(2\pi t)\hat{u}, \| \hat{u} \| = 1$$ would then be $$\| 6\pi \cos(2 \pi t) \hat{u} \| = |6\pi \cos(2 \pi t)|$$
\item \textbf{Acceleration} $a(t) = v'(t) = x''(t)$ is straightforward.  Acceleration of $$x(t) = \langle -1 + \cos(t), 1, \cos(t)\rangle = \langle -\cos(t), 0, -\cos(t)\rangle$$
\item An example position vector for a planet of distance $r$ from the sun could be $\langle r \cos(t), r \sin(t) \rangle$.  The acceleration vector points in the opposite direction: $\langle - r \cos(t), - r \sin(t) \rangle$.  In addition to being the analytical second derivative, consider that the \emph{force} of gravity, (which, by $F = ma$ is proportional to acceleration) points towards the sun.  
\item A \textbf{helix} could be a 3D extension like $\langle r \cos(t), r \sin(t), b\cdot t \rangle$.  
\end{itemize}

\section{Chapter 2.2: Space Curves}
\begin{itemize}
\item TODO: Problem 5 - rotating ellipses and solving intersections with planes
\item Note that while $\overrightarrow{x}(t) = \langle \cos(t), \sin(t), 5 \rangle$ and  $\overrightarrow{x}(t) = \langle \cos(2t), \sin(2t), 5 \rangle$ describe the same curve, the space curve also records motion in time, so the \emph{velocity} may be different.
\item If $\overrightarrow{x}(t) = t\overrightarrow{v}$, then speed is $\frac{\| \overrightarrow{x}(t+\Delta t) - \overrightarrow{t} \|}{\Delta t} = \| \overrightarrow{v} \|$,  direction is $\frac{\overrightarrow{v}}{\| \overrightarrow{v} \|}$, and velocity $\overrightarrow{v}$ is the product of speed and direction.
\item So $\overrightarrow{v}(t) = \lim_{\Delta t \rightarrow 0} \frac{\overrightarrow{x}(t + \Delta t) - \overrightarrow{x}(t)}{\Delta t} = \overrightarrow{x}'(t) = \frac{d\overrightarrow{x}}{dt} = \langle x'(t), y'(t), z'(t) \rangle$
\item Neat conceptual result: any $y = f(x)$ can be made into $x(t) = \langle t, f(t), 0 \rangle$, and then $v(t) = \langle 1, f'(t), 0 \rangle$, which points along the tangent line at $\langle t, f(t), 0 \rangle$.
\item Note that dot product derivatives work like regular product: $[\overrightarrow{a}(t) \cdot \overrightarrow{b}(t)]' = \overrightarrow{a}'(t) \cdot \overrightarrow{b}(t) + \overrightarrow{a}(t) \cdot \overrightarrow{b}'(t)$, 
but the cross product does not work the same since $\frac{d}{dt}[a \times b] = a' \times b + a \times b'$, but since $a \times b' = -b' \times a$, can't switch the order to $a' \times b + b' \times a$ due to this non-commutativity.
\item If $$\overrightarrow{x}(t) = \overrightarrow{p}+ t \overrightarrow{v},$$ calculating velocity with respect to origin  becomes
$$ \frac{d}{dt} \| \overrightarrow{x}(t) \| = \frac{\overrightarrow{x}(t) \cdot \overrightarrow{x}'(t)}{\| \overrightarrow{x}(t) \|} = \frac{\overrightarrow{x}}{\| \overrightarrow{x} \|} \cdot \overrightarrow{v},$$ after rewriting the distance formula and chugging through the chain rule.
\item However, it becomes more clear when considering that $(\overrightarrow{v} \cdot \hat{x}) \hat{x}$ is the projection of the velocity vector onto the position vector.  So, the length of this is the rate of change of distance from origin!
\end{itemize}

\section{Chapter 2.3: Integrals and Arc Length}{
\begin{itemize}
\item Integral of a vector function can be defined componentwise in a straightforward way: $\int_a^b {\overrightarrow{x}(t)} = \langle \int_a^b {x(t)},  \int_a^b {y(t)},  \int_a^b {z(t)} \rangle $ 
\item Example: if ball launched from origin with velocity $\langle 1, 2, 3 \rangle$ and acceleration $\langle 0, 0, -1 \rangle$, it lands at
\begin{align} 
\frac{dv}{dt}dt = \langle 0, 0, -1 \rangle \\ 
\int{\frac{dv}{dt}dt} = v =  \langle C , D , -t +F \rangle  = \langle 1, 2, 3 \rangle =  \langle 1, 2, -t + 3\rangle, t = 0 \\
x = \int{v}  = \langle t+K, 2t+M, -\frac{1}{2}t^2 + 3t + N \rangle, x(\overrightarrow{0}) = \langle 0 ,0, 0\rangle \\ 
\overrightarrow{x}(t) = \langle t, 2t, 3t-\frac{1}{2}t^2 \rangle \\
z(t) = 0 \rightarrow t = 6 \rightarrow \overrightarrow{x}(6) = \langle 6, 12, 0 \rangle \\
\end{align} 
\item Also, generalizing $ds = \sqrt{(dx)^2  + (dy)^2}$, the length of an arc from point $a$ to $b$ aporoaches $\int_a^b \| x'(t) \| dt$
\item Example: a helix $\langle a \cos (\omega t), a \sin (\omega t), b \omega t \rangle$, parametrized by time $t$ can be rewritten in terms of $s$, the arc length: 
\begin{align}
s =  \int \|x'(t)\| dt \\
s = \int \sqrt{(-\omega a \sin (\omega t))^2 + (\omega a \cos (\omega t))^2 + (b \omega)^2}dt \\
s = |\omega| \int \sqrt{(a^2 + b^2)}dt \\ 
s = |\omega| \sqrt{a^2 + b^2}t
\end{align}
\item \emph{Note: It's weird to think of time in terms of length}.  Could be analytically useful?
\end{itemize}

\end{document}