\documentclass[11pt, oneside]{article}   	% use "amsart" instead of "article" for AMSLaTeX format
\usepackage{geometry}                		% See geometry.pdf to learn the layout options. There are lots.
\geometry{letterpaper}                   		% ... or a4paper or a5paper or ... 
\usepackage[parfill]{parskip}    		% Activate to begin paragraphs with an empty line rather than an indent
\usepackage{graphicx}				% Use pdf, png, jpg, or eps§ with pdflatex; use eps in DVI mode
								% TeX will automatically convert eps --\rangle pdf in pdflatex		
\usepackage{amssymb}
\usepackage{amsmath}

\usepackage{verbatim}
\usepackage{tikz} 

\usepackage{syntonly}
% \syntaxonly \langle -- use this for checking syntax only
% \mbox {text} - keep together
% \fbox {text} - keep together and draw around

%\pagestyle{plain|headings|empty} % header and footer p.27
%SetFonts
%\include{filename}, \includeonly{filename1, filename2} , \input[fiename}

%SetFonts% 

\title{Brilliant: Vector Calculus}
\author{Dave Fetterman}
\date{6/21/22}							% Activate to display a given date or no date

\begin{document}
\maketitle
Note: Latex reference: http://tug.ctan.org/info/undergradmath/undergradmath.pdf
\section{Chapter 1.2: Calculus of Motion}

Consider vectors of motion against $t$ of the form $ \overrightarrow{x}(t) = \langle x(t), y(t), \ldots \rangle$.
\begin{itemize}
\item A \textbf{line} through $p = (a, b, c)$ parallel to $\overrightarrow{v} = \langle v_x, v_y, v_z\rangle$ is $\overrightarrow{x}(t) = \overrightarrow{p} + t \overrightarrow{v}$ 
\item \textbf{velocity} is characterized completely by $\overrightarrow{v}(t) = \overrightarrow{x}'(t) = \langle x'(t), y'(t), z'(t)\rangle$.
\item The \textbf{speed} of an object along that line versus $t$ is the length of $v$ ($\|v\|$)  
\item Therefore, the speed of an object along line 
$$ \langle x(t), y(t), z(t)\rangle = \langle 0, 2, -3\rangle + t\langle 1,-2,2\rangle$$ is $$\sqrt{1^2+(-2)^2 + 2^2} = 3$$
\item Note that $\overrightarrow{v}$ need not be constant.  The speed of $$\overrightarrow{x}(t) =  \overrightarrow{p} + 3 \sin(2\pi t)\hat{u}, \| \hat{u} \| = 1$$ would then be $$\| 6\pi \cos(2 \pi t) \hat{u} \| = |6\pi \cos(2 \pi t)|$$
\item \textbf{Acceleration} $a(t) = v'(t) = x''(t)$ is straightforward.  Acceleration of $$x(t) = \langle -1 + \cos(t), 1, \cos(t)\rangle = \langle -\cos(t), 0, -\cos(t)\rangle$$
\item An example position vector for a planet of distance $r$ from the sun could be $\langle r \cos(t), r \sin(t) \rangle$.  The acceleration vector points in the opposite direction: $\langle - r \cos(t), - r \sin(t) \rangle$.  In addition to being the analytical second derivative, consider that the \emph{force} of gravity, (which, by $F = ma$ is proportional to acceleration) points towards the sun.  
\item A \textbf{helix} could be a 3D extension like $\langle r \cos(t), r \sin(t), b\cdot t \rangle$.  
\end{itemize}


\end{document}