\documentclass[11pt, oneside]{article}   	% use ``amsart'' instead of ``article'' for AMSLaTeX format
\usepackage{geometry}                		% See geometry.pdf to learn the layout options. There are lots.
\geometry{letterpaper}                   		% ... or a4paper or a5paper or ... 
\usepackage[parfill]{parskip}    		% Activate to begin paragraphs with an empty line rather than an indent
\usepackage{graphicx}				% Use pdf, png, jpg, or eps§ with pdflatex; use eps in DVI mode
								% TeX will automatically convert eps --\rangle pdf in pdflatex		
\usepackage{amssymb}
\usepackage{amsmath}

\usepackage{verbatim}
\usepackage{tikz} 

\usepackage{syntonly}
% \syntaxonly \langle -- use this for checking syntax only
% \mbox {text} - keep together
% \fbox {text} - keep together and draw around

%\pagestyle{plain|headings|empty} % header and footer p.27
%SetFonts
%\include{filename}, \includeonly{filename1, filename2} , \input[fiename}

%SetFonts% 

\title{Brilliant: Vector Calculus}
\author{Dave Fetterman}
\date{6/21/22}							% Activate to display a given date or no date

\begin{document}
\maketitle
Note: Latex reference: http://tug.ctan.org/info/undergradmath/undergradmath.pdf

\section{Chapter 2: Basics, Vector fields}
\subsection{Chapter 2.1: Calculus of Motion}

Consider vectors of motion against $t$ of the form $ \vec{x}(t) = \langle x(t), y(t), \ldots \rangle$.
\begin{itemize}
\item A \textbf{line} through $p = (a, b, c)$ parallel to $\vec{v} = \langle v_x, v_y, v_z\rangle$ is $\vec{x}(t) = \vec{p} + t \vec{v}$ 
\item \textbf{velocity} is characterized completely by $\vec{v}(t) = \vec{x}'(t) = \langle x'(t), y'(t), z'(t)\rangle$.
\item The \textbf{speed} of an object along that line versus $t$ is the length of $v$ ($\|v\|$)  
\item Therefore, the speed of an object along line 
$$ \langle x(t), y(t), z(t)\rangle = \langle 0, 2, -3\rangle + t\langle 1,-2,2\rangle$$ is $$\sqrt{1^2+(-2)^2 + 2^2} = 3$$
\item Note that $\vec{v}$ need not be constant.  The speed of $$\vec{x}(t) =  \vec{p} + 3 \sin(2\pi t)\hat{u}, \| \hat{u} \| = 1$$ would then be $$\| 6\pi \cos(2 \pi t) \hat{u} \| = |6\pi \cos(2 \pi t)|$$
\item \textbf{Acceleration} $a(t) = v'(t) = x''(t)$ is straightforward.  Acceleration of $$x(t) = \langle -1 + \cos(t), 1, \cos(t)\rangle = \langle -\cos(t), 0, -\cos(t)\rangle$$
\item An example position vector for a planet of distance $r$ from the sun could be $\langle r \cos(t), r \sin(t) \rangle$.  The acceleration vector points in the opposite direction: $\langle - r \cos(t), - r \sin(t) \rangle$.  In addition to being the analytical second derivative, consider that the \emph{force} of gravity, (which, by $F = ma$ is proportional to acceleration) points towards the sun, \emph{with acceleration perpendicular to velocity}.

\item A \textbf{helix} could be a 3D extension like $\langle r \cos(t), r \sin(t), b\cdot t \rangle$.  
\end{itemize}

\section{Chapter 2.2: Space Curves}
\begin{itemize}
\item Note that while $\vec{x}(t) = \langle \cos(t), \sin(t), 5 \rangle$ and  $\vec{x}(t) = \langle \cos(2t), \sin(2t), 5 \rangle$ describe the same curve, the space curve also records motion in time, so the \emph{velocity} may be different.
\item If $\vec{x}(t) = t\vec{v}$, then speed is $\frac{\| \vec{x}(t+\Delta t) - \vec{t} \|}{\Delta t} = \| \vec{v} \|$,  direction is $\frac{\vec{v}}{\| \vec{v} \|}$, and velocity $\vec{v}$ is the product of speed and direction.
\item So $\vec{v}(t) = \lim_{\Delta t \rightarrow 0} \frac{\vec{x}(t + \Delta t) - \vec{x}(t)}{\Delta t} = \vec{x}'(t) = \frac{d\vec{x}}{dt} = \langle x'(t), y'(t), z'(t) \rangle$
\item Neat conceptual result: any $y = f(x)$ can be made into $x(t) = \langle t, f(t) \rangle$, and then $v(t) = \langle 1, f'(t) \rangle$, which points along the tangent line at $\langle t, f(t)\rangle$.
\item Note that dot product derivatives work like regular product: $[\vec{a}(t) \cdot \vec{b}(t)]' = \vec{a}'(t) \cdot \vec{b}(t) + \vec{a}(t) \cdot \vec{b}'(t)$, 
but the cross product does not work the same since $\frac{d}{dt}[a \times b] = a' \times b + a \times b'$, but since $a \times b' = -b' \times a$, can't switch the order to $a' \times b + b' \times a$ due to this non-commutativity.
\item If $$\vec{x}(t) = \vec{p}+ t \vec{v},$$ calculating velocity with respect to origin  becomes
$$ \frac{d}{dt} \| \vec{x}(t) \| = \frac{\vec{x}(t) \cdot \vec{x}'(t)}{\| \vec{x}(t) \|} = \frac{\vec{x}}{\| \vec{x} \|} \cdot \vec{v},$$ after rewriting the distance formula and chugging through the chain rule.
\item However, it becomes more clear when considering that $(\vec{v} \cdot \hat{x}) \hat{x}$ is the projection of the velocity vector onto the position vector.  So, the length of this is the rate of change of distance from origin!
\end{itemize}

\section{Chapter 2.3: Integrals and Arc Length}
\begin{itemize}
\item Integral of a vector function can be defined componentwise in a straightforward way: $\int_a^b {\vec{x}(t)} = \langle \int_a^b {x(t)},  \int_a^b {y(t)},  \int_a^b {z(t)} \rangle $ 
\item Example: if ball launched from origin with velocity $\langle 1, 2, 3 \rangle$ and acceleration $\langle 0, 0, -1 \rangle$, it lands at
\begin{align} 
\frac{dv}{dt}dt = \langle 0, 0, -1 \rangle \\ 
\int{\frac{dv}{dt}dt} = v =  \langle C , D , -t +F \rangle  = \langle 1, 2, 3 \rangle =  \langle 1, 2, -t + 3\rangle, t = 0 \\
x = \int{v}  = \langle t+K, 2t+M, -\frac{1}{2}t^2 + 3t + N \rangle, x(\vec{0}) = \langle 0 ,0, 0\rangle \\ 
\vec{x}(t) = \langle t, 2t, 3t-\frac{1}{2}t^2 \rangle \\
z(t) = 0 \rightarrow t = 6 \rightarrow \vec{x}(6) = \langle 6, 12, 0 \rangle \\
\end{align} 
\item Also, generalizing $ds = \sqrt{(dx)^2  + (dy)^2}$, the length of an arc from point $a$ to $b$ approaches \fbox{$\int_a^b ds = \int_{t_a}^{t_b} \| x'(t) \| dt$}
\item Example: a helix $\langle a \cos (\omega t), a \sin (\omega t), b \omega t \rangle$, parametrized by time $t$ can be rewritten in terms of $s$, the arc length: 
\begin{align}
s =  \int \|x'(t)\| dt \\
s = \int \sqrt{(-\omega a \sin (\omega t))^2 + (\omega a \cos (\omega t))^2 + (b \omega)^2}dt \\
s = |\omega| \int \sqrt{(a^2 + b^2)}dt \\ 
s = |\omega| t \sqrt{a^2 + b^2}
\end{align}
\item \emph{Note: It's weird to think of time in terms of length}.  Could be analytically useful?
\end{itemize}


\section{Chapter 2.4: Frenet Formulae}

Main idea: Establish three new vectors $\hat{T}(s), \hat{N}(s), \hat{B}(s)$ that change as we move along a space curve, instead of $\vec{x}(t)$ that changes over an external ``time'' idea.

Remember that  $s = \int_0^t \| \vec{x}'(\tilde{t}) \| d\tilde{t}$, so $\frac{ds}{dt} = \| \vec{x}'(t)\|$ .

\subsection{$\hat{T}$: Vector tangent to space curve}
\begin{itemize}
\item Remember arc length is $s = \int_0^t \| \vec{x}'(\tilde{t})d\tilde{t} \|$
\item $\hat{T}$ is just normalized grad: $\frac{\vec{x}'(t)}{\|\vec{x}'(t)\|}$
\item This implies \fbox{$\frac{d\vec{x}}{ds} = \hat{T}$} since 
\begin{align}
s = \int_0^t \| \vec{x}'(\tilde{t})d\tilde{t} \| \\
\frac{ds}{dt} = \| \vec{x}(t) \| \\
\hat{T} = \frac{\vec{x}'(t)}{\|\vec{x}'(t)\|}  = \frac{d\vec{x}}{dt} \cdot \frac{dt}{ds} \\
\hat{T} = \frac{d\vec{x}}{ds} \\
\end{align}
\item So this is how the space curve $\vec{x}$ changes as it moves along the curve at length $s$.
\item It's normalized, so it's the same whether parameterized by t, s, or whatever.
\end{itemize}


\subsection{$\hat{N}$: Normal to curve (perpendicular to $\hat{T}$)}
Normal vectors are:
\begin{itemize}
\item $\vec{x}''(t)$ normalized as \fbox{$\frac { \frac{d\hat{T}}{ds} }   {\|  \frac{d\hat{T}}{ds}  \|}  = \hat{N}$}
\item The normal vector to the curve
\item $\bot$ to $\hat{T}$ in direction of acceleration.  So a multiple of acceleration vector.
\item $\frac{\hat{T}'(t)} { \| \hat{T}'(t) \|}$.  The following sequence shows any unit length vector is perpendicular to its derivative.
\begin{align}
\| \hat{T} \| = 1 \\
d(\| \hat{T} \|^2) = 0 \\
d(\| \hat{T} \|^2) = d(\hat{T} \cdot \hat{T}) = \hat{T}(t) \cdot 2\hat{T}'(t) \\
\hat{T}(t) \cdot \hat{T}'(t)  = 0
\end{align}
\item $\frac{ \frac{d\hat{T}}{ds}  }  { \| \frac{d\hat{T}}{ds}  \| }$ since it's the same as the above, but parametrized over $s$ instead of $t$.  Doesn't change the direction of the vector!
\end{itemize}

Example:if  $\vec{x}(t) = \langle R\cos(\omega t), R\sin(\omega t), 0 \rangle$, then:
\begin{itemize}
\item $\vec{a} = \frac{d^2\vec{x}}{dt^2}$ just by definition
\item $\vec{a} = -\omega^2 \vec{x}$ just by calculation
\item $\hat{T}(t) = \langle -\sin(\omega t), \cos (\omega t), 0 \rangle$
\item $\| \hat{T}(t) \| = 1$
\item $\hat{N} = \frac{\hat{T}'(t)} { \| \hat{T}'(t) \|} =  \langle -\cos(\omega t), -\sin (\omega t), 0 \rangle$
\item So $\vec{a} = R\omega^2\hat{N}$ by these formulae.
\end{itemize}

This leads us to believe acceleration and $\hat{N}$, the normed derivative of $\hat{T}$ are related.

The part of acceleration $\vec{a}$ parallel to $\hat{T}$ is the projection $(\vec{a} \cdot \hat{T}) \hat{T}$

The perpendicular part is then $\vec{a}$ minus that: $\vec{a} - (\vec{a} \cdot \hat{T}) \hat{T}$

This also equals $(\frac{ds}{dt})^2 \|\frac{d\hat{T}}{ds} \| \hat{N}$ because
\begin{align}
\vec{x}' = \frac{dx}{dt} = T = \hat{T} \cdot \|  \frac{dx}{dt} \| \\
s = \int_0^t \| \vec{x}'(t) \| \rightarrow \frac{ds}{dt} = \| \vec{x}'(t) \|  
\end{align}

$\hat{N} = \frac{d\hat{T}}{ds}$ normalized, so
\begin{align}
\vec{a} = \frac{d^2\vec{x}}{dt^2} = \frac{d}{dt}(\|\frac{dx}{dt}\|  \frac{\frac{dx}{dt}} {\|\frac{dx}{dt}\|} ) = \frac{d}{dt}(\| \vec{x}'(t)\|\hat{T}(t)) = \frac{d \|\vec{x}'(t) \|}{dt}\hat{T} + \| \vec{x}'(t) \| \frac{d\hat{T}}{dt} \\
= \frac{d \|\vec{x}'(t) \|}{dt}\hat{T} + \frac{ds}{dt} \frac{d\hat{T}}{ds} \frac{ds}{dt} \\ 
= \frac{d \|\vec{x}'(t) \|}{dt}\hat{T} + (\frac{d{s}}{dt} )^2 \| \frac{d\hat{T}}{ds} \| \hat{N}
\end{align}

This is ``a = $\hat{T}$'s parallel part plus $\hat{T}$'s perpendicular (N) part'', so the second term is $a_{\bot}$

\subsection{$\hat{T}$ and $\hat{N}$}

\begin{itemize}
\item Form a plane, since first, any normal vector's derivative is perpendicular to the vector
\item $\kappa$ is curvature: how much we're curving in that $T \times N$ plane.
\item \fbox{$\kappa =   \|  \frac{d\hat{T}}{ds}  \| $ }
\item Therefore, by the definition of $\hat{N} = \frac{d\hat{T}/ds}{\|d\hat{T}/ds\|}$, \fbox{$\frac{d\hat{T}}{ds} = \kappa \hat{N}$ }\textbf{(Frenet equation 1)}
\end{itemize}

Note that curvature $\kappa(s) = \| \frac{d\hat{T}}{ds}\|$ is geometric (depends on s, not time) and changes as $\hat{T}$ changes.

Example: Curvature of $\vec{x}(t) = \langle \cos(t), \sin(t), bt \rangle$ 
\begin{align}
x'(t) = \langle -\sin(t), \cos(t), b \rangle \\ 
\|x'(t) \| = \sqrt{1 + b^2} \\ 
s = \int_0^t\|x'(\tilde{t})\| d\tilde{t} = \int_0^t \sqrt{(1+b^2)} = t\sqrt{(1+b^2)} \rightarrow t = \frac{s}{\sqrt{1+b^2}}
\end{align}

Do the substitution of $\frac{s}{\sqrt{1+b^2}}$ for $t$ above to get $x'(s)$, and from there, you can figure out $\frac{d\hat{T}}{ds}$ and normalize to get $\kappa = \frac{1}{1+b^2}$

\subsection{$\hat{B}$ is binormal: perpendicular to both}
\begin{itemize}
\item defined as \fbox{$\hat{B} = \hat{T} \times \hat{N}$}
\item Therefore, by derivative 
\begin{align}
\frac{d\hat{B}}{ds} = \frac{d\hat{T}}{ds} \times \hat{N} +  \hat{T} \times \frac{d\hat{N}}{ds}\\ 
\frac{d\hat{B}}{ds} = \kappa \hat{N} \times \hat{N}  +  \hat{T} \times \frac{d\hat{N}}{ds}\\ 
\frac{d\hat{B}}{ds} =  \hat{T} \times \frac{d\hat{N}}{ds}
\end{align}
This means T is orthogonal to dB, and we already know B and dB are orthogonal.
We're working in 3D with the cross product, so dB is parallel to N.
\item Therefore, we define \textbf{torsion} $\tau$ so that \fbox{$- \frac{d\hat{B}}{ds}  = \tau \hat{N}$  }\textbf{(Frenet equation 2)}.  Negative sign by convention.
\item Can also cross by $N$ on both sides to get $- \frac{d\hat{B}}{ds}  \times \hat{N} = \tau $ 
\item $\tau$ measures how the plane defined by $\hat{T}, \hat{N}$ twists around.  On a circle, $\hat{B}$ wouldn't change, so the derivative would be zero.
\item \textbf{Final Frenet equation}.  Prereq: $\hat{B} = \hat{T} \times \hat{N} \rightarrow \hat{N} = \hat{B} \times \hat{T} \rightarrow \hat{T} = \hat{N} \times \hat{B}$
\begin{align}
\frac{d\hat{N}}{ds} = \frac{d\hat{B}}{ds} \times \hat{T} +   \hat{B} \times  \frac{d\hat{T}}{ds} \\
\frac{d\hat{N}}{ds} = -\tau \hat{N} \times \hat{T} + \hat{B} \times \kappa \hat{N} \\ 
\frac{d\hat{N}}{ds}  = \tau \hat{B} - \kappa \hat{T}
\end{align}

\end{itemize}

\section{Chapter 2.5: Parametrized Surfaces}

Main approaches to describing a surface:
\begin{itemize}
\item Can parameterize by $\vec{x}(u,v) = x(u,v), y(u,v), z(u,v)$
\item Can perhaps parameterize $f(x,y,z) = c$ by $z = g(x,y)$
\item Can also use ideas like $\nabla f = 0$ to find a normal.
\end{itemize}

There are many out-of-the-box paremetrizations including:

\begin{itemize}
\item Sphere at (0,0,0): $\vec{x}(u,v) = \langle R\cos(u)\sin(v), Rsin(u)\sin(v), R\cos(v)\rangle$, where $u \in [0, 2\pi), v \in [0, \pi]$
\item Rotate function $y = f(x)$ around the x-axis: $\vec{x}(u,v) = \langle u, f(u)\cos(v), f(u)\sin(v) \rangle$, where $u \in D, v \in [0, 2\pi]$
\end{itemize}

Tangent vectors to $\vec{x}(u,v)$ are $\frac{\delta \vec{x}} {du}, \frac{\delta \vec{x}} {dv}$, so unit normal \fbox{	$\hat{n} = \pm \frac 
					{\frac{d \vec{x}} {du} \times \frac{\delta \vec{x}}{dv}}
					{\| \frac{d \vec{x}} {du} \times \frac{\delta \vec{x}}{dv} \|}
					$}
					
\textbf{Example}: Torus $\vec{x} = \langle [2 + \cos(v)]\cos(u),  [2 + \cos(v)]\sin(u), \sin(v)\rangle, u,v \in [0,2 \pi)$.  What's the tangent plane at $u = \frac{\pi}{4}, v=0$?
\begin{align}
d\vec{x}/du = \langle -\sin(u)(2 + \cos(v)),  \cos(u)(2 + \cos(v)), 0 \rangle \\
d\vec{x}/dv = \langle -\sin(v)\cos(u), -\sin(v)\sin(u), \cos(v) \rangle \\
u = \frac{\pi}{4}, v=0 :  \\
d\vec{x}/du = \langle -\frac{3}{\sqrt{2}}, \frac{3}{\sqrt{2}}, 0, \rangle \\
d\vec{x}/dv = \langle 0, 0, 1 \rangle \\
dx/du \times dx/dv = \langle  \frac{3}{\sqrt{2}}, \frac{3}{\sqrt{2}}, 0\rangle \\
\hat{n} =  \langle \frac{1}{\sqrt{2}}, \frac{1}{\sqrt{2}}, 0 \rangle \\
\hat{n}  \cdot \vec{x} = 0 \rightarrow \hat{n} \cdot (x - x_0, y-y_0, z-z_0) = 0 \\
\rightarrow ... \rightarrow x + y = 3\sqrt{2}  \\
\end{align}



\subsection{Example: Ellipsoid $x^2 + 2y^2 + z^2 = 4$ What's the normal at $(1, \frac{1}{\sqrt{2}}, \sqrt{2})$?}

\textbf{Method 1: parametrize with spherical u, v}
First, transform to sphere with change of coordinates, then flip to speherical coordinates.
\begin{align}
x^2 + 2y^2 + z^2 = 4 \\
X = x/2, Y=\frac{Y}{\sqrt{2}}, Z = z/2 \\
X^2 + Y^2 + Z^2 = 1 \\
X = \cos(u)\sin(v), Y = \sin(u)\sin(v), Z = \cos(v) \\
p = (1, \frac{1}{\sqrt{2}}, \sqrt{2})  \rightarrow u = v = \frac{\pi}{4} \\
\frac{dx}{du}(\frac{\pi}{4}, \frac{\pi}{4}) = \langle -1, \frac{1}{\sqrt{2}}, 0 \rangle \\
\frac{dx}{dv}(\frac{\pi}{4}, \frac{\pi}{4}) = \langle 1, \frac{1}{\sqrt{2}}, -\sqrt{2} \rangle \\
\frac{dx}{du}(\frac{\pi}{4}, \frac{\pi}{4}) \times \frac{dx}{dv}(\frac{\pi}{4}, \frac{\pi}{4}) = \langle 1, \sqrt{2}, \sqrt{2}\rangle \\
\hat{n}_{out} = \frac{\langle -1, -\sqrt{2}, -\sqrt{2}\rangle }{\sqrt{5}}
\end{align}


\textbf{Method 2: rewrite as z = g(x,y)}

\begin{align}
x^2 + 2y^2 + z^2 = 4 \\
z = (4 - x^2 - 2y^2)^\frac{1}{2} \\
dz / dx = \frac{1}{2} \times -2x (4 - x^2 - 2y^2)^{-\frac{1}{2}} = -\frac{1}{\sqrt{2}} \\
dz / dy = \frac{1}{2} \times -4y (4 - x^2 - 2y^2)^{-\frac{1}{2}} =-2\sqrt{2}/\sqrt{2} = -1\\
f \approx \sqrt{2}  + dz / dx(1, \frac{1}{\sqrt{2}})(x - 1) + dz / dy (1,  \frac{1}{\sqrt{2}}) (y - \frac{1}{\sqrt{2}})\\
\rightarrow ... \rightarrow \frac{1}{\sqrt{2}}x + y + z = 2\sqrt{2} \\
\end{align}

giving us normal vector $\langle \frac{1}{\sqrt{2}}, 1, 1 \rangle = \frac{\langle 1, \sqrt{2}, \sqrt{2}\rangle }{\sqrt{5}}$
after normalization.

\textbf{Method 3: gradient}

Gradient is always normal to the tangent plane.
Recognize level set of $f(x,y,z) = x^2 + 2y^2 + z^2$.  

$\nabla f = \langle 2x, 4y, 2z \rangle \rightarrow \nabla f(1, \frac{1}{\sqrt{2}}, \sqrt{2}) = \langle 2, 2\sqrt{2}, 2\sqrt{2}\rangle$

Then normalize.

\subsection{Mobius strip and ``outward direction''}
Mobius strip is 
\begin{itemize}
\item $x = 2\cos(u) + v\cos(\frac{u}{2})$
\item $y = 2\sin(u) + v\cos(\frac{u}{2})$
\item $z = v \sin(\frac{u}{2})$
\item $u \in [0, 2\pi], v \in [-\frac{1}{2}, \frac{1}{2}]$
\end{itemize}

$\hat{n} = \frac
	{\vec{x}_u \times  \vec{x}_v}
	{\|\vec{x}_u \times  \vec{x}_v\|}
	$ at (0,0) is $\langle 0, 0, -1 \rangle$, 
	
but at the same point $(2\pi, 0)$ $\hat{n} = \langle 0, 0, 1 \rangle$!!
	


\section{Chapter 2.6: Vector Fields}

(Lots of intuition questions here...)

One nugget: using \textbf{gradient vector fields}:
Suppose $\vec{F}(x,y) = \langle 2, -4y^3 \rangle$.  
If $\vec{F} = \nabla f$ for some (single value function) $f$, then $F$'s arrows are perpendicular to a level set $f = c$.
So look at $f  =  2x -y^4 $ and find perpendicular arrows to these.  That's actually F!

\textbf{Linear approximation for $\vec{F}: D \in \mathbf{R}^n \rightarrow \mathbf{R}^m$}

Main idea: 
$\vec{F}(\vec{x}) = \vec{F}(\vec{a}) + A(\vec{a})(\vec{x} - \vec{a})$

Note that $A$ takes in vectors of size $n$ (so it has as many columns as the input space), and has $m$ functions (rows) that operate on it.
So the Jacobian, $A$, has as row $i$, column $j$, the quantity $\frac{dF_i}{dx_j}(\vec{a})$.

$dF_i/d\vec{x}$ extends across row $i$.

\section{Chapter 2.7: Jack and the Beanstalk (Newton's method)}

\textbf{Basis for Newton's:}

If we're estimating $x_1$ by following the derivative at $x_0$, this means we're looking at the line with x-intercept $x_1$, with slope $f'(x_0)$.

  So instead of $y = mx+b$, we'll flip the two and use 
  
 $x = y/m + x_{int}$ 
 
or $x_0 = f(x_0)\frac{1}{f'(x_0)} + x_1$, 

or \fbox{$x_1 = x_0 - \frac{f(x_0)}{f'(x_0)}$}

Note that, under Newton's something like $|x|$ will converge immediately, $x^3$ will converge moderately, and a S-curve might barely converge if at all.

The extension of this with the Jacobian matrix $A = DF'(x_0)$ is $\vec{x}_1 = \vec{x}_0 - (D\vec{F}(\vec{x}_0))^{-1}\vec{F}(x_0)$ 


\section{Chapter 2.8: Electrostatic bootcamp}

Electric charge radiates out equally in all directions, and is inversely proportional to distance.  

Formula, with $Q$ as the charge, $\epsilon_0$ is a constant: $\vec{E}(x,y,z) = \frac{Q}{4\pi\epsilon_0\|x\|^2}\hat{x}$

A field line is a special case of a \textbf{flow line} - the space curve that follows $\vec{F}$'s arrows.  The tangent vector to the flow line is $\vec{F}(\vec{x}(\tilde{t}))$ ($\tilde{t}$ is not time here), so $\frac{d\vec{x}}{d\tilde{t}} = \vec{F}(\vec{x}(\tilde{t}))$

Example: Vector field $\vec{F}(x,y) = \langle -2y, 3x \rangle$.  What's the flow line through (2,0)?

Solution: Need to solve 
$dx/dt = -2y, dy/dt = 3x$.  Key is ``separating the equations''.  Remember x and y are functions of t!

\begin{align}
\frac{d^2x}{dt^2} = -2 \frac{dy}{dt}= -2 \times 3x = -6x.\\
\frac{d^2y}{dt^2} = -2 \frac{dy}{dt}= -2 \times 3x = -6y.\\
x(t) = -6x''(t), y(t) = -6y''(t) \\
\rightarrow x = A\cos(\sqrt{6}t)+ B\sin(\sqrt{6}t),  y = C\cos(\sqrt{6}t)+ D\sin(\sqrt{6}t) \\
\frac{dx}{dt} = -2y(t) \rightarrow   \frac{\sqrt{6}}{2}A\sin(\sqrt{6}t)- \frac{\sqrt{6}}{2}B\cos(\sqrt{6}t)  = y(t) \\
x(t=0) = 2 \rightarrow A = 2 \\
y(t=0) = 0 \rightarrow B = 0 \\
\vec{F}(t) = \langle 2\cos(\sqrt{6}t), \sqrt{6}\sin(\sqrt{6}t) \rangle \\
\end{align}

Note: \textbf{Field lines} follow rules:
\begin{itemize}
\item Go from positive charges to negative
\item Density of lines directly relates to how much charge a point has
\item Lines don't intersect.
\item Corollary: If count of out equals count of in, point has zero charge
\item ``Number'' (to be defined) of field lines in and out of a \emph{surface} related to the charge inside.  Upcoming.
\end{itemize}

\section{Chapter 3: Surface integrals, Flux, Divergence}

\subsection{3.1: Surface Integrals}

Example: Fluid pressure in a tank is:
\begin{itemize}
\item Proportional (via some weight constant $p_{fluid}$) to depth of the point
\item Pushes out via the normal $\hat{n}$ 
\item So, for the $x=l$ side of a cube of length l, this would be 

$\vec{F}_{x=l} = (\iint_{[0,l] \times [0,l]} p_{fluid} [1 - \frac{z}{l}] dy dz) \hat{i}$
\end{itemize}



\textbf{Example}: Hemisphere of size l, sitting at (0, 0, 0)

Finding the out pointing unit normal of hemisphere at point $(x, y, \sqrt{l^2-x^2-y^2})$

Note: Can just eyeball this, but one way is the \textbf{gradient}.  

First, the relation is $x^2 + y^2 + (z-l)^2 = l^2$.
Make it a function $g$ and take the level set at $l^2$:

\begin{align}
g(x,y,z) = x^2  + y^2 + (z - l)^2 = l^2 \\
\nabla g(x,y,z) = \langle 2x , 2y , 2(z-l)  \rangle \\
\hat{n} = \pm \frac{\nabla g(x,y,z)}{ \|\nabla g(x,y,z) \|} \\ 
\hat{n} = \pm \frac{\langle x , y , (z-l)  \rangle}{\sqrt{x^2 + y^2 + (z-l)^2}} \\
\hat{n} = \pm \frac{\langle x , y , (z-l) \rangle}{\sqrt{l^2}} \\
\hat{n} = \pm \langle \frac{x}{l}, \frac{y}{l} ,\frac{z}{l} - 1\rangle \\
\end{align}


Note: Integrating over a patch dA on the surface means finding the area of micro-patches 
$\Delta A_{ij}$, which is the parallelogram defined by 
\begin{align}
s_1 = \langle \Delta x_i, 0, \Delta x_i f_x(x_i^*,y_j^*) \rangle \\
s_2 = \langle 0, \Delta y_j, \Delta y_j f_y(x_i^*,y_j^*) \rangle \\ 
\Delta A_{ij} \approx \| s_1 \times s_2 \| \\ 
= \sqrt{ (1 + [ f_x(x_i^*,y_j^*)]^2 + [f_y(x_i^*,y_j^*)]^2 }\Delta x_i \Delta y_j \\ 
\end{align}

So if \fbox{$z = f(x,y), dA = \sqrt{1 + f_x^2 + f_y^2}$}.


\textbf{So the total pressure ends up being}
$\vec{F}_{tot} = p_{fluid} \iint (p \cdot \hat{n}) dA$
\begin{align}
= p_{fluid} \iint_{x^2 + y^2 \leq l^2} [ 1- \frac{f(x,y)}{l}] \hat{n}  \sqrt{ 1 +  [f_x]^2 +[f_y]^2 }dxdy  \\
f(x,y) = l - \sqrt{l^2-x^2-y^2} \\
\hat{n} = \langle \frac{x}{l},\frac{t}{l}, \frac{f(x,y)}{l} - 1 \rangle \\
f_x = \frac{x}{\sqrt{l^2 - x^2 - y^2}}, f_y = \frac{y}{\sqrt{l^2 - x^2 - y^2}}
\end{align}

And for the only-nonzero component, $\hat{k}$, this simplifies after a lot of hand-math to 
$F_{tot} = -p_{fluid} ( \iint_{x^2+y^2 \leq l^2} \sqrt{1 - (\frac{x^2 + y^2}{l^2})} dx dy) \hat{k}$

Side Note during solving: $dx dy \rightarrow r dr d\theta$.  
\begin{itemize}
\item TODO: This looks to be something to do with the determinant of the Jacobian matrix $F_i / x_j$.
\item Intuitively, consider that a patch $dx \cdot dy$ is a slice of a big disk which has dimensions $dr$ on the ray, $r d\theta$ on the arc.
\end{itemize}


\section{3.2: Flux Part I}
Main idea: Field lines are innumerable, so counting them through a surface makes no sense.  Instead, we'll use \textbf{flux} to help us measure charge pushed through a surface per unit time.

Example: If charge $q$ of mass $m$ in a field of $\vec{E} = E_0\hat{i}$ moves from origin along x towards R according to $\frac{d^2x}{dt^2} = \frac{q}{m}E_0$, then solving the diff eq. means that $x = \frac{q}{2m}E_0(\Delta t)^2 = R$.  This means we're pushing all charges within $ \frac{q}{2m}E_0(\Delta t)^2$ to the left of the disk through it.

Then, if we're considering a cylinder of base area $A$, mass density $\delta$, charge density $\rho$:
\begin{itemize}
\item Every test charge chunk $\Delta V$ within $\frac{\rho \Delta V}{2 \delta \Delta V}E_0(\Delta t)^2$ passes through.  That's the height.
\item Area is A, so total volume is  $\frac{\rho (\Delta t)^2}{2 \delta}E_0A$
\item Density of charge per volume is $\rho$, so total is $\frac{\rho^2 (\Delta t)^2}{2 \delta}E_0A$
\end{itemize}

Note: Tilting this forward from the z-axis by $\theta$ multiplies the cross-section area of the cylinder (now an ellipse) by $\cos (\theta)$.  Can work out the ellipse volume, or just note that each ``Riemann bar'' orthogonal to x-axis just got squished by  $\cos (\theta)$. 

So we define \textbf{flux} as amount of charge through a closed surface.  \fbox{$\Phi = (\vec{E} \cdot \hat{n})A$  } if $\vec{E}$ is a constant field.  (Units: joules/second/$m^2$, or watts/$m^2$), and \fbox{$\Phi = \iint_S (\vec{E} \cdot \hat{n})dA$} generally.

We can further note $(\vec{E} \cdot \hat{n}) = \|\vec{E}\| \cos(\theta)$ by last problem.

Example: Flux through an empty cube from the origin is necessarily 0 since every face cancels the other.

Another example: A square pyramind with top at $(0,0,1)$, sides at 1 on each axis:
\begin{itemize}
\item All the triangles will cancel in the x, y directions.
\item A triangle $(1, 0 ,0) (0, 1, 0), (0, 0, 1)$ has two displacement vectors $P_1P_3 = P_3 - P_1 = (-1, 0, 1), P_2P_3 = (0, -1 , 1)$.
\item $P_1P_3 \times P_2P_3 = (1, 1, 1) \rightarrow \hat{n} = \frac{(1, 1, 1) }{\sqrt{3}}$
\item $A = \frac{1}{2} \| P_1P_3 \times P_2P_3 \| = \frac{\sqrt{3}}{2}$
\item $\Phi = (\vec{E} \cdot \hat{n}) A = (E_0 \frac{1}{\sqrt{3}}) \frac{\sqrt{3}}{2} = \frac{E_0}{2}$
\item So total flux through these is $4 \cdot \frac{1}{2} E_0 = 2E_0$
\item However, the bottom has area $\sqrt{2}^2 = 2$ and flux $E_0$ , so total is 0!
\end{itemize}

\section{3.3: Flux Part II}

Note: 
\begin{itemize}
\item Charge ($q$) is the volts of the point charge.  Total charge $Q_{tot}$ is total charge inside some surface.
\item Electric field is sum of those point charges acting at a distance, and a is a single vector.
\item Flux is the sum of the electric field flowing through a surface.
\item Total charge $Q_{tot}$ of a surface is basically the sum of all the flux going in/out, except that it's that divided by some constant $\epsilon_0$. 
\end{itemize}
Note: $\vec{E}$ isn't usually constant, and the surface $S$ is usually curved.  So we need calculus to break up surface $S$ into small pieces $\Delta A_i$ and evaluate $\vec{E}_i$ there at that normal $\hat{n}_i$.  So

$\sum_{patches} (\vec{E}_i \cdot \hat{n}_i) \Delta A_i = \iint_S (\vec{E} \cdot \hat{n})) dA = \Phi$

Easy Example: If, say, $ (\vec{E} \cdot \hat{n}) = 1 $ everywhere, we're just looking at $\iint_S dA$, or the total surface area.

Another example.  Given:
\begin{itemize}
\item Real electric field law: $\vec{E} = \frac{q}{4\pi \epsilon_0} \frac{\vec{x}}{\|\vec{x}\|^3}$
\item Real observation: Total electric flux through a surface ($\Phi$) is proportional to total charge inside ($Q_{tot}$).  $\Phi = \iint_S (\vec{E} \cdot \hat{n}) \Delta A \propto Q_{tot}$
\item Then constant must be $\frac{1}{\epsilon_0} $. Why?
\begin{itemize}
\item On unit sphere, $\hat{n}= \frac{\vec{x}}{\| \vec{x}\|}$
\item So $\vec{E} \cdot \hat{n} = \frac{q}{4\pi \epsilon_0} \frac{\vec{x}}{\|\vec{x}\|^3} \cdot \hat{n}$
\item $= \frac{q}{4\pi \epsilon_0}$ since $\| \vec{x}  \|=1$ on unit sphere
\item Then $\Phi = \iint_S \frac{q}{4 \pi \epsilon_0} dA$
\item  $= \frac{q}{4 \pi \epsilon_0} 4 \pi$ by surface area of unit sphere
\item $= \frac{q}{\epsilon_0}$
\end{itemize}
\item Therefore, because all of the field goes through the surface (no matter the shape), \fbox{\textbf{Gauss's law} says $\iint_S (\vec{E} \cdot \vec{n}) dA = \frac{Q_{tot}}{\epsilon_0}$}
\end{itemize}

\textbf{Note}: Because (UNEXPLAINED!) symmetry of a contained \emph{ball} implies that, for distance $\rho$ from origin, $\vec{E} = E(\rho)\hat{\rho}$, the above works the same for a point charge or a uniform (contained) ball.

Example: For a big radius $R$ ball of charge $Q$ containing a small ball of radius $\rho$ with charge $Q_{tot}$, what must the charge $E(\rho)$ at any point be?

\begin{itemize}
\item Small charge $Q_{tot}$ is proportional to volume of the big charge $Q$ by $Q_{tot} = Q \frac{V_{small}}{V_{big}} = Q\frac{\rho^3}{R^3}$
\item $\frac{Q_{tot} }{\epsilon_0}$ = total charge = $\iint_S E(\rho)(\|\hat{\rho}\|)dA = E(\rho) \iint_S 1 dA  = E(\rho) 4 \pi \rho^2$
\item So $\frac{Q_{tot} }{\epsilon_0} = Q\frac{\rho^3}{R^3 {\epsilon_0}}  = E(\rho) 4 \pi \rho^2$
\item So $E(\rho) = \frac{Q}{4 \pi \epsilon_0} \frac{\rho}{R^3}$
\end{itemize}

Example: Infinite wire, x=y=0, charge per length is $\lambda$.  What's the magnitude of the field r units away?
\begin{itemize}
\item Use a cylinder.
\item What's the total charge of the cylinder?  Top and bottom are perpendicular to the field so can be ignored.
\item There's some function $E(r)$ which, time $\hat{r}$, is the field by symmetry.
\item $\Phi = \iint_{cylinder} (E(r) \cdot \hat{r})dA = E(r)  \iint_{cylinder} 1 dA  = E(r) 2 \pi r h$.
\item $\frac{Q_{tot}}{\epsilon_0} =  E(r) 2 \pi r h \Rightarrow E(r) = \frac{\lambda}{2\pi\epsilon_0 r}$
\end{itemize}


Example: Infinite plane, x=y=0, charge per area is $\sigma$.  What's the mangitude of the field at height $h$?
\begin{itemize}
\item  Use a cylinder again
\item What's the total charge of the cylinder?  Side is perpendicular to the field so can be ignored.  Looking at top and bottom, $\phi = 2EA + 2EA.$, where E is charge through the top.
\item $2EA = \frac{\sigma A}{\epsilon_0} \rightarrow E = \frac{\sigma}{2 \epsilon_0}$
\item Note: It appears it's height-invariant!
\end{itemize}

\section{3.4: Surface Integrals}
\begin{itemize}
\item Flux is a specific form of the general $\iint_S F da$.
\item dA is a patch of a parallelogram on the surface.  This is defined by corners $\vec{x}(u_0, v_0), \vec{x}(u_0, v_0) + \delta_u \vec{x}(u_0, v_0)$, and $\vec{x}(u_0, v_0) + \delta_v \vec{x}(u_0, v_0)$
\item Therefore, using the parallelogram area formula,$dA = \Delta_u \Delta_v \| \vec{x}_u \times \vec{x}_v \|$
\item Taking to the limit, this means the area is $\iint_D F(\vec{x}(u,v)) \|\vec{x}_u \times \vec{x}_v \| du dv$
\end{itemize}

Example: Sphere $x^2 + y^2 + z^2 = R^2$ surface area.  Take $\theta$ as angle around $\phi$ as angle from top of z axis.
\begin{itemize}
\item Parametrization $x = R \sin \phi \cos \theta,  y = R \sin \phi \sin \theta,  z = R \cos \phi$
\item $dx/d_{\theta} =  -R \sin \phi \sin \theta, dy/d_{\theta} =  R \sin \phi \cos \theta, dz/d_{\theta} =  0$
\item $dx/d_{\phi} =  R \cos \phi \cos \theta, dy/d_{\phi} =  R \cos \phi \cos \theta, dz/d_{\phi} =  -R \sin \phi$
\item After working it out, $dx/d_{\theta} \times dx/d_{\phi} = R^2 \sin \phi \langle -\sin \phi \cos \theta,  -\sin \phi \sin \theta, -\cos \phi \rangle$ 
\item Doing the math, $ \|dx/d_{\theta} \times dx/d_{\phi} \| = R^2 \sin \phi $
\item So $\int_{\theta = 0}^{2\pi} \int_{\phi = 0}^{\pi} 1 \cdot  R^2 \sin \phi = 2 \pi \int_{\phi = 0}{\pi} R^2 \sin \phi  = 2\pi R^2 [ -\cos \phi]_0^{\pi} = 4 \pi R^2$
\end{itemize}

Example: Parabaloid $z = 1 - x^2 - y^2, x^2 + y^2 \leq 1$
\begin{itemize}
\item Parametrization $x = R \sin \phi \cos \theta,  y = R \sin \phi \sin \theta,  z = R \cos \phi$
\item $dz/dx = \langle 1, 0, -2x \rangle, dz/dy = \langle 0, -1, -2y \rangle$
\item $\| dz/dx \times dz/dy \| = 1 + 4x^2 + 4y^2$
\item Area = $\iint_D 1 \cdot dA = \iint_D \sqrt{ 1 + 4x^2 + 4y^2} dx dy$
\item Change to polar, remembering this square depends on r: $\int_{\theta = 0}^{2\pi} \int_{r = 0}^{1} \sqrt{ 1 + 4r^2 \cos^2\pi  4r^2 \sin^2\pi} r dr d\theta = 2\pi\int_0^1 \sqrt{1 + 4r^2}rdr$
\item After working it out, this ends up being $[\frac{2}{3} \cdot \frac{1}{8}(4r^2 +1)^{\frac{3}{2}}]_0^1 = \frac{\pi}{6} (5\sqrt{5} - 1)$
\end{itemize}


Example: Torus $x(u,v) = [R+r\cos(u)]\sin(v)], y(u,v) = [R+r\cos(u)]\cos(v)], z = r \sin(u), u, v \in [0, 2\pi)$
\begin{itemize}
\item Already parametrized in polar, basically,
\item $d\vec{x}/du = \langle -r \sin (u) \sin (v),  -r \sin (u) \cos (v), r \cos (u)\rangle$
\item $d\vec{x}/dv = \langle Rcos(v) + r\cos(u)\cos(v), -R\sin(v) -r\cos(v)\sin(v), 0\rangle$
\item After lots of math, $\| d\vec{x}/du \times \vec{x}/dv \| = r(R+ r\cos(u))$
\item $\int_{u = 0}^{2\pi} \int_{v = 0}^{2 \pi}  r(R+ r\cos(u)du = 2 \pi r \int_{u = 0}^{2\pi}r(R+ r\cos(u))du$
\item $= 2\pi r [2 \pi R] = 4 \pi^2 R r$
\end{itemize}


Example: Center of mass of unit (hollow?) hemisphere sitting on origin.
\begin{itemize}
\item Center of mass for density $\rho$ is $\frac{\iint_S \vec{x} \rho dA}{\iint_S \rho dA}$
\item Obvious that $x, y$ center at zero.
\item For denominator, $\iint_S dA$ is just surface area, or half of $4 \pi 1^2 = 2 \pi$.
\item For numerator:
\begin{itemize}
\item Do typical $\theta, \phi$ parametrization.
\item $\vec{x}_{\theta} \times \vec{x}_{\phi} = \langle \sin^2(\phi) \cos(\theta), \sin^2(\phi)\sin(\theta), \sin(\phi)\cos(\phi) \rangle$
\item Pull out the $\sin(\phi)$ and the remaining norm is one, so $\| \vec{x}_{\theta} \times \vec{x}_{\phi}\| = \sin(\phi)$ 
\item $\int_{\theta = 0}^{2\pi} \int_{\phi = 0}^{\pi / 2} z \cdot dA = \int_{\theta = 0}^{2\pi} \int_{\phi = 0}^{\pi / 2} \cos(\phi)\sin(\phi) = \frac{1}{2}$
\end{itemize}

\end{itemize}

Example: Moment of inertia
\begin{itemize}
\item Formula: $I_z = M \iint_S(x^2 + y^2)dA$.
\item Object to spin: helicoid $\vec{x}(\theta,v) = \langle \theta \cos (v), \theta \sin (v), v\rangle \theta \in [0,R], v \in [0, 2\pi]$
\item Assumption for the problem: $\int_{\theta=0}^{\theta=R} \theta^2 \sqrt{1 +\theta^2}d\theta = 2$
\item Center of mass for density $\rho$ is $\frac{\iint_S \vec{x} \rho dA}{\iint_S \rho dA}$
\item Use polar coordinates $r, \theta$.
\item After computation, $\| \vec{x}_r  \times \vec{x}_{\theta} \| = \sqrt{1+r^2}$
\item $M \int_{r=0}^{r=R} \int_{\theta = 0}^{\theta=2\pi}  \sqrt{1+r^2} (r^2 \cos^2(\theta) + r^2 \cos^2(\theta)) dA = M \iint  \sqrt{1+r^2} r^2 = 2\pi M \iint  \sqrt{1+r^2} r^2  = 4 \pi$ by hint
\end{itemize}


Example: Flux through unit hemisphere
\begin{itemize}
\item Formula: $\Phi = \iint_S (\vec{E} \cdot \vec{n}) dA = \iint_S FdA$
\item Field: $\vec{E} = \langle yz, xz, xy \rangle$
\item Use polar coordinates
\item \textbf{Base}: $\hat{n} = -\hat{k}$ so $\langle yz, xz, xy \rangle \cdot \langle 0, 0, -1 \rangle = -xy$ It's clear by symmetry that $\iint_{u^2 + v^2 \leq 1} - xy dx dy = 0$
\item \textbf{Top}: Set $u = \theta \in [0, 2 \pi), v = \phi \in [0, \frac{\pi}{2}]$.  
\item As usual, $dA = \| \vec{x}_u \times \vec{x}_v \| = \sin(v)$.
\item Norm just points out from the center: $\hat{n} = \langle  \cos(u)\sin(v), \sin(u)\sin(v), \cos(v) \rangle $ 
\item $\vec{E} = \langle \sin(u)\sin(v)\cos(v), \cos(u)\sin(v)\cos(v), \cos(u)\sin^2(v)\sin(u)\rangle$ 
\item So $\vec{E} \cdot \hat{n} = 3\cos(u)\sin(u)\cos(v)\sin^2(v)$
\item Looking at this, this is really $\int_{u = 0}^{u = 2\pi} k(v) \sin^2(u)$  for some $k(v)$, so this will be 0.
\item Therefore, total flux is zero, and by Gauss's law, total field contained inside has to be 0 too.
\end{itemize}

Example: Field $\vec{E} = \ln(x^2 + y^2)\langle x, y, 0\rangle$ through $R$-wide cylinder, height $h$
\begin{itemize}
\item Parameterize: $x = r \cos \theta, y = r \sin \theta, z = z$
\item \textbf{Top/Bottom}: $\hat{n} = \langle 0, 0, 1 \rangle, \vec{E} = f(x,y) \langle x, y, 0\rangle \rightarrow \hat{n} \cdot \vec{E} = 0$
\item \textbf{Side}: $\hat{n} = \frac{1}{R} \langle R \cos (\theta), R \sin (\theta), 0 \rangle$
\item $\Phi = \iint_{cylinder} \frac{1}{R} \langle R \cos (\theta), R \sin (\theta), 0 \rangle \cdot  \langle R \cos (\theta), R \sin (\theta), 0 \rangle \ln(R^2 \cos^2(\theta) + R^2 \sin^2(\theta))$
\item $= R \iint_{cylinder} \ln(R^2) + \ln(\cos^2(\theta)+\sin^2(\theta)) = R \cdot 2 \ln (R) \cdot h \cdot 2 \pi R = 4 \pi R^2 \ln(R)h$
\end{itemize}

Example: Field $\vec{E} = e^{-x^2-y^2-z^2}\vec{x}$ with sphere S at radius R, setting $\epsilon_0 = 1$
\begin{itemize}
\item Parameterize: $x = R \cos (\theta)\sin(\phi), y = R\sin (\theta)\sin(\phi), z =R \cos(phi)$
\item $\hat{n} = \langle \cos (\theta)\sin(\phi), y =  \sin (\theta)\sin(\phi), z = \cos(phi) \rangle$
\item $\vec{x} = R \hat{n}$, so $\vec{E} \cdot {n} = R e^{-R^2}$
\item $R \iint_{sphere} e^{-R^2} = 4 \pi R^3 e^{-R^2}$
\end{itemize}

\textbf{Note}: In the future we write \fbox{$\hat{n}dA = \vec{dA}$}

\subsection{3.5: Divergence part I}

\textbf{Main idea}: Last chapter was all about having field $\vec{E}$ and wanting to figure out $Q_{tot}$ (or $\frac{\phi}{\epsilon_0}$). Usually, we have the charge distribution Q and want to figure out $\vec{E}$.  Most of the field derivation from 3.3 was through tricks for highly symmetric spaces (infinite line, infinite plane, uniform ball, etc.)

\textbf{Point}: The flux through a sphere in a uniform field is zero.  Why?  Move the center point to the origin, rotate so field is $\hat{k}$ (both don't change the flux), and consider that what goes out at $\langle x, y, z \rangle$ comes in at $\langle x, y, -z \rangle$.  This same argument applies for $\iint_{S = sphere} \hat{n}_i \hat{n}_j dA$, where $i, j$ are components in $\{x, y, z\}$.


 However, if $i = j$, then $\iint_S \hat{n}_i \hat{n}_j dA = \iint_S \hat{n}^2_i = \frac{4}{3}\pi R^2$, since $\iint_S (\hat{n}_x^2 + \hat{n}_y^2 + \hat{n}_z^2) dA = \iint_S 1 dA = 4 \pi R^2$, so each of the components must be a third of that.

\subsubsection{Defining Divergence}
Remember that, in Gauss's law $\frac{Q}{\epsilon_0} = \iint_S \vec{E} \cdot \vec{dA}$, we're using information about $\vec{E}$ spread out over surface $S$. We can also shrink this to a smaller surface.

Shrinking to a point $\vec{P}$, $\lim_{R \rightarrow 0} \frac{1}{4 \pi R^3}  \iint_S \vec{E} \cdot \vec{dA} = \frac{Q_{tot}}{\epsilon_0 4 \pi R^3} = \frac{\rho(\vec{P})}{\epsilon_0}$. \emph(This works by dividing both sides by volume of a sphere)

\textbf{Deriving Divergence}: Computing $\lim_{R \rightarrow 0} \frac{1}{4 \pi R^3}  \iint_S \vec{E} \cdot \vec{dA}$
\begin{itemize}
\item $\iint_S \hat{n}_i \hat{n}_j dA  = 0$ if $i \ne j$
\item $\iint_S \hat{n}_i \hat{n}_j dA  = \frac{4}{3} \pi R^3 $  if $i = j$
 \item Use linear approximation with Jacobian $D = \frac{\delta E_i}{\delta x_j}$, $\vec{E}(\vec{x}) = \vec{E}(\vec{P}) + D\vec{E}(\vec{P})(\vec{x} - \vec{P})$
 \item $\iint_S \vec{E}(\vec{P}) = 0$ for any constant.  (think of the flux of a sphere in a constant field as above)
\item This leaves $D\vec{E}(\vec{P})(\vec{x} - \vec{P}) \cdot \hat{n} = \sum_{i,j} \hat{n}_i [\vec{x}-\vec{P}]_j D\vec{E}(\vec{P})_{ij}$
\item Since it's a sphere, the normal $\hat{n} = \frac{\vec{x} - \vec{P}}{R}$
\item Therefore $D\vec{E}(\vec{P})(\vec{x} - \vec{P}) \cdot \hat{n} = R \sum \hat{n}_i \hat{n}_j  D\vec{E}(\vec{P})_{ij}$ (swap $R \hat{n}_j$ for $[\vec{x} - \vec{P}]_j$)
\item These terms are all 0 except where $i = j$, so $D\vec{E}(\vec{P})(\vec{x} - \vec{P}) \cdot \hat{n} = \frac{4}{3} \pi R^2 \times R \times
   [\frac{\delta E_x}{\delta x} +  \frac{\delta E_y}{\delta y} +  \frac{\delta E_z}{\delta z}]$
\item This equals $\lim_{R \rightarrow 0} \frac{1}{4 \pi R^3}  \iint_S \vec{E} \cdot \vec{dA}$ so eliminating the sphere volume gives us
\\
   \fbox{$\frac{\rho(\vec{P})}{\epsilon_0} = [\frac{\delta E_x}{\delta x} +  \frac{\delta E_y}{\delta y} +  \frac{\delta E_z}{\delta z}] = \nabla \cdot \vec{E}$}
\end{itemize}

We can think of the divergence  $\nabla$, also like an operator: 
\\
\fbox{$\nabla \cdot \vec{F} = \nabla \cdot (F_x\hat{i} + F_x\hat{j} + F_x\hat{k}) = 
(\frac{\delta}{\delta x} \hat{i} + \frac{\delta}{\delta y} \hat{j} + \frac{\delta}{\delta z} \hat{k}) \cdot (F_x\hat{i} + F_x\hat{j} + F_x\hat{k})$}

\textbf{Shifting to Cylindrical Coordinates}: If instead we want to describe $\vec{F} = \vec{F}_r \hat{r} + \vec{F}_{\theta} \hat{\theta} + \vec{F}_z \hat{z}$, we have 
\fbox{$\nabla \cdot \vec{F} = \frac{1}{r} \frac{\delta r F_R}{\delta r} + \frac{1}{r}\frac{\delta F_{\theta}}{\delta \theta} + F_z \frac{\delta F_z}{\delta z}$}.  How to derive?
\begin{itemize}
\item Note identities $\hat{r} = \cos(\theta) \hat{i} + \sin(\theta) \hat{j}, \hat{\theta} = -\sin(\theta) \hat{i} + \cos(\theta) \hat{j}$.  If $\theta = 0$, these point right and up, corresponding to $\hat{i}, \hat{j}$.  If $\theta$ rotates, these do too.
\item $F_x \hat{i} + F_y \hat{j} + F_z \hat{k} = \vec{F}  = (F_r( \cos(\theta) \hat{i} + \sin(\theta) \hat{j}) + F_{\theta}(-\sin(\theta) \hat{i} + \cos(\theta) \hat{j}) + F_z \hat{k}$
\item Rearrange so that $\vec{F} =  F_x\hat{i} + F_x\hat{j} + F_x\hat{k} = (F_r\cos(\theta)   + F_{\theta}(-\sin(\theta)) \hat{i} +( F_r \sin(\theta) + F_{\theta}\cos(\theta))\hat{j} + F_z\hat{k}$.
\item Compute $\frac{\delta}{\delta x} = \frac{\delta}{\delta r} \frac{\delta r}{\delta x}  + \frac{\delta}{\delta \theta} \frac{\delta \theta}{\delta x} =
\frac{\delta}{\delta x} = \cos(\theta)  \frac{\delta}{\delta r}  - \frac{\sin(\theta)}{r} \frac{\delta}{\delta \theta} $.  
\begin{itemize}
\item The second term: $\frac{d\theta}{dx} = \tan^{-1} (y/x)) = \frac{y}{1+y^2/x^2} * \frac{-1}{ x^2} = -\frac{r\sin(\theta)}{r^2(\sin^2 + \cos^2)} = -\frac{\sin (\theta)}{ r}$
\end{itemize}
\item Do something similar for similar for $\frac{d}{dy}$ in the second term.
\item Combine and shake it out.
\end{itemize}



\textbf{Shifting to Spherical Coordinates}: Using a similar process, we get 
\\
\fbox{
$\nabla \cdot \vec{F} = \frac{1}{\rho^2} \frac{\delta (\rho^2 F_{\rho})}{\delta\rho} + \frac{1}{\rho \sin(\phi)} \frac{\delta}{\delta \phi} (\sin(\phi)F_{\phi}) + \frac{1}{\rho \sin(\phi)} \frac{\delta F_{\theta}}{\delta \theta}$}




\subsection{3.6:  Divergence Part 2}

Example: Compute divergence of electric field $E = \frac{Q}{4 \pi \epsilon_0}\frac{\vec{x}}{\| \vec{x} \|^3}$ outside radius R.
\begin{itemize}
\item $\frac{\delta E_x}{\delta x} (\frac{Q}{4 \pi \epsilon_0}\frac{x}{(x^2 + y^2 + z^2)^\frac{3}{2}}) = v\frac{-2x^2 + y^2 + z^2}{(x^2 + y^2 + z^2)^\frac{5}{2}}$.
\item Symmetrical for $\frac{\delta E_y}{\delta y}, \frac{\delta E_z}{\delta z}$
\item Sums to 0.
\end{itemize}

Example: Compute divergence of electric field $E = \frac{Q}{4 \pi \epsilon_0}\frac{\vec{x}}{R^3}$ inside radius R.
\begin{itemize}
\item $\frac{\delta E_x}{\delta x} (\frac{Q}{4 \pi \epsilon_0}\frac{x}{R^3}) =  \frac{Q}{4 \pi \epsilon_0}\frac{1}{R^3}$
\item Symmetrical for $\frac{\delta E_y}{\delta y}, \frac{\delta E_z}{\delta z}$
\item Sums to $\frac{Q}{4 \pi \epsilon_0}\frac{3}{R^3}$
\end{itemize}

So, the divergence of an electric field is proportional to $\frac{Q}{R^3}$ inside the sphere, and 0 outside the sphere.  

So, divergence at a point intuitively measures \textbf{how much the field spreads out} or sinks into the point.  For electric charge, 
\fbox{$\nabla \cdot \vec{E} = \frac{\rho}{\epsilon_0}$} means that at that point, the spready-ness is proportional to the charge.

Example: if $\epsilon_0 = 1$ and the field is $\vec{E} = x \hat{i} + 2y \hat{j} + z \hat {k}$, how much charge is in the $[0,1] \times [0,1] \times [0,1]$ box?
\begin{itemize}
\item	Answer: $\rho = \nabla \cdot \vec{E} = 1 + 2 + 1 = 4$.  So 4 units.
\end{itemize}


Another Example: if $\vec{E} = \sin(yz) \hat{i} + \sin(xz) \hat{j} + \sin(xy) \hat {k}$ in some complicated surface, then what?
\begin{itemize}
\item	Noticing that $\nabla \cdot \vec{E} = 0$ shows you this is 0 no matter the shape of the region.  This means \emph{the vectors pointing into the region (in fact, any part of the space) are balanced out by those pointing out from the region.}
\end{itemize}


\subsection{3.7:  The Divergence Theorem}

\textbf{The Divergence Theorem} falls out of equating finding charge $Q$ with a double integral over a bounded surface with the triple integral of the contained volume:
\begin{itemize}
\item $\frac{Q}{\epsilon_0} = \iint_S \vec{E} \cdot \vec{dA}$ (Gauss's law)
\item $\Rightarrow \nabla \cdot \vec{E} = \frac{\rho}{\epsilon_0}$ within R (Proved Divergence equivalent from last section)
\item $Q = \iiint_R \rho dx dy dz$ (Just integrating charge over volume)
\item $\Rightarrow Q = \iiint_R \rho dx dy dz = \epsilon_0  \iiint_R \nabla \cdot \vec{E} dx dy dz = \epsilon_0 \iint_S \vec{E} \cdot \vec{dA}$ 
\item \fbox{$\Rightarrow   \iint_S \vec{E} \cdot \vec{dA} = \iiint_R \nabla \cdot \vec{E} dx dy dz$} (Divergence Theorem)
\end{itemize}


\textbf{Smooshy thought}: This looks like another version of Fundamental Theorem of Calculus.  The integral of the function evaluated at the boundaries is the same as the function summed inside the boundary.


\textbf{Proving Divergence Generally}: We're gluing micro-cubes together and not changing the total flux.  This means any surface is the flux going in and out of its "skin".

\begin{itemize}
\item Note that since the flux outward through a cube face is the negative of it inward, gluing two cubes together on this face means we're summing the total fluxes.
\item Do this for tiny cubes approximating the surface we care about.
\item In a cube centered on point $P, F \approx \vec{F}(P) + D\vec{F}(P)(\vec{x} - \vec{P})$.
\item $\iint_S \vec{F}(P)\vec{dA} = 0$ since it's constant, since every face $i$ has normal $\hat{n}_i$, and a partner of equal size with normal $-\hat{n}_i$.
\item However,  for a cube of side $\epsilon$ the flux through, say, Face I ($x = \epsilon + P$) is  $\iint_S  D\vec{F}(P)(\vec{x} - \vec{P})$ ends up being $\frac{\delta F_x}{\delta x} 4 \epsilon^3$, since:
\begin{itemize}
\item Consider side $x = p_x + \epsilon$
\item $D\vec{F}(P)(\vec{x} - \vec{P}) \cdot \hat{n} =  [D\vec{F}(P)]_{xx} (x - p_x) + [D\vec{F}(P)]_{xy} (x - p_y) + [D\vec{F}(P)]_{xz} (z - p_z)$.
\item So, the functons that consider the inputs of y, z don't matter.
\item So $\iint_{Face I} (y - p_y)dA = 0$  around $p_y$ by symmetry.  Same for z on that face.
\item But for $x$, $\iint_{Face I} (x - p_x)dA = \int_{p_y - \epsilon}^{p_y + \epsilon} \int_{p_z - \epsilon}^{p_z + \epsilon}  \epsilon dy dz = 4 \epsilon^3$ 
\item $D_{ij}\vec{F}(P)$ is constant for all $i, j \in \{x, y, z\}$, so this face is then $\frac{\delta F_x}{\delta x} 4 \epsilon^3$.
\item Summing the opposite face (with the same flux), yields $\frac{\delta F_x}{\delta x} 8 \epsilon^3$ = $\frac{\delta F_x}{\delta x} V$.
\item Summing across the other faces yields $\frac{\delta F_x}{\delta x} V + \frac{\delta F_y}{\delta y} V + \frac{\delta F_z}{\delta z} V$.
\end{itemize}

Finally, this shows the \textbf{flux on one of these microcubes is} \fbox{ $\nabla \cdot \vec{F}(P) V$.}

In total, the \textbf{divergence theorem}: \fbox{ $\iint_{\delta C} \vec{F} \cdot \vec{dA} \approx \nabla \cdot \vec{F}(P) V \approx \iiint_C \nabla \cdot \vec{F} dx dy dz$}

\end{itemize}


\textbf{Example of using divergence to calculate flux}: Unit hemisphere with $\vec{E} = \langle yz, xz, xy \rangle$:
\\
\emph{Answer}: $\nabla \cdot \vec{E} = \frac{\delta}{\delta x} yz +  \frac{\delta}{\delta y} xz +  \frac{\delta}{\delta z} xy  = 0$

\textbf{Example of using divergence to calculate flux}: Cylinder of radius $R$, height $h$, sitting on $z = 0$ with $\vec{E} = \ln(x^2 + y^2) \langle x, y, 0 \rangle$:
\\
Answer:
\begin{itemize}
\item $\frac{\delta}{\delta x} E_x =  \ln(x^2 + y^2) + \frac{2x^2}{x^2 + y^2}$.  Similar for $E_y$.
\item Transform to polar: $E_x + E_y = \ln(r^2) + \frac{2 r^2 \cos(\theta)^2 + r^2 \sin(\theta)^2}{r^2} = 2\ln(r) + 2$
\item Set up the integral, remembering the Jacobian: $\Phi = 2 \pi \int_{z = 0}^h \int_{r = 0}^{r=R} [ 2\ln(r) + 2 ] r dr d\theta$
\item Working it out, with identity $\int x \ln(x) = -\frac{x^2}{4} + \frac{x^2}{2}ln(x)$, you get $\Phi = 4 \phi R^2 ]\ln(R) h$
\end{itemize}

\textbf{Example of using divergence to calculate flux}: Unit sphere at origin with $\vec{E} = (x^3 + y^3)\hat{i} +  (z^3 + y^3)\hat{j} +  (x^3 + z^3)\hat{k}$ 
\\
Answer:
\begin{itemize}
\item $ E_x + E_y + E_z =  3x^2 + 3y^2 + 3z^2  = 3 \iiint \rho^2 dx dy dz$
\item Each $d \rho$ is a sphere of volume $4 \pi f(\rho)^2 = 4 \pi \rho^4$
\item So the integral is $12 \pi \int_{\rho = 0}^{\rho = 1} \rho^4 = \frac{12\pi}{5}$ 
\end{itemize}


\textbf{Example of using divergence to calculate flux}: $\vec{F} =  (\cos(z)+x^2)\hat{i}, + (xe^{-z})\hat{j} + (\sin(y) +x^2z)\hat{k}$  on parabaloid $z = x^2+y^2, z \leq 4$ with top $x^2+y^2 \leq 4, z=4$
\begin{itemize}
\item $\iiint_R \nabla \cdot \vec{E} = \int_{z = r^2}^4 \int_{x^2 + y^2 = 0}^2 (y^2 + x^2) dx dy dz$
\item $\iiint_R \nabla \cdot \vec{E} = \int_{\theta = 0}^{2 \pi} \int_{r=0}^2 \int_{z=r^2}^4 (r^2\cos^2(\theta) + r^2\sin^2(\theta)) r d\theta dr dz = r^3 d\theta dr dz$
\item $= 2 \pi  \int_{r=0}^2 4r^3 - r^5 = 2\pi[r^4 - \frac{r^6}{6} ]_0^2 = \frac{32}{3}\pi$.
\end{itemize}

\textbf{What's crazy}: Evaluating divergence of a point charge $\vec{E} = \frac{Q}{4 \pi \epsilon_0}\frac{\vec{x}}{\|\vec{x}\|^3}$
\begin{itemize}
\item $\frac{\delta}{\delta x} E_x = \frac{Q}{4 \pi \epsilon_0} \frac{\delta}{\delta x} x (x^2+y^2+z^2) = \frac{-2x^2 + y^2 + z^2}{(x^2 + y^2 + z^2)^\frac{5}{2}}$
\item $E_y, E_z$ follow symmetrically.
\item The sum is infinite at the origin and zero everywhere else
\item Therefore, they had to invent a $\delta$ function that is infinite at origin, 0 elsewhere, and $\iiint_{\mathbb{R}^3} \delta({\vec{x}}) = 1$
\end{itemize}

\subsection{3.8: Divergence and Fluids}

Looking back to section 3.1, this hydrostatic force function should follow similar patterns to flux: \fbox{$\vec{F}_{tot} = \iint_S p \hat{n} dA$}.

Extended Example: A round ball of radius $R$, center at depth $h$ , with force $\vec{F}_{tot} = p_0 \iint_S [1 - \frac{z}{h}] \hat{n} dA$.
\begin{itemize}
\item $\hat{n} = \frac{\langle x, y, z \rangle}{R}$
\item For the integral, note that x, y are completely symmetric around z axis, so they contribute 0.
\item For the integral, we're then looking at  $\frac{p_0}{R} \iint_S [1 - \frac{z}{h}]zdA$
\item Use spherical coordinates: $\frac{p_0}{R} \iint_S [1 - \frac{R\cos(\phi)}{h}] R\cos(\phi)dA$
\item Working through $dA = \sqrt{1 + f_x^2 + f_y^2}dx dy$ with $f = R - \sqrt{R^2 + x^2 + y^2}$, we get $dA = \frac{dxdy}{\sqrt(1 - \frac{x^2 + y^2}{R^2})}$
\item This $dA$ term, in spherical coordinates, becomes $R^2 \sin(\phi)d\phi d\theta$
\item Combining and substituting $u = \cos(\phi)$, this integral is $-\frac{4\pi R^3}{3}\frac{p_0}{h}\hat{k}$
\end{itemize}

The neat idea: $F_{tot} = \frac{4\pi R^3}{3} \times -\frac{p_0}{h}\hat{k}$ is really ``ball's volume'' times a constant.
\begin{itemize}
\item $\frac{4\pi R^3}{3} \times  -\frac{p_0}{h}\hat{k}$
\item $= \iiint_B 1 dx dy dz  \times  -\frac{p_0}{h}\hat{k}$

\item $= \iiint_B( -\frac{p_0}{h})\hat{k} dx dy dz$, with $p = p_0[1-\frac{z}{h}]$
\item $=  \iiint_B(\frac{\delta p}{\delta z}) \hat{k} dx dy dz$
\item $=  \iiint_B \nabla \cdot p dx dy dz$
\item So the upshot is the divergence theorem again:\fbox{ $\iint_{S} p \hat{n}dA = \iiint_B \nabla \cdot p dx dy dz = \iiint_B \nabla \cdot p d\vec{x}$}
\end{itemize}

Final example three ways: ``oxygen flow'' (really, flux) through ball of radius R at origin, under field $J = j_0\hat{i}$.
\begin{itemize}
\item Intuitive: what comes in at (-x, y, z)  goes out at (x, y, z), so total is zero.
\item Flux integra under spherical: $\iint \vec{F} \hat{n} dA = \iint_S j_0\hat{i} \cdot \frac{\langle x,y,z \rangle }{R} dA = \int_{\theta = 0}^{2\pi}\int_{\phi = 0}^{\pi} j_0 \cos(\theta)\sin(\phi) R^2 \sin(\theta) d\theta d\phi = 0 = \frac{\pi j_0 R^2}{2} \int_{\theta = 0}^{2\pi} \cos(\theta) d\theta = 0$
\item Divergence: $\nabla \cdot \vec{J} = \frac{\delta}{\delta x} j_0 + 0 + 0 = 0, so \iiint_B 0 = 0$.
\end{itemize}

\subsection{3.9: Flows and Divergence}

Main idea: Divergence ($\nabla \cdot \vec{V}$) measures how much the flow changes volumes at that point.

Example: What is the function described by field of velocity vectors $\vec{V}(\vec{x}) = \langle -y, x \rangle$?
\begin{itemize}
\item $x'(t) = -y, y'(t) = x$
\item $\Rightarrow x''(t) = -y' = -x, y''(t) = x' = -y$
\item $\Rightarrow x = A\cos(t) + B\sin(t), y = C\cos(t) + D\sin(t)$, work it out to $x = \cos(t), y = \sin(t)$
\end{itemize}

Idea: dump a $dA=s_1$ by $s_2 =  \Delta x\hat{i} \times \Delta y\hat{j} $rectangle into the flow and see how it deforms over time.  Over a long time, it'll distort a lot, but consider for $\Delta t$:
\begin{itemize}
\item dA has sides of length $\Delta x, \Delta y$ but area of dA: $\| s_1 \times s_2\|$ (cross product norm is parallelogram area)
\item What is side $s_1$ after $\Delta t$? The starting point plus (how the endpoint moves minus how the start point moves): $\vec{s_1} + \Delta t[\vec{V}(x_0 + \Delta x, y_0) - \vec{V}(x_0, y_0)]$
\item Expanding the iterated $s_1$, which we call $s_1'$ out: $\vec{s_1}' = \Delta x[(1 + \Delta t\frac{\delta V_x}{\delta x})\hat{i} + \Delta t \frac{\delta V_y}{\delta x} \hat{j}]$.  Do the same for $s_2'$ and work out in 3D: $s_1' \times s_2' \approx \Delta x \Delta y [\hat{k} + \Delta t[(V_x)_x + (V_y)_y]\hat{k}$
\item We end up with $s_1' \times s_2' \approx \Delta x \Delta y [1 + \Delta t \nabla \cdot \vec{V}]\hat{k}$, so vs. original area $\Delta x \Delta y$, the ratio is $1 + \Delta t \nabla \cdot \vec{V}$
\item This means \textbf{divergence $\nabla \cdot \vec{V}$ is proportional to the change in area due to the flow}!
\end{itemize}

An \textbf{incompressible} field preserves volume under flow (so $\nabla \cdot \vec{V} = 0)$, like $\langle y, z, x\rangle, \langle 0, 2\sqrt{x}, 0 \rangle,  \langle x, y, -2z \rangle$.

A cool interactive on the page shows how a sphere migrating its points via $\langle x, y, z \rangle$ grows and changes volume, while one under $0.3 \langle  y, z , x \rangle$  distorts but doesn't.

\section{Chapter 4: Work, Line Integrals, Stokes's Theorem}
\subsection{4.1: Work Part I}

\textbf{Energy U} for charge $q$ in field $\vec{E}$ is \fbox{$-q\vec{E} = \nabla U$}.
\\
\emph{Example}: Right-pointing constant field $\vec{E} = E_0\hat{i}$ means $(\frac{\delta}{\delta x}\hat{i} + \frac{\delta}{\delta y}\hat{j} + \frac{\delta}{\delta z}\hat{k})U = -qE_0\hat{i} \Rightarrow U = -qE_0x$

\textbf{Work} is change in energy, e.g. $W_{field} = -[U(\vec{x_f} )- U(\vec{x_0})]$.  If the charge flows with the field, then the field is doing positive work.  If the charge flows against the field, the field is doing negative work.  So in the above case, $W_{field} = -[U(x_f,0,0) )- U(x_0, 0, 0)]  \Rightarrow -[-qE_0x_f - -qE_0x_0 = qE_0(x_f - x_0)$.

Continuing the example, moving from $(x_0,0, z_0) $ to $(x_f,0, z_f)$ in $\vec{E} = E_0\hat{i}$: only the x-coordinate affects the energy, so $U(x) = -qE_0x$ and $W_{field} = qE_0(x_f - x_0)$.  If $s$ is the distance between the two, then $\cos(\theta) = \frac{|x_f - x_0|}{s} \Rightarrow W_{field} = qE_0 s \cos(\theta)$.

Expanding the example, consider $\vec{E}(\vec{x}) = \vec{E_0}$, a constant that may not be aligned just with x-axis.  Then $-q\vec{E} = \nabla U \Rightarrow \nabla (-q\vec{E_0} \vec{x} + C) = -q\vec{E} \Rightarrow U = -q \vec{E_0}\vec{x} +C.$ Can also solve the diff eq, more generally.  This also means that, still, $W_{field} = q\vec{E_0}(\vec{x_f} - \vec{x_0})$.

Big idea: Though most fields aren't constant, they are near-constant between a small displacement $\Delta x$.  
\begin{align}
W_{field} = q\vec{E} \cdot \Delta \vec{x} \\
= \sum q\vec{E}(\vec{x}(t_i)) \cdot [\vec{x}(t_{i+1}) - \vec{x}(t_i)] \\
= \sum q\vec{E}(\vec{x}(t_i)) \frac{\Delta x(t_{i+1})} {\Delta t}\Delta t 
= \int_a^b q \vec{E}(\vec{x}(t)) \cdot \frac{d\vec{x}}{dt}dt
\end{align}

So we can take a \textbf{line integral} to measure work over a path in a field.

\subsection{4.2: Work Part 2}

In general, the \textbf{work} for moving a charge $q$ along path $\vec{x}(t)$ through field $\vec{E}(\vec{x})$ is \fbox{$W = \int_a^b q\vec{E}(\vec{x}(t)) \cdot \frac{d\vec{x}}{dt} dt$}.  More generally, $q\vec{E}$ is just a force, so we're looking at \fbox{$W_{field} = \int_a^b \vec{F}(\vec{x}(t)) \cdot \frac{d\vec{x}}{dt} dt$}

Example: $\vec{x}(t) = \langle t+1, t+1, t \rangle, t \in [0, 8], \vec{E}(x,y,z) = \frac{\lambda}{2 \pi \epsilon_0} \frac{x \hat{i} + y \hat{j}}{x^2+y^2}$
\begin{align}
W = \frac{q \lambda}{2 \pi \epsilon_0} \int_{t=0}^8 \frac{\langle t+1, t+1, t\rangle \cdot \langle 1, 1, 0 \rangle}{2(t+1)^2} \\
=\frac{q \lambda}{2 \pi \epsilon_0} \int_{t=0}^8 \frac{1}{t+1} \\
= \frac{q \lambda}{2 \pi \epsilon_0} \int_{u=1}^9 \frac{1}{u}  \\
= \frac{q \lambda}{2 \pi \epsilon_0} \ln(9)
\end{align}

However, what happens if we keep the endpoints but change the path?

Extended example: $\vec{x}(t) = \langle \sqrt{2}t \cos(\frac{\pi t}{4}), \sqrt{2}t \sin(\frac{\pi t}{4}), t-1\rangle$, still with $\vec{E}(x,y,z)= \frac{\lambda}{2 \pi \epsilon_0} \frac{x \hat{i} + y \hat{j}}{x^2+y^2}$.  
\begin{align}
\vec{E}(\vec{x}(t))  \cdot \frac{d\vec{x}}{dt} = ... = 2t \\
x^2 + y^2 = 2t^2 \\
W =  \frac{q \lambda}{2 \pi \epsilon_0} \int_{1=1}^9 \frac{1}{t}  \\
= \frac{q \lambda}{2 \pi \epsilon_0} \ln(9)
\end{align}

So it looks like there \emph{might be path independence here}.  Consider: rubbing your hands together - is there more work done oscillating and ending at the initial position than doing nothing? (Clearly yes).

Example: Work due to friction. $\vec{F} = -\gamma \frac{d\vec{x}}{dt}, \gamma > 0$.  Move from $(0,0,0) \rightarrow (1,0,0)$ via $\vec{x}(t) = \langle t, 0, 0 \rangle$.  If $\gamma = 1$, what is $\int_a^b \vec{F}(\vec{x}(t)) \cdot \frac{d \vec{x}}{dt} dt$?

Answer: $-\int_{t=0}^1 \frac{d \vec{x}}{dt}^2dt = - \int_{t=0}^1 \langle 1, 0, 0 \rangle ^2 = -1$  We lose 1 unit of energy.

To illustrate that there's not path independence always, consider an oscillating object following $\vec{x}(t) = \langle t, \sin(n\pi t), \sin(n \pi t) \rangle, t \in [0,1]$.  Take $\gamma = 1$ so $\vec{F} = \vec{x}'(t).$

Answer: $- \int_{t=0}^1 \vec{x}'(t) \cdot \vec{x}'(t) dt = - \int_{t=0}^1 (1+2n^2\pi^2\cos^2(n \pi t)) dt =  -\int_{t=0}^1(1+ 2n^2\pi^2 \frac{1}{2}(1 + \cos(2 n \pi t))  dt = -\int_{t=0}^1 (1 + n^2\pi^2+2n^2\pi^2\cos(2n\pi t))dt = -[ t + tn^2\pi^2 =+n\pi \sin(2 n \pi t)]_0^1 = -[1 + n^2\pi^2]$

Example: Spring force $\vec{F} = -\frac{\|x - l_0\|}{\|x\|} \vec{x}$ along path $\vec{x}(t) = \langle 0, 1-t, 2t\rangle, t \in [0,1]$

Answer: $\int_{t=0}^1 \vec{F}(\vec{x}(t)) \frac{d \vec{x}}{dt}dt = - \int_{t=0}^1 \frac{\sqrt{5t^2 - 2t + 1} - 1}{\sqrt{5t^2 - 2t + 1}} \langle 0, 1-t, 2t \rangle \cdot \langle 0, -1, 2 \rangle= \int_0^1 [1 - (5t^2 - 2t + 1)^\frac{-1}{2}](-1 + 5t)dt = -[-t + \frac{5}{2}t^2 - (5t^2 - 2t +1)^\frac{1}{2}]_0^1 = -\frac{1}{2}$

Example: \emph{SAME} Spring force $\vec{F} = -\frac{\|x - l_0\|}{\|x\|} \vec{x}$ and \emph{SAME} endpoints but along path $\vec{x}(t) = \langle 0, \cos(t), 2\sin(t) \rangle,  t \in [0,1]$

Answer: $\int_{t=0}^{\pi / 2} \vec{F}(\vec{x}(t)) \frac{d \vec{x}}{dt}dt  = - \int_{t=0}^{\pi / 2} (1 - \frac{1}{\|x\|})\langle 0, \cos(t), 2\sin(t) \cdot \langle 0, -\sin(t), 2\cos(t) \rangle= -3  \int_0^{\pi / 2} \sin(t)\cos(t)(1 - \frac{1}{\sqrt{1+3t^2}}dt = 3 \int_{t=0}^{\pi / 2} - \sin(t)\cos(t) + 3 \int_{t=0}^{\pi / 2} \sin(t)\cos(t) \frac{1}{\sqrt{1+3\sin^2(t)}} = 3[\frac{1}{2}\cos^2(t)]_0^{\pi / 2} + 3[\frac{2}{6}(1+3\sin^2(t))^{1/2}]_0^{\pi / 2} = -1/2$

So sometimes path does not matter (electricity and spring) but sometimes it seems to matter (friction). Something about the ``swirl'' of the field (curl)?  Addressed upcoming.


\subsection{4.3: Line Integrals}

A line integral is 
\begin{itemize} 
\item Most generally: $\int_{s=0}^{s=l} f(\vec{x}(s))ds$ or $\int_C f ds$.
\item Built out of function heights $h_i$ over curve snippet lengths $\Delta s_i$ : $\sum_i h_i \Delta s_i$.
\item More practically written as $\sum f(x_i, y_i)\sqrt{[\Delta x]^2 + [\Delta y]^2}$
\item Riemanned up: $\int_{s=0}^{s=l} f(\vec{x}(s))ds$
\item Practical integral based off of some $t: \int_C f(\vec{x}(t))\|\vec{x}'(t)\|dt$
\end{itemize}

Example: Area of curtain, height = $f(x,y) = y^2$, base is origin circle of radius 2 on xy-plane.
\begin{itemize}
\item $A = \int_{t=0}^{t=2\pi} f(x,y) \|  \frac{d<2 \cos (t), 2 \sin (t), 0>}{dt} \|dt$
\item $A = \int_{t=0}^{t=2\pi} 4\sin^2(t)  * 2 = 4 \int_{t=0}^{t=2\pi} (1 - \cos(2t)) = 4[t - \frac{\sin(2t)}{2}]_0^{2\pi} = 8 \pi$
\end{itemize}

Example: Geometric area of $f(x,y)=4y^3$ over the curve $x = \frac{y^3}{3}, (\frac{-1}{3}, 1) \rightarrow  (\frac{1}{3}, 1)$

\begin{itemize}
\item $\vec{x}(t) = \langle t^3 / 3, t \rangle \Rightarrow \vec{x}'(t)  = \langle t^2, 1 \rangle \Rightarrow \| \vec{x}'(t)  \| = \| \sqrt{1 + t^4} \|$
\item Note that this is an odd function: the part where $y \in [-1, 0]$ is a negative of [0,1].  So we'll double the right half.
\item $A = 2 \int_{t=0}^{t=1} 4t^3 \sqrt{1+t^4}dt =[\frac{4}{3}(1+t^4)^\frac{3}{2}]_0^1 = \frac{4}{3}[2\sqrt{2}- 1]$
\end{itemize}

Example: Moment of interatia around z-axis: $\int[x^2+y^2]\rho ds$ for $\vec{x}(t)=\langle 2\sin(t), 2\cos(t), 3t\rangle t \in [0,2\pi]$ if $\rho = \frac{1}{2\pi\sqrt{13}}$:
\begin{itemize}
\item $\vec{x}'(t) = \langle 2\cos(t), 2\sin(t), 3 \Rightarrow \| \vec{x}'(t)  \| = \sqrt{4 + 9}$
\item $x^2 + y^2 = (2\sin(t))^2 + (2\cos(t))^2 = 4$
\item $A = \rho * \int_{t=0}^{t=2\pi} 4 \sqrt{13} =  \frac{1}{2\pi\sqrt{13}} * 2\pi * 4\sqrt{13} = 4$
\end{itemize}

Example: Infinite wire of current going up on the x-axis.  Field equation, for (generated?) field $\vec{B}$, penetrated region bounded by curve $C$, unit tangent (to the curve) $\hat{T}$ , constant $\mu_0$ and current amount $I$: \fbox{$\int_C [\vec{B} \cdot \hat{T}]ds = \mu_0I$}.  What is $\| \vec{B} \|$ at some distance from the z-axis $r$?
\begin{itemize} 
\item C should be a circle of radius $r$.  Then, $ds = \frac{d\vec{x}}{dt}dt = r \| \cos(t), \sin(t) \| = r$
\item $\vec{B} = \|B\| \hat{T}$ by definition I suppose.
\item $\int_{t = 0}^{t=2\pi} [ \|B\| \hat{T} \cdot \hat{T}] r dt$
\item $\|B\| * 2\pi r = \mu_0 I$
\item $\Rightarrow \|\vec{B}\| = \frac{\mu_0 I}{ 2\pi r }$

\end{itemize} 


\end{document}

