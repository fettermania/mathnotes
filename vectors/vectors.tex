\documentclass[11pt, oneside]{article}   	% use "amsart" instead of "article" for AMSLaTeX format
\usepackage{geometry}                		% See geometry.pdf to learn the layout options. There are lots.
\geometry{letterpaper}                   		% ... or a4paper or a5paper or ... 
\usepackage[parfill]{parskip}    		% Activate to begin paragraphs with an empty line rather than an indent
\usepackage{graphicx}				% Use pdf, png, jpg, or eps§ with pdflatex; use eps in DVI mode
								% TeX will automatically convert eps --\rangle pdf in pdflatex		
\usepackage{amssymb}
\usepackage{amsmath}

\usepackage{verbatim}
\usepackage{tikz} 

\usepackage{syntonly}
% \syntaxonly \langle -- use this for checking syntax only
% \mbox {text} - keep together
% \fbox {text} - keep together and draw around

%\pagestyle{plain|headings|empty} % header and footer p.27
%SetFonts
%\include{filename}, \includeonly{filename1, filename2} , \input[fiename}

%SetFonts% 

\title{Brilliant: Vector Calculus}
\author{Dave Fetterman}
\date{6/21/22}							% Activate to display a given date or no date

\begin{document}
\maketitle
Note: Latex reference: http://tug.ctan.org/info/undergradmath/undergradmath.pdf
\section{Chapter 2.1: Calculus of Motion}

Consider vectors of motion against $t$ of the form $ \overrightarrow{x}(t) = \langle x(t), y(t), \ldots \rangle$.
\begin{itemize}
\item A \textbf{line} through $p = (a, b, c)$ parallel to $\overrightarrow{v} = \langle v_x, v_y, v_z\rangle$ is $\overrightarrow{x}(t) = \overrightarrow{p} + t \overrightarrow{v}$ 
\item \textbf{velocity} is characterized completely by $\overrightarrow{v}(t) = \overrightarrow{x}'(t) = \langle x'(t), y'(t), z'(t)\rangle$.
\item The \textbf{speed} of an object along that line versus $t$ is the length of $v$ ($\|v\|$)  
\item Therefore, the speed of an object along line 
$$ \langle x(t), y(t), z(t)\rangle = \langle 0, 2, -3\rangle + t\langle 1,-2,2\rangle$$ is $$\sqrt{1^2+(-2)^2 + 2^2} = 3$$
\item Note that $\overrightarrow{v}$ need not be constant.  The speed of $$\overrightarrow{x}(t) =  \overrightarrow{p} + 3 \sin(2\pi t)\hat{u}, \| \hat{u} \| = 1$$ would then be $$\| 6\pi \cos(2 \pi t) \hat{u} \| = |6\pi \cos(2 \pi t)|$$
\item \textbf{Acceleration} $a(t) = v'(t) = x''(t)$ is straightforward.  Acceleration of $$x(t) = \langle -1 + \cos(t), 1, \cos(t)\rangle = \langle -\cos(t), 0, -\cos(t)\rangle$$
\item An example position vector for a planet of distance $r$ from the sun could be $\langle r \cos(t), r \sin(t) \rangle$.  The acceleration vector points in the opposite direction: $\langle - r \cos(t), - r \sin(t) \rangle$.  In addition to being the analytical second derivative, consider that the \emph{force} of gravity, (which, by $F = ma$ is proportional to acceleration) points towards the sun.  
\item A \textbf{helix} could be a 3D extension like $\langle r \cos(t), r \sin(t), b\cdot t \rangle$.  
\end{itemize}

\section{Chapter 2.2: Space Curves}
\begin{itemize}
\item TODO: Problem 5 - rotating ellipses and solving intersections with planes
\item Note that while $\overrightarrow{x}(t) = \langle \cos(t), \sin(t), 5 \rangle$ and  $\overrightarrow{x}(t) = \langle \cos(2t), \sin(2t), 5 \rangle$ describe the same curve, the space curve also records motion in time, so the \emph{velocity} may be different.
\item If $\overrightarrow{x}(t) = t\overrightarrow{v}$, then speed is $\frac{\| \overrightarrow{x}(t+\Delta t) - \overrightarrow{t} \|}{\Delta t} = \| \overrightarrow{v} \|$,  direction is $\frac{\overrightarrow{v}}{\| \overrightarrow{v} \|}$, and velocity $\overrightarrow{v}$ is the product of speed and direction.
\item So $\overrightarrow{v}(t) = \lim_{\Delta t \rightarrow 0} \frac{\overrightarrow{x}(t + \Delta t) - \overrightarrow{x}(t)}{\Delta t} = \overrightarrow{x}'(t) = \frac{d\overrightarrow{x}}{dt} = \langle x'(t), y'(t), z'(t) \rangle$
\item Neat conceptual result: any $y = f(x)$ can be made into $x(t) = \langle t, f(t), 0 \rangle$, and then $v(t) = \langle 1, f'(t), 0 \rangle$, which points along the tangent line at $\langle t, f(t), 0 \rangle$.
\item Note that dot product derivatives work like regular product: $[\overrightarrow{a}(t) \cdot \overrightarrow{b}(t)]' = \overrightarrow{a}'(t) \cdot \overrightarrow{b}(t) + \overrightarrow{a}(t) \cdot \overrightarrow{b}'(t)$, 
but the cross product does not work the same since $\frac{d}{dt}[a \times b] = a' \times b + a \times b'$, but since $a \times b' = -b' \times a$, can't switch the order to $a' \times b + b' \times a$ due to this non-commutativity.
\item If $$\overrightarrow{x}(t) = \overrightarrow{p}+ t \overrightarrow{v},$$ calculating velocity with respect to origin  becomes
$$ \frac{d}{dt} \| \overrightarrow{x}(t) \| = \frac{\overrightarrow{x}(t) \cdot \overrightarrow{x}'(t)}{\| \overrightarrow{x}(t) \|} = \frac{\overrightarrow{x}}{\| \overrightarrow{x} \|} \cdot \overrightarrow{v},$$ after rewriting the distance formula and chugging through the chain rule.
\item However, it becomes more clear when considering that $(\overrightarrow{v} \cdot \hat{x}) \hat{x}$ is the projection of the velocity vector onto the position vector.  So, the length of this is the rate of change of distance from origin!
\end{itemize}

\section{Chapter 2.3: Integrals and Arc Length}
\begin{itemize}
\item Integral of a vector function can be defined componentwise in a straightforward way: $\int_a^b {\overrightarrow{x}(t)} = \langle \int_a^b {x(t)},  \int_a^b {y(t)},  \int_a^b {z(t)} \rangle $ 
\item Example: if ball launched from origin with velocity $\langle 1, 2, 3 \rangle$ and acceleration $\langle 0, 0, -1 \rangle$, it lands at
\begin{align} 
\frac{dv}{dt}dt = \langle 0, 0, -1 \rangle \\ 
\int{\frac{dv}{dt}dt} = v =  \langle C , D , -t +F \rangle  = \langle 1, 2, 3 \rangle =  \langle 1, 2, -t + 3\rangle, t = 0 \\
x = \int{v}  = \langle t+K, 2t+M, -\frac{1}{2}t^2 + 3t + N \rangle, x(\overrightarrow{0}) = \langle 0 ,0, 0\rangle \\ 
\overrightarrow{x}(t) = \langle t, 2t, 3t-\frac{1}{2}t^2 \rangle \\
z(t) = 0 \rightarrow t = 6 \rightarrow \overrightarrow{x}(6) = \langle 6, 12, 0 \rangle \\
\end{align} 
\item Also, generalizing $ds = \sqrt{(dx)^2  + (dy)^2}$, the length of an arc from point $a$ to $b$ aporoaches $\int_a^b \| x'(t) \| dt$
\item Example: a helix $\langle a \cos (\omega t), a \sin (\omega t), b \omega t \rangle$, parametrized by time $t$ can be rewritten in terms of $s$, the arc length: 
\begin{align}
s =  \int \|x'(t)\| dt \\
s = \int \sqrt{(-\omega a \sin (\omega t))^2 + (\omega a \cos (\omega t))^2 + (b \omega)^2}dt \\
s = |\omega| \int \sqrt{(a^2 + b^2)}dt \\ 
s = |\omega| t \sqrt{a^2 + b^2}
\end{align}
\item \emph{Note: It's weird to think of time in terms of length}.  Could be analytically useful?
\end{itemize}


\section{Chapter 2.4: Frenet Formulae}

Main idea: Establish three new vectors $\hat{T}(s), \hat{N}(s), \hat{B}(s)$ that change as we move along a space curve, instead of $\overrightarrow{x}(t)$ that changes over an external "time" idea.

Remember that  $s = \int_0^t \| \overrightarrow{x}'(\tilde{t}) \| d\tilde{t}$, so $\frac{ds}{dt} = \| \overrightarrow{x}'(t)\|$ .

\subsection{$\hat{T}$: Vector tangent to space curve}
\begin{itemize}
\item Remember arc length is $s = \int_0^t \| \overrightarrow{x}'(\tilde{t})d\tilde{t} \|$
\item $\hat{T}$ is just normalized grad: $\frac{\overrightarrow{x}'(t)}{\|\overrightarrow{x}'(t)\|}$
\item This implies $\frac{d\overrightarrow{x}}{ds} = \hat{T}$ since 
\begin{align}
s = \int_0^t \| \overrightarrow{x}'(\tilde{t})d\tilde{t} \| \\
\frac{ds}{dt} = \| \overrightarrow{x}(t) \| \\
\hat{T} = \frac{\overrightarrow{x}'(t)}{\|\overrightarrow{x}'(t)\|}  = \frac{d\overrightarrow{x}}{dt} \cdot \frac{dt}{ds} \\
\hat{T} = \frac{d\overrightarrow{x}}{ds} \\
\end{align}
\end{itemize}


\subsection{$\hat{N}$: Vector normal to space curve and also in the direction of acceleration}
Normal vectors include
\begin{itemize}
\item $\frac{\hat{T}(t)} { \| \hat{T}(t) \|}$ since, as $\| \hat(T) \| $ is just 1:
\begin{align}
d(\| \hat{T} \|^2) = 0 \\
d(\| \hat{T} \|^2) = d(\hat{T} \cdot \hat{T}) = \hat{T}(t) \cdot 2\hat{T}'(t) \\
\hat{T}(t) \cdot \hat{T}'(t)  = 0
\end{align}
\item $\frac{ \frac{d\hat{T}}{ds}  }  { \| \frac{d\hat{T}}{ds}  \| }$ since it's the same as the above, but parametrized over $s$ instead of $t$.  Doesn't change the direction of the vector!
\end{itemize}

Example:if  $\overrightarrow{x}(t) = \langle R\cos(\omega t), R\sin(\omega t), 0 \rangle$, then acceleration $\overrightarrow{a}(t)$ is
\begin{itemize}
\item $\overrightarrow{a} = \frac{d^2\overrightarrow{x}}{dt^2}$ just by definition
\item $\overrightarrow{a} = -\omega^2 \overrightarrow{x}$ just by calculation
\item $\hat{T}(t) = \langle -\sin(\omega t), \cos (\omega t), 0 \rangle$
\item $\| \hat{T}(t) \| = 1$
\item $\hat{N} = \frac{\hat{T}(t)} { \| \hat{T}(t) \|} =  \langle -cos(\omega t), -\sin (\omega t), 0 \rangle$
\item So $\overrightarrow{a} = R\omega^2\hat{N}$ by these formulae.
\end{itemize}

This leads us to believe acceleration and $\hat{N}$, the normed derivative of $\hat{T}$ are related.

The part of acceleration $\overrightarrow{a}$ parallel to $\hat{T}$ is the projection $(\overrightarrow{a} \cdot \hat{T}) \hat{T}$

The perpendicular part is then $\overrightarrow{a}$ minus that: $\overrightarrow{a} - (\overrightarrow{a} \cdot \hat{T}) \hat{T}$

This also equals $(\frac{ds}{dt})^2 \|\frac{d\hat{T}}{ds} \| \hat{N}$ because
\begin{align}
\overrightarrow{x} = \frac{dx}{dt} = T = \hat{T} \cdot \|  \frac{dx}{dt} \| \\
s = \int_0^t \| \overrightarrow{x}'(t) \rightarrow \frac{ds}{dt} = \| \overrightarrow{x}'(t) \|  \\
\hat{N} = \frac{d\hat{T}}{ds} normalized, so \\
\overrightarrow{a} = \frac{d^2\overrightarrow{x}}{dt^2} = \frac{d}{dt}(\| \overrightarrow{x}'(t)\hat{T}(t)\|) = \frac{d \|\overrightarrow{x}'(t) \|}{dt}\hat{T} + \| \overrightarrow{x}'(t) \| \frac{d\hat{T}}{dt} \\
= \frac{d \|\overrightarrow{x}'(t) \|}{dt}\hat{T} + \frac{ds}{dt} \frac{d\hat{T}}{ds} \frac{ds}{dt} \\ 
= \frac{d \|\overrightarrow{x}'(t) \|}{dt}\hat{T} + (\frac{d{s}}{dt} )^2 \| \frac{d\hat{T}}{ds} \| \hat{N}
\end{align}

This is ``a = parallel part plus perpendicular (N) part'', so the second term is $a_{\bot}$

\subsection{Binormal vector $\hat{B}$}

Note that curvature $\kappa(s) = \| \frac{d\hat{T}}{ds}\|$ is geometric (depends on s, not time) and changes as $\hat{T}$ changes.

Example: Curvature of $\overrightarrow{x}(t) = \langle \cos(t), \sin(t), bt \rangle$ 
\begin{align}
x'(t) = \langle -\sin(t), \cos(t), b \rangle \\ 
\|x'(t) \| = \sqrt(1 + b^2) \\ 
s = \int_0^t\|x'(t)\| = \int_0^t \sqrt{(1+b^2)} = t\sqrt{(1+b^2)} \rightarrow t = \frac{s}{\sqrt{1+b^2}}
\end{align}


\subsection{$\hat{T}$ is:} 
\begin{itemize}
\item $\overrightarrow{x}'(t)$ normalized
\item The tangent vector to the curve
\item The same whether paremetrized by $\hat{T}'(t)$ or $\frac{dx}{ds}$
\end{itemize}

\subsection{$\hat{N}$ is:} 
\begin{itemize}
\item $\overrightarrow{x}''(t)$ normalized as $\frac { \frac{d\hat{T}}{ds} }   {\|  \frac{d\hat{T}}{ds}  \|}  = \hat{N}$
\item The normal vector to the curve
\item $\bot$ to $\hat{T}$ in direction of acceleration.  So a multiple of acceleration vector.
\item The same whether paremetrized by $\hat{T}'(t)$ or $\frac{dx}{ds}$
\end{itemize}

\subsection{$\hat{T}$ and $\hat{N}$}:

\begin{itemize}
\item Form a plane, since first, any normal vector's derivative is perpendicular to the vector
\begin{align}
\frac{d}{ds} \|\hat{T}\|^2 = \frac{d}{ds} \hat{T} \cdot \hat{T} \\
= 2\hat{T} \cdot \hat{T}' \\
\frac{d}{ds} \|\hat{T}\|^2 = \frac{d}{ds} 1 = 0 \\
\end{align}
and 
\begin{align}
\hat{T} \cdot \hat{N} =   \hat{T} \cdot \frac { \frac{d\hat{T}}{ds} }   {\|  \frac{d\hat{T}}{ds}  \|} \\ 
=  \hat{T} \cdot \frac { \hat{T}'(s)}  {\|  \frac{d\hat{T}}{ds}  \|} = 0 
\end{align}
\item $\kappa$ is curvature: how much we're curving in that $T \times N$ plane.
\item $\kappa =   \|  \frac{d\hat{T}}{ds}  \| $ 
\item Therefore, by above, $\frac{d\hat{T}}{ds} = \kappa \hat{N}$ \textbf{(Frenet equation 1)}

\end{itemize}

\subsection{$\hat{B}$ is binormal: perpendicular to both}
\begin{itemize}
\item defined as $\hat{B} = \hat{T} \times \hat{N}$
\item Therefore, by derivative 
\begin{align}
\frac{d\hat{B}}{ds} = \frac{d\hat{T}}{ds} \times \hat{N} +  \hat{T} \times \frac{d\hat{N}}{ds}\\ 
\frac{d\hat{B}}{ds} = \kappa \hat{N} \times \hat{N}  +  \hat{T} \times \frac{d\hat{N}}{ds}\\ 
\frac{d\hat{B}}{ds} =  \hat{T} \times \frac{d\hat{N}}{ds}\\ 
\end{align}
but this means T is orthogonal to dB, and we already know B and dB are orthogonal.
We're working in 3d with the cross product, so dB is parallel to N.
\item Therefore, we define "torsion" $\tau$ so that $- \frac{d\hat{B}}{ds}  = \tau \hat{N}$  \textbf{(Frenet equation 2)}.  Negative sign by convention.
\item Can also cross by $N$ on both sides to get $- \frac{d\hat{B}}{ds}  \times \hat{N} = \tau $ 
\item $\hat{B}$ measures how the plane defined by $\hat{T}, \hat{N}$ twists around.  On a circle, $\hat{B}$ wouldn't change, so the derivative would be zero.
\item \textbf{Final Frenet equation}.  Prereq: $\hat{B} = \hat{T} \times \hat{N} \rightarrow \hat{N} = \hat{B} \times \hat{T} \rightarrow \hat{T} = \hat{N} \times \hat{B}$
\begin{align}
\frac{d\hat{N}}{ds} = \frac{d\hat{B}}{ds} \times \hat{T} +   \hat{B} \times  \frac{d\hat{T}}{ds} \\
\frac{d\hat{N}}{ds} = -\tau \hat{N} \times \hat{T} + \hat{B} \times \kappa \hat{N} \\ 
\frac{d\hat{N}}{ds}  = \tau \hat{B} - \kappa \hat{T}
\end{align}

\end{itemize}



\end{document}