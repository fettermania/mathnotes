\documentclass[11pt, oneside]{article} 	% use "amsart" instead of "article" for AMSLaTeX format
\usepackage{geometry} 		% See geometry.pdf to learn the layout options. There are lots.
\geometry{letterpaper}  		% ... or a4paper or a5paper or ... 
\usepackage[parfill]{parskip} 		% Activate to begin paragraphs with an empty line rather than an indent
\usepackage{graphicx}				% Use pdf, png, jpg, or eps§ with pdflatex; use eps in DVI mode
								% TeX will automatically convert eps --> pdf in pdflatex		
\usepackage{amssymb}
\usepackage{amsmath}
\usepackage{authblk}
\usepackage[
backend=biber,
style=alphabetic,
]{biblatex}
\usepackage{graphicx}
\graphicspath{ {./images/} }
\usepackage{verbatim}
\usepackage{tikz} 
\usepackage{subfig}
\usepackage{hyperref}

\usepackage{syntonly}
% \syntaxonly <-- use this for checking syntax only
% \mbox {text} - keep together
% \fbox {text} - keep together and draw around

%\pagestyle{plain|headings|empty} % header and footer p.27
%SetFonts
%\include{filename}, \includeonly{filename1, filename2} , \input[fiename}

%SetFonts% 


\title{Shorty Paper: Waldman-Beltrone Divsion}
\author{Dave Fetterman}
\affil{Obviously Unemployed}
\date{9/29/23}
\begin{document}
\maketitle

\section{Motivation and Standard Approach}

\textbf{\emph{Suppose you were looking to receive \$1000 from a tax-deferred savings account with a tax rate of $t=40\%$.  How much would you have to withdraw in order to end up with \$1000? }}

There are at least two approaches to this.  To societies who've discovered division, we can simply set up equation (1), meaning ``My post-tax remainder $r$ of what total withdrawal $x$ yields 
\$1000?''

\begin{align}
rx = 1000 \\
(1-t)x = 1000 \\
x = \frac{1000}{1-t} \\ 
\end{align}

In this case, (3) show us we'd need to withdraw $\frac{1000}{1-.4} = 1666.67$ to get to 1000 after taxes.

However, there does exist another way to get this figure without the bother of conventional division.

\section{Iterated Withdrawal}

Considered by L. F. Waldman, possibly among others, the following method \emph{also} produces the desired effect of getting the right amount of post-tax money out.
\begin{enumerate}
\item Start with your shortfall $x_0$. In the the above example $x_0 = 1000$.
\item Withdraw your shortfall $x_i$ from the bank.  After setting aside tax debt $x_i \cdot t$, set your new shortfall $x_{i+1} \leftarrow x_i \cdot t$.
\item If shortfall is less than the quantum unit of currency $\epsilon$, end with your withdrawal and debt piles completed.  Else set $i \leftarrow i+1$ and go to step (1).
\end{enumerate}

Thus, instead of withdrawing \$1666.67 and setting aside \$666.67 for taxes, we:
\begin{itemize}
\item \textbf{Withdraw 1000}, set aside 1000 * .4 = 400.  Total post-tax: 600.
\item \textbf{Withdraw 400}, set aside 400 * .4 = 160.  Total post-tax: 600 + 240.
\item \textbf{Withdraw 160}, set aside 160 * .4 = 64.   Total post-tax: 600 + 240 + 96.
\item \textbf{Withdraw 64}, set aside 64 * .4 = 25.60.  Total Post tax = 600 + 240 + 96 + 38.4
\item ... 
\end{itemize}

After an infinite number of iterations, We end up with \$1666.67 withdrawn from the bank, \$1000 in our pocket and \$666.67 for the tax man.  Simple!

\section{Proof} 

A pre-tax withdrawal of  $\frac{x}{1-t}$  before application of tax at rate $t$ produces post-tax income of $x$.  This exercise is left as proof to Larry Waldman\footnote{Also known as ``the'' reader.}.  We prove that the Waldman-Beltrone method of withdrawal also produces this result. 

 If $|t| < 1$, the well known series $Q = 1 + t + t^2 + t^3 ...$ converges:

\begin{align}
Q = 1 + t + t^2 + t^3 + ... = \sum_{r=0}^{\infty} t^r\\ 
tQ = t  + t^2 + t^3 + t^4 ... = \sum_{r=0}^{\infty} t^r \\
(1-t)Q = 1 \\
Q = \frac{1}{1-t} \\
\end{align}

But we see that $Q$ is exactly what we're calculating in Waldman-Beltone wfithdrawal:

\begin{itemize}
\item \textbf{Withdraw shortfall of 1000} $ = 1000 * 1 = xt^0$
\item \textbf{ Withdraw shortfall of 400} $= 1000 * .4. = xt^1$
\item \textbf{Withdraw shortfall of 160} $= 1000 * .4 * .4 = xt^2$
\item \textbf{Withdraw shortfall of 64} $= 1000 * .4 * .4 * .4  = xt^3$.
\item ...
\end{itemize}

We can see that our total withdrawal ends up being $x(1+t+t^2 + t^3 + ...) = x\frac{1}{1-t}$ as above.

\section{So what?}


This means that \emph{we can compute division $\frac{x}{d}$ with  $-1 < d < 1$ entirely from the operations of addition, subtraction, and multiplication}.  We can easily expand this to add $d \neq 0$ with the addition of a simple [decimal] shift operator  $shift(x, a)$ which shifts the decimal point $a$ units left if $a \leq 0$  and $a$ units right if $a > 0$.  If we're operating in base $b$, this means  multiplying by $b^a$.

\textbf{Computing $\frac{x}{r}$ via Waldman-Beltrone division to precision $\epsilon$}:
\begin{enumerate}
\item Shift $r$ by $a$ places until $|r| < 1$.
\item $tot \leftarrow 1, i \leftarrow 0, t_0 = 1-r$
\item $t_{i+1} \leftarrow t*t_{i}, tot \leftarrow t_{i+1}$
\item $i \leftarrow i+1$
\item If $xt_{i} < \epsilon$, go to step 3. 
\item Otherwise, shift $x * tot$ by $a$ places, and return.
\end{enumerate}

The advantages to W-B divison include:
\begin{itemize}
\item Ability to implement with only addition, subtraction (to get $t=1-r$), multiplication, and shift operators.
\item May impresses your friends.
\end{itemize}

The disadvantages include:
\begin{itemize}
\item Theoretically takes infinite time to complete.
\item Difficult and absurd.
\end{itemize}

We also acknowledge that long division also only requires the operations of addition, subtraction, multiplication, and shift (and some sort of ``compare''), but we are not personal friends with Mr. Long, nor do we care to be.

\end{document}
