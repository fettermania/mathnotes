\documentclass[11pt, oneside]{article} 	% use "amsart" instead of "article" for AMSLaTeX format
\usepackage{geometry} 		% See geometry.pdf to learn the layout options. There are lots.
\geometry{letterpaper} 		% ... or a4paper or a5paper or ... 
\usepackage[parfill]{parskip} 		% Activate to begin paragraphs with an empty line rather than an indent
\usepackage{graphicx}				% Use pdf, png, jpg, or eps� with pdflatex; use eps in DVI mode
								% TeX will automatically convert eps --> pdf in pdflatex		
\usepackage{amssymb}
\usepackage{amsmath}
\usepackage{amsthm}
\usepackage{authblk}
\usepackage{framed}
\usepackage[
backend=biber,
style=alphabetic,
]{biblatex}
\usepackage{graphicx}
\graphicspath{ {./images/} }
\usepackage{verbatim}
\usepackage{tikz} 
\usepackage{subcaption}
\captionsetup{compatibility=false}



\usepackage{syntonly}

% \syntaxonly \langle -- use this for checking syntax only
% \mbox {text} - keep together
% \fbox {text} - keep together and draw around

%\pagestyle{plain|headings|empty} % header and footer p.27
%SetFonts
%\include{filename}, \includeonly{filename1, filename2} , \input[fiename}

%SetFonts% 


\title{Turan's Theorem}
\author{Dave Fetterman}
\affil{Employed in not math}
\date{7/12/25}
\newtheorem{theorem}{Theorem}[section]
\newtheorem{lemma}[theorem]{Lemma}
\newtheorem{corollary}[theorem]{Corollary}
\newtheorem{proposition}[theorem]{Proposition}

\begin{document}

\section{Turan's Theorem Original Proof}

Set
\[
  f(n,k)=\frac12\,n^{2}\Bigl(1-\frac1k\Bigr).
\]

\subsection*{Turan's theorem}

\begin{theorem}[Turan]
If a graph \(G=(V,E)\) on \(n\) vertices satisfies \(|E|>f(n,k)\), then \(G\) contains a \((k+1)\)-clique.
\end{theorem}

\paragraph{Contrapositive.}
If \(G\) has no \((k+1)\)-clique, then
\[
  |E|\le f(n,k).
\]

\bigskip
\textbf{Proof Plan}: If we can transform any graph  \(G=(V,E)\)  without a \((k+1)\)-clique into a graph  \(G^*=(V,E^*)\) where \(G^*\) has no  \((k+1)\)-clique, and \(|E| = |E^*| \leq f(n,k)\), the contrapositive of Turan's theorem follows directly.

We consider such a set of graphs next: \(k\)-partite graphs.  
 


\begin{lemma}[\(k\)-partite bound]\label{lem:kpartite}

A \(k\)-partite graph on \(n\) vertices contains no \((k+1)\)-clique and satisfies
\[
  |E|\le f(n,k).
\]
(The proof of the \(k\)-partite bound lemma is in a later section.)
\end{lemma}


\begin{lemma}[Transformation lemma]\label{lem:transform}
Let \(G\) be a graph on \(n\) vertices with no \((k+1)\)-clique.  
Then there exists a \(k\)-partite graph on the same vertex set with the same number of edges, in particular:
\[
  |E|\le\frac12 n^{2}\Bigl(1-\frac1k\Bigr),
\]
Additionally, \(G\) possesses a vertex of degree at most \(n\bigl(1-\frac1k\bigr)\).
\end{lemma}

\subsection*{Base case}
For \(n\le k+1\) the claim is immediate: the complete graph \(K_{k+1}\) contains \(\frac12(k+1)k\) edges.  If we have less than or equal to \(\frac12(k+1)k - 1\) edges, we cannot form a \(k+1\)-clique.  We show that this edge count is less than the bound.

\begin{itemize}
\item $\frac1{2}(k+1)k - 1 \leq \frac1{2}(k+1)^2\frac{k-1}{k}$
\item $\frac1{2}k^3+\frac1{2}k^2-k \leq \frac1{2}(k^2+2k+1)(k-1)$
\item $\frac1{2}k^3+\frac1{2}k^2-k \leq \frac1{2}k^3+k^2-\frac1{2}k-\frac1{2}$
\item $-\frac1{2}k^2-\frac{k}{2} + \frac1{2} \leq 0$
\end{itemize}

which is clear since $k > 1$.

Then, any graph avoiding a \((k+1)\)-clique already satisfies the bound and is itself \(k\)-partite (some parts may be empty).


\subsection*{Average degree observation}

Because \(|E| \leq \frac12 n^{2}(1-\frac1k)\), the average degree is at most \(n(1-\frac1k)\); hence some vertex has degree at most \(\lfloor(n(1-\frac1k)\rfloor\).

\subsection*{Inductive step}

Assume \(n>k+1\) and that the lemma holds for all  graph of node count \(n\) or fewer.  
Let \(G\) be a graph on \(n+1\) vertices with no \((k+1)\)-clique.

Every induced subgraph of size \(n\) also avoids \((k+1)\)-cliques and so meets the edge bound \(f(n,k)\).

\bigskip
\textbf{Goal:} find a vertex \(v\) of degree
\[
  \deg(v)\le D:=n-\left\lfloor\frac{n}{k}\right\rfloor.
\]

With such a vertex we delete \(v\), apply the induction hypothesis to \(G-v\) to obtain a \(k\)-partite graph, then assign \(v\) to a smallest partition (necessarily of size \(\leq \lfloor\frac{n}{k}\rfloor\)), rewiring its at most \(D\) edges across the remaining partitions.  The resulting graph is \(k\)-partite with the same edge count as \(G\).

\begin{lemma}[Degree bound lemma]\label{lem:degbound}

The degree bound \(\lfloor n(1-\frac1{k})\rfloor\) will be equal to the bound \(D\) if \(k\mid n\) and equal to \(D-1\) if \(k\nmid n\).

\end{lemma}

If \(k\mid n\) then
\[
 n-\Bigl\lfloor\frac{n}{k}\Bigr\rfloor=\Bigl\lfloor n\Bigl(1-\frac1k\Bigr)\Bigr\rfloor,
\]
 the floor operation becomes the identity and equality follows.
 
If \(n = qk + r, \lfloor n(1-\frac1{k})\rfloor = \lfloor(qk+r) - q - \frac{r}{k}\rfloor\), and \(n-\lfloor \frac{n}{k}\rfloor\) = \(qk+r-\lfloor(q+\frac{r}{k})\rfloor\).

Taking out the integer \(qk+r, \lfloor -q - \frac{r}{k}\rfloor < -\lfloor q + r/k\rfloor\), and they lie on either side of integer \(-q\), so \(\lfloor n(1-\frac1{k})\rfloor = 
n- \lfloor \frac{n}{k}\rfloor  - 1\).

\subsection*{Existence of a small degree vertex}

\begin{itemize}
\item If \(k\nmid n\), choose any subset of \(n\) vertices of \(G\), which has \(n+1\) nodes.

  By induction the subset contains a vertex \(w\) of degree \(\le n(1-\frac1k)\), which is  \(\leq D - 1\) by the Degree Bound Lemma, so the whole graph has such a vertex.  Even if \( (v, w) \in E\), \(w\) satisfies the degree bound.

\item Suppose now \(k\mid n\) (no more floors!) and, toward contradiction, that every vertex satisfies \(\deg(v) \geq D+1\).

  Count edges in all \(\binom{n+1}{n}=n+1\) induced \(n\)-vertex subgraphs.  
  By induction each has at most
  \[
    f(n,k)=\frac12 n^{2}\Bigl(1-\frac1k\Bigr)
  \]
  edges.  Summing and dividing by the multiplicity \(n-1\) with which each edge is counted (two subgraphs will contain its endpoints), we obtain an upper bound
  \[
    T_{\text{upper}}=\frac{n+1}{n-1}f(n,k).
  \]

  On the other hand, our degree assumption forces at least
  \[
    T_{\text{lower}}=\frac{n+1}{2}\bigl(n-\tfrac{n}{k}+1\bigr)
  \]
  edges in total.  Compute
  \[
    T_{\text{lower}}-T_{\text{upper}}
    =\frac{n+1}{2}\,\frac{n-k}{k(n-1)}>0
  \]
  because \(n>k\).  This contradiction establishes that a vertex of degree \(\le D\) exists.
\end{itemize}

\subsection*{Completion of the inductive step}

Removing the low-degree vertex and applying the rewiring argument yields a \(k\)-partite graph on \(n+1\) vertices with the same edge count as \(G\).  
Hence every \((n+1)\)-vertex graph without a \((k+1)\)-clique satisfies
\[
  |E|\le\frac12 n^{2}\Bigl(1-\frac1k\Bigr).
\]

\bigskip
\noindent
\textbf{Corollary (Turan).}\;
If \(|E|>\frac12 n^{2}(1-\frac1k)\) then \(G\) must contain a \((k+1)\)-clique.


\section{Proof of k-partite lemma}

A \(k\)-partite graph (perhaps a real term but I'll use it here) has \(k\) partitions of nodes, within which all are disconnected.  The nodes from differing partitions may connect.  A \em{perfectly balanced} graph has partitions all of equal size.

\begin{proposition}
The edge-count formula of a perfectly balanced $k$-partite graph is
\[
E_p(n,k)=\frac{n^{2}}{2}\Bigl(1-\frac{1}{k}\Bigr).
\]
\end{proposition}

Each of \(n\) nodes connects to \(n\frac{k-1}{k}\) nodes in other partitions.  We divide by two to get the number of edges.

\begin{proposition}
The edge-count formula of a perfectly balanced graph of node total $n$ is the upper bound to any partitioning of nodes among $k$ partitions.
\end{proposition}

We are looking at maximizing $f(\vec{a}) = \frac1{2}\sum_{i \neq j; i, j < k}^k a_i a_j$, where $a_i$ is the node count of partition $i$.

We can write this as $f(\vec{a}) = \frac1{2}\sum_{i \neq j; i, j < k} a_i a_j =  (\sum_i^k a_i)^2-  \frac1{2} \sum_i^k{a_i}^2  = n^2-  \frac1{2} \sum_i^k{a_i}^2 $.  So we are looking at minimizing the final term to maximize $f(\vec{a})$

Consider the definition of statistical variance: $Var(a_1, a_2 ... a_k) = \sum_{i=1}^k (a_i - \bar{a})^2 =  \sum_{i=1}^k (a_i^2 - 2\bar{a}a_i+\bar{a}^2) $.

So $Var(a_1, a_2 ... a_k) = \frac1{k}[\sum_{i=1}^k (a_i^2) - 2\bar{a}(\sum_{i=1}^k a_i) + k\bar{a}^2]$

Thus, ignoring constant terms, minimizing $Var(a_1, a_2 ... a_k)$ is the same as minimizing  $\sum_{i=1}^k{a_i}^2$.  But the variance reaches its minimum at zero when all values $a_i$ are equal.  This is therefore bounded by the edge count of the equal partition $E_p(n,k)=\frac{n^{2}}{2}\Bigl(1-\frac{1}{k}\Bigr).$

\subsection{Conclusion}
Therefore, any $k$-partite graph has less than or equal to $\frac{n^{2}}{2}\Bigl(1-\frac{1}{k}\Bigr)$ edges.




\end{document}
