\documentclass[11pt, oneside]{article}   	% use ``amsart'' instead of ``article'' for AMSLaTeX format
\usepackage{geometry}                		% See geometry.pdf to learn the layout options. There are lots.
\geometry{letterpaper}                   		% ... or a4paper or a5paper or ... 
\usepackage[parfill]{parskip}    		% Activate to begin paragraphs with an empty line rather than an indent
\usepackage{graphicx}				% Use pdf, png, jpg, or eps§ with pdflatex; use eps in DVI mode
								% TeX will automatically convert eps --\rangle pdf in pdflatex		
\usepackage{amssymb}
\usepackage{amsmath}

\usepackage{verbatim}
\usepackage{tikz} 

\usepackage{syntonly}
% \syntaxonly \langle -- use this for checking syntax only
% \mbox {text} - keep together
% \fbox {text} - keep together and draw around

%\pagestyle{plain|headings|empty} % header and footer p.27
%SetFonts
%\include{filename}, \includeonly{filename1, filename2} , \input[fiename}

%SetFonts% 

\title{Brilliant: Differential Equations II}
\author{Dave Fetterman}
\date{9/2/22}							% Activate to display a given date or no date

\begin{document}
\maketitle
Note: Latex reference: http://tug.ctan.org/info/undergradmath/undergradmath.pdf

\section{Chapter 1: Basics}
\subsection{Chapter 1: Nonlinear Equations}



The two types of problems in this course are: 


\begin{itemize}
\item Nonlinear equations (several equations on one independent variable)
\item Partial differential equations (single equation with several independent variables)
\end{itemize}

\textbf{Linear} equations have solutions like $y_1,y_2$ that can be combined using any $c \in \mathbb{R}$ like $y_1 + cy_2$.

\textbf{Example}: Bacteria in a dish with a lot of food, no deaths
\begin{itemize}
\item $b'(t) = r_bb(t), r_b > 0. r_b$ would be the rate of growth.
\item This is linear. Reason 1: $\frac{d}{dt}(y_1 +cy_2) = y_1' +cy_2' = r_b(y_1 +c_y2)$ since $y' = r_by(t)$, and same for y2.
\item Also, this works because the solution is $b(t) = b(0)e^{r_b t}$, so $b_1(t) + cb_2(t) = b_1(0)e^{r_b t} +cb_2(0)e^{r_b t} = (b_1(0) + cb_2(0))e^{r_bt}$
\end{itemize}

\textbf{Example}: \textbf{Logistic} equation: Bacteria in a dish with a lot of food, limited by carrying capacity $M$.
\begin{itemize}
\item $b'(t) = r_bb(t)[M-b(t)]$.
\item This is nonlinear. Reason: $\frac{d}{dt}(y_1' +cy_2') = y_1'+cy_2' = r_b[y_1 + cy_2][M-y_1-cy_2] = My_1+Mcy_2-y_1^2-2cy_1y_2 -cy_1^2y_2^2$
\item $\neq My_1 - y_1^2 + Mcy_2-c^2y_2^2$ because of the extra $-2cy_1y_2$ term.
\end{itemize}

Sidebar: Note that this equation $b' = r_bb[M-b]$ is \emph{separable}, so it can be solved.
\begin{itemize}
\item $\frac{db}{dt} = rb[M-b]$
\item $\frac{db}{b(M-b)} = r dt$
\item $\frac{1}{M }(\frac{1}{b} + \frac{1}{M-b}) db = r dt$ after partial fractions work
\item $(\ln(b)-\ln(M-b)) = Mrt + C \Rightarrow \ln(\frac{b}{M-b}) = Mrt + C$
\item $\frac{b}{M-b} = e^{Mrt}e^{C}$
\item Initial conditions $b=b(0), t=0 \Rightarrow \frac{b}{M-b} = \frac{b(0)}{M-b(0)} e^{Mrt}$
\item  $b(1+\frac{b(0)}{M-b(0)} e^{Mrt}) = M \frac{b(0)}{M-b(0)} e^{Mrt}$
\item  $b(M-b(0)+b(0)e^{Mrt}) = M b(0) e^{Mrt}$
\item $b = \frac{Mb(0)e^{Mrt}}{M+b(0)[e^{Mrt} -1]}$
\end{itemize}


This logistic solution will taper off to M at some point. Note that $\lim_{t \rightarrow \infty} b(t) = M$ since the non-exponential terms stop mattering.  Also $b(t) = M$ sticks as a constant solution or \textbf{equilibrium} immediately.  \emph{These equilibria tell us what matters - the long-term behavior of solutions!}

Another \textbf{Example}: Lotka-Volterra equation pairs: Bacteria (b) and bacteria-killing phages (p), with kill rate $k$.
\begin{itemize}
\item The ``product'' $k b(t)p(t)$ measures the interactions and kills resulting from this.
\item $b'(t) = r_b b(t) - k p(t) b(t)$, or the normal growht rate minus kill rate
\item $p'(t) = kp(t)b(t)$ since its population grows as it kills bacteria.
\item Equilibria include $b = 0, p = 0$ and $b = 0, p > 0$, since these are \emph{constant} solutions, or places where $b'(t) = 0, p'(t) = 0$.
\end{itemize}

\textbf{Direction fields}, with vector pointing towards $\langle b'(t), p'(t) \rangle$ (TODO - I think) let us follow the arrows to determine the curve over time.  In this case, the bacteria will always go extinct.

\ 
However, if we add a new death rate term $-d_pp(t)$ so $p'(t) = -d_pp(t) + kp(t)b(t)$:
\begin{itemize}
\item We get an equilibrium at $b = \frac{d_p}{k}, p = \frac{r_b}{k}$. (Since $0 = b'(t) = rb - kpb, (\Rightarrow pk = r), 0 = p'(t)  -dp + kpb,(\Rightarrow bk = d)$)
\item But otherwise the solutions swirl around this point.  This is called a \textbf{cycle}.  TODO What is a \textbf{limit cycle}?
\end{itemize}

Note that there are systems where the ``solution particle'' neither reaches an equilibrium or cycles around one point.  The \textbf{Lorenz system} famously has this owl-eye shaped double attractor (an example of \textbf{strange sets}) where initially close particles diverge unpredictably if the constants $\rho, \sigma, b$ are chosen right:

\begin{itemize}
\item $x'(t) = \sigma(y-x)$
\item $y'(t) = x(\rho - z) - y$
\item $z'(t) = xy-bz$
\end{itemize}

\begin{itemize}
\item TODO
\end{itemize}

\subsection{Chapter 1.2: PDEs}

Many methods of attack for PDEs

\begin{itemize}
\item Separation of variables
\item Power series (Note: did we actually touch on this?)
\item Fourier Transform
\end{itemize}

\textbf{Example}: Standing wave, where one end of a rope is fixed.  
\begin{itemize}
\item Vertical displacement from a line of rope: $u(x,t)$ depends on where ($x$) and when ($t$t).
\item Rope's \textbf{wave equation} is $u_{tt} = v^2u_{xx}$, where $v$ is the ``constant wave speed'', and the others are the space, time partials.
\item Note that $u = \cos(vt)\sin(x)$ and $u = \sin(vt)cos(x)$ both work.
\item If you guess the solution has split variables like $u = X(x)Y(y)T(t)$, then, upon substitution and division by $X(x)Y(y)T(t)$, $\frac{\delta^2 u}{\delta t^2} = v^2 [\frac{\delta^2 u}{\delta x^2} + \frac{\delta^2 u}{\delta y^2}]$ yields $\frac{T''(t)}{T(t)} = v^2 [\frac{X''(x)}{X(x)} + \frac{Y''(y)}{Y(y)} ]$
\item This method may or may not work.  But if it does, it means that since $x, y, $ and $t$ are independent variables, each individual piece must be constant.
\item So, for example, if we know $\frac{X''(x)}{X(x)} = -4\pi^2$, we can get to $X(x) = \sin(2\pi x)$
\item The wave equation is similar in 3D: $u_{tt} = v^2[u_{xx}+u_{yy}+u_{zz}]$, or using the Laplacian, $u_{tt} = v^2\nabla^2 u$.  Here, $u$ measures not displacement but expansion/compression of air at $(x,y,z)$, time $t$.
\end{itemize}


Using Fourier transforms helps turn difficult PDEs into an easier problem like an ODE.  \emph{Fourier transforms work best} when 
\begin{itemize}
\item The domain is all of $\mathbb{R}^n$
\item The function $u$ vanishes at infinity.
\end{itemize}
 
The Fourier transform changes the domain of $x$ to that of $\omega$.  It comes with the (highly simplified) rule (see Vector Calculus course): $F[\frac{\delta f}{\delta x}] = i \omega F[f]$.  
\textbf{Example}: Drunkard's walk.
\begin{itemize}
\item One dimensional: moves left or right in a random way.  Starts at $x=0, t=0$.
\item $u(x,t)$ is probability of being at point $x$ at time $t$.  Naturally, $\int_{x=-\infty}^{x=\infty} u(x,t)dx = 1$.
\item Also, it obeys the 1-dD diffusion equation $\frac{\delta u}{\delta t} = \frac{\delta ^2 u}{\delta x^2}$
\item The Fourier transform doesn't affect $t$ at all.
\item So by taking Fourier transform of both sides of diffusion equation we get 
\begin{itemize}
\item $F(u_t)=  \frac{\delta}{\delta t} F(u) $ since $F$ doesn't care about $t$.
\item $\frac{\delta ^2 u}{\delta x^2} = i\omega F(\frac{\delta u}{\delta x}) = -\omega^2 F(u)$
\item So $  \frac{\delta}{\delta t} F(u)  = -\omega^2 F(u)$
\item This is solvable as $F(u) = ce^{-\omega^2 t}$.  Take it on faith that $c = \frac{1}{2\pi}$ for now. TODO
\item Known fact: $F[Ae^-{\frac{ax^2}{2}}] = \sqrt{\frac{1}{2 \pi a}} Ae^{\frac{-\omega^2}{2a}}$
\item This means $t = \frac{1}{2a}$ and $a = \frac{1}{2t}$
\item $F(u)  = 	\frac{1}{2\pi}e^{-\omega ^2 t}, F[Ae^-{\frac{ax^2}{2}}] = \sqrt{\frac{1}{2 \pi a}} Ae^{\frac{-\omega^2}{2a}}$ so $u  = Ae^{\frac{-ax^2}{2}}$
\item Solving, you get $A = \sqrt{\frac{1}{4\pi t}}, a=\frac{1}{2t}$, so $u(x,t) = \sqrt{\frac{1}{4\pi t}} e^{-\frac{x^2}{4t}}$
\end{itemize}

\end{itemize}


\section{Chapter 2: Nonlinear Equations}
\subsection{2.1: Lotka-Volterra I}


Major ideas:
\begin{itemize}
\item \textbf{phase plane}: TODO
\item \textbf{nullcline}: TODO
\item \textbf{direction field}: TODO
\item \textbf{equilibria}: TODO
\end{itemize}
 
\textbf{Example}: Bacteria vs. phages (again)
\begin{itemize}
\item Bacteria unrestrained grow in proportion to their population, so $\frac{db}{dt} = r_b b(t)$ (solved: $b(t) = b(0)e^{r_b t}$)
\item Phages unfed decrease in proportion to current size, so $\frac{dp}{dt} = -d_p p(t)$  (solved: $p(t) = p(0)e^{-d_p t}$)
\item Bacteria die with likelihood of meeting a phage, and phages increase with likelihood of meeting a bacterium.  So the set of equations, for constant $k$, becomes:
\begin{itemize}
\item $b'(t) = r_b b(t) - kb(t)p(t)$
\item $p'(t) = -d_p p(t) + kb(t)p(t)$
\item \emph{The product of p and b makes our equations nonlinear} (WHY?)
\item I guess, very generally, $b_1p_1 = k, b_2p_2 = k,$ but $(b_1+b_2)(p_1+p_2) = b_1p_1 + b_2p_2 + b_1p_2+b_2p_1 = 2k + b_1p_2+b_2p_1 \neq 2k$, so the last two ``mixed'' terms mean you can't just add solutions $(b_1, p_1)$ and  $(b_2, p_2)$.
\end{itemize}
\end{itemize}

General thoughts on this solution:
\begin{itemize}
\item So a solution $(b(t), p(t))$, traces out a curve on the bp-phase plane (b is x-axis, p is y-axis) as time (unrepresented in the plane) continues.
\item If we add a unit tangent vector at every point $(B, P)$ aligned with $(b'(t), p'(t)) = ( r_bB - kBP, -d_p P + kBP)$, we can follow the arrows to see the solution over time.
\item The above is called a \textbf{direction field}
\item This is sometimes hard to sketch analytically, so we can look to the \textbf{nullclines}: places where one of the components of the direction field is zero.
\item In this case, $r_bB - kBP = (r_b-kP)B = 0$ when $P = 0$ or $P = \frac{r_b}{k}$, and $-d_p P + kBP = (kB-d_p)P = 0$ when $P=0$ or $B = \frac{d_p}{k}$.
\item The \textbf{upshot of nullclines} (since we don't care about $P, B \leq 0$): The lines $B = \frac{d_p}{k}, P = \frac{r_b}{k}$ \emph{divide the plane into pieces where the components of this (continuous) function pair can't change sign}.  
\item For instance, $B > \frac{d_p}{k}, P <  \frac{r_b}{k}$  means $r_b b - kbp > 0, -d_p p+ kdp > 0$, so both populations are growing here.  This helps to sketch the curve.
\item The curve looks like a counterclockwise whirlpool around the $(B, P) = ( \frac{d_p}{k}, \frac{r_b}{k})$.  (bacteria grow with low but growing phages; bacteria decrease as phages overwhelm; both decrease as phages starve; bacteria start coming back)
\item The center point is a (constant \textbf{equilibrium}) solution, and other solutions swirl around it but don't get attracted or repelled.
\end{itemize}

There are a few types of equilibria:
\begin{itemize}
\item This one is a \textbf{center} around which solutions circle.
\item A \textbf{stable equilibrum} would see small upsets come back to an unchanging state.
\item An \textbf{unstable equilibrum} would see small upsets create wildly divergent paths.


\end{itemize}

\subsection{2.2: Lotka-Volterra II}

In the Bacteria-Phage system, we can't yet prove everything rotates around the \textbf{center}.  Let's do that.

Developing a \textbf{conserved quantity} will help to do that.  \textbf{Example}: Block on a horizontal spring with mass $m$, spring constant $k_s$:
\begin{itemize}
\item $x(t)$: Displacement from rest position.
\item $v(t) = \frac{dx}{dt}$: Horizontal velocity
\item $\frac{dv}{dt} = -\frac{k_s}{m}x(t)$ by Hooke's law, I think.
\item Suppose there's some Energy function $E(x,v)$.  By chain rule $\frac{d}{dt}E(x(t), v(t)) = \frac{dE}{dx}\frac{dx}{dt} +  \frac{dE}{dv}\frac{dv}{dt}$
\item $=  \frac{dE}{dx}v - \frac{k_s}{m} \frac{dE}{dv}x$.  If we set E as conserved, as in $E'(t) = 0$, then $\frac{dE}{dx}v = \frac{k_s}{m} \frac{dE}{dv}x$
\item We can eyeball and see that $E = \frac{1}{2}k_sx^2 +  \frac{1}{2}mv^2$ solves this equation, or we can assume $E(x,v) = F(x) + G(v) \Rightarrow 0 = E'(t) = F'(x)v - \frac{k_s}{m}G'(v)x = 0$ from the above equations and guess from there.
\item This means in the xv phase space, that there's a fixed E such that the particle follows the ellipse $E = \frac{1}{2}k_sx^2 +  \frac{1}{2}mv^2$ in phase space around the solution point (0,0).
\end{itemize}

\textbf{Extended Example}: Continuing on finding a conserved quantity for Bacteria / Phage:
\begin{itemize}
\item We need to find $U(b(t), p(t))$ such that $U'(t) = 0$, or by chain rule $\frac{\delta U}{\delta b}\frac{\delta b}{\delta t}  + \frac{\delta U}{\delta p}\frac{\delta p}{\delta t} = 0$
\item Subbing in, $\frac{\delta U}{\delta b}[r_bb-kbp]  + \frac{\delta U}{\delta p}[-d_pp+kbp] = 0$
\item A hint suggests finding $U$ such that $\frac{\delta U}{\delta b} = -\frac{d_p}{b} + k, \frac{\delta U}{\delta p} = -\frac{r_b}{p} + k$ to make terms cancel.
\item Integrating these gives us $U$ as both $-d_p\ln(b)+kb+Q(p)$ and $-r_b\ln(p)+kp+R(b)$ so $U = -d_p\ln(b)-r_b\ln(p)+kb+kp$.  This weird curve consistutes a level set in pb-space upon which a solution sits.
\item The spring example has an elliptic paraboloid solution.  There's an absolute minimum ($E = 0$ at $(0,0)$) but level sets become closed loops away from it.
\item For the Lotka example, there is a critical point ($\nabla U = \vec{0}$) when $\nabla U(b,p) = (\frac{\delta U}{\delta b}, \frac{\delta U}{\delta p}) = (k-\frac{d_p}{b}, k - \frac{r_b}{p})$, which is $(0, 0)$ at our known center $(\frac{d_p}{k}, \frac{r_b}{k})$
\item Showing that we always increase gong away from the point $(\frac{d_p}{k}, \frac{r_b}{k})$ should guarantee us closed level sets.
\item One method: Assume we're picking a unit vector $\vec{v} = \langle \hat{v_b}, \hat{v_p} \rangle$ so that our line from our center is $\vec{v} = \langle \frac{d_p}{k}+t{v_b}, \frac{r_b}{k}+t{v_b} \rangle$.  $U=F(b) + G(p)$ in this case, so sub the $b$ part into $F$ to get $F (\frac{d_p}{k} +t\hat{v_b}) = d_p[1-\ln(\frac{d_p}{k} + t\hat{v_b})]+kt\vec{v_b}$.  Taking derivative of that w.r.t $t$ shows it is always positive.  Same goes for the G(p) portion of U.
\item Another (DF) method: Note that $\nabla U = (k - \frac{d_p}{b}, k - \frac{r_b}{p})$'s grad (second derivative) is always positive.  So derivative always has positive curvature (maybe using that term wrong), and we'll always increase around this point.
\item Also, we know that the particle travels around the level set (loop) and doesn't reverse course, because then, $b'(t) = p'(t) = 0$, and we only have that at the center point (nullcline intersection)
\end{itemize}


\begin{itemize}
\item TODO 
\end{itemize}



\end{document}


