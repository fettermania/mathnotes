\documentclass[11pt, oneside]{article} 	% use "amsart" instead of "article" for AMSLaTeX format
\usepackage{geometry} 		% See geometry.pdf to learn the layout options. There are lots.
\geometry{letterpaper}  		% ... or a4paper or a5paper or ... 
\usepackage[parfill]{parskip} 		% Activate to begin paragraphs with an empty line rather than an indent
\usepackage{graphicx}				% Use pdf, png, jpg, or eps§ with pdflatex; use eps in DVI mode
								% TeX will automatically convert eps --> pdf in pdflatex		
\usepackage{amssymb}
\usepackage{amsmath}
\usepackage{authblk}
\usepackage[
backend=biber,
style=alphabetic,
]{biblatex}
\usepackage{graphicx}
\graphicspath{ {./images/} }
\usepackage{verbatim}
\usepackage{tikz} 
\usepackage{subfig}
\usepackage{hyperref}

\usepackage{syntonly}
% \syntaxonly <-- use this for checking syntax only
% \mbox {text} - keep together
% \fbox {text} - keep together and draw around

%\pagestyle{plain|headings|empty} % header and footer p.27
%SetFonts
%\include{filename}, \includeonly{filename1, filename2} , \input[fiename}

%SetFonts% 

\title{The Office DVD Problem}
\author{Dave Fetterman}
\affil{Obviously Unemployed}
\date{7/10/22}
\begin{document}
\maketitle

Screensavers have captivated [this] man since the 1990s. If watched long enough, what will the spirits of the machine tell us?

Specifically, the question of whether a bouncing rectangle will slide \emph{exactly} into the corner of the screen, for a satisfying, perfectly diametric rebound, was even addressed on \emph{The Office}
\href{https://www.youtube.com/watch?v=QOtuX0jL85Y}{(link)}.

However, though these characters reportedly watched this sleep-mode drama play out for years until payoff, we ask - under what conditions will the rectangle \emph{definitely} perfectly bounce into the screen's corner?
 
\begin{figure}
\centering
\includegraphics[scale=.5]{setup}
\caption{The Office DVD problem's most generic setup}
\end{figure}


\begin{figure}[!htb]
\centering
\includegraphics[scale=.5]{problem1trajectory}
 \caption{Success for $m = \frac{2}{3}, j, k = 0, h, l = 1$ (not to scale)}
\end{figure}

\subsection{Statement} 
Suppose we have a screen of length $l$, height $h$, containing an axis-aligned rectangle of length $j$ and height $k$ centered at point $(x, y)$.

Suppose this rectangle is launched at direction $\langle 1, m\rangle$ \footnote {Think of this as slope $m$} and ``bounces'' according to billiard rules \footnote{Glancing off a horizontal boundary, our trajectory goes from $\langle 1, m \rangle$ to $\langle 1, -m \rangle$, with $\langle \pm 1, m \rangle$ to $\langle \mp 1, m \rangle$ for a horizontal one}.

Given $l, h, j, k, m \in \mathbb{R}$, can we tell whether the rectangle ever bounce perfectly into a corner?  

We can approach this problem from the simplest version to the most complex.

\subsection{Problem 1} 

Suppose $j = k = 0$ and $x = y = 0$. In other words, suppose we have a \emph{point} starting at the bottom left corner
 (origin).  Under what conditions (i.e. choice of $m$) does this bounce into a corner?


\subsection{Problem 2} 

Suppose $j, k > 0, x = \frac{j}{2}, y = \frac{k}{2}$. In other words, suppose we have a rectangle starting at the bottom left corner. Under what conditions does this bounce into a corner?

\subsection{Problem 3} 

Suppose we have maximally open (reasonable) conditions, with $x \in [\frac{j}{2}, l - \frac{j}{2}], y \in [\frac{k}{2}, h - \frac{k}{2}]$ (that is, a $j \times k$ rectangle fitting entirely in the screen). Under what conditions does this bounce into a corner?


\end{document}
