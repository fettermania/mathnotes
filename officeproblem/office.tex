\documentclass[11pt, oneside]{article} 	% use "amsart" instead of "article" for AMSLaTeX format
\usepackage{geometry} 		% See geometry.pdf to learn the layout options. There are lots.
\geometry{letterpaper}  		% ... or a4paper or a5paper or ... 
\usepackage[parfill]{parskip} 		% Activate to begin paragraphs with an empty line rather than an indent
\usepackage{graphicx}				% Use pdf, png, jpg, or eps§ with pdflatex; use eps in DVI mode
								% TeX will automatically convert eps --> pdf in pdflatex		
\usepackage{amssymb}
\usepackage{amsmath}
\usepackage{authblk}
\usepackage[
backend=biber,
style=alphabetic,
]{biblatex}
\usepackage{graphicx}
\graphicspath{ {./images/} }
\usepackage{verbatim}
\usepackage{tikz} 
\usepackage{subfig}
\usepackage{hyperref}

\usepackage{syntonly}
% \syntaxonly <-- use this for checking syntax only
% \mbox {text} - keep together
% \fbox {text} - keep together and draw around

%\pagestyle{plain|headings|empty} % header and footer p.27
%SetFonts
%\include{filename}, \includeonly{filename1, filename2} , \input[fiename}

%SetFonts% 

\title{The Office DVD Problem}
\author{Dave Fetterman}
\affil{Obviously Unemployed}
\date{7/10/22}
\begin{document}
\maketitle

Screensavers have captivated [this] man since the 1990s. If watched long enough, what will the spirits of the machine tell us?

Specifically, the question of whether a bouncing rectangle will slide \emph{exactly} into the corner of the screen, for a satisfying, perfectly diametric rebound, was even addressed on \emph{The Office}
\href{https://www.youtube.com/watch?v=QOtuX0jL85Y}{(link)}.

However, though these characters reportedly watched this sleep-mode drama play out for years until payoff, we ask - under what conditions will the rectangle \emph{definitely} perfectly bounce into the screen's corner?
 
\begin{figure}
\centering
\includegraphics[scale=.5]{setup}
\caption{The Office DVD problem's most generic setup}
\end{figure}


\begin{figure}[!htb]
\centering
\includegraphics[scale=.5]{problem1trajectory}
 \caption{Success for $m = \frac{2}{3}, j, k = 0, h, l = 1$ (not to scale)}
\end{figure}

\subsection{Statement} 
Suppose we have a screen of length $l$, height $h$, containing an axis-aligned rectangle of length $j$ and height $k$ centered at point $(x, y)$.

Suppose this rectangle is launched at direction $\langle 1, m\rangle$ \footnote {Think of this as slope $m$} and ``bounces'' according to billiard rules \footnote{Glancing off a horizontal boundary, our trajectory goes from $\langle 1, m \rangle$ to $\langle 1, -m \rangle$, with $\langle \pm 1, m \rangle$ to $\langle \mp 1, m \rangle$ for a horizontal one}.

Given $l, h, j, k, m \in \mathbb{R}$, can we tell whether the rectangle ever bounce perfectly into a corner?  

We can approach this problem from the simplest version to the most complex.

\subsection{Problem 1} 

Suppose $j = k = 0$ and $x = y = 0$. In other words, suppose we have a \emph{point} starting at the bottom left corner
 (origin).  Under what conditions (i.e. choice of $m$) does this bounce into a corner?


\subsection{Problem 2} 

Suppose $j, k > 0, x = \frac{j}{2}, y = \frac{k}{2}$. In other words, suppose we have a rectangle starting at the bottom left corner. Under what conditions does this bounce into a corner?

\subsection{Problem 3} 

Suppose we have maximally open (reasonable) conditions, with $x \in [\frac{j}{2}, l - \frac{j}{2}], y \in [\frac{k}{2}, h - \frac{k}{2}]$ (that is, a $j \times k$ rectangle fitting entirely in the screen). Under what conditions does this bounce into a corner?


\section{Solutions}

Note: If our initial slope is zero $m = \langle 1, 0 \rangle$ or ``infinite'' ($m = \langle 0, 1 \rangle$, actually disallowed in our setup), the solution is obvious: if we're axis-aligned at the outset, we'll be in a corner shortly, otherwise we never will be.

Note also that, for the sake of simplicity, we can treat $m$ as always positive (the box going up and to the right).  If not, inverting the problem ($m \rightarrow -m, y = h - y)$ will work equivalently.  The box initially moving leftwards (disallowed in the problem setup) reduces to our setup by the same sort of flip.

\subsection{Problem 1 solution}

The key insight here is that though the point bounces ``within a box'' until meeting $(0,0), (0, h), (l, 0), $ or $(l, h)$ (as in Figure 2), we can instead look at the path of the point in an unconstrained space, seeing if we hit a point of the form $(a \cdot l, b \cdot h)$ with $a,b \in \mathbb{N}$.

On meeting the point $(0, \frac{2}{3})$, we can either consider what happens if we reflect ``back'' into our original box as in the left side of Figure 3, or, equivalently, if we pass ``through'' to a mirrored box on the right side of Figure 3.  We quickly see that:

\begin{itemize} 
\item If the left-hand side meets a corner, the mirror-image on the right-hand side will meet a corner as well.
\item Likewise for the converse: the right meeting a corner means the left hand will as well.
\item If the left-hand side does \emph{not} meet a corner, the right-hand side cannot.
\item Likewise for the converse.
\end{itemize}

\begin{figure}[!htb]
\centering
\includegraphics[scale=.5]{mirrorright}
 \caption{``Passing through'' on the right equivalent to ``bouncing back left''}
\end{figure}

Note that this applies for top-bottom just as easily as left-right (Figure 4).  

\begin{figure}[!htb]
\centering
\includegraphics[scale=.5]{mirrorup}
 \caption{``Passing through'' the top equivalent to ``bouncing back down''}
\end{figure}

Therefore, composing these two, we can cast the path of the smaller rectangle as entirely ``up and to the right''.  This allows us to reframe the question as: \textbf{Starting at $(0,0)$, will a line of slope $m$ reach a corner of a lattice of $l \times h$ blocks} or, equivalently, reach a point $(a \cdot l, b \cdot h)$ for some pair $a,b \in \mathbb{N}$?

\begin{figure}[!htb]
\centering
\includegraphics[scale=.4]{fullpath}
 \caption{Recasting the bouncing path as ``up and to the right'' on the l, h lattice}
\end{figure}


\begin{align}
a \cdot m \cdot l = b \cdot h \\
\frac{a}{b} = \frac{h}{m \cdot l} \\ 
\Rightarrow \frac{h}{m \cdot l} \in \mathbb{Q} 
\end{align}

We see that we reach a corner \textbf{if and only if $\frac{h}{m \cdot l}$ is a rational number} (quotient of integers).



\subsection{Problem 2 solution}

The key insight here is that instead of each ``frame'' extending from $(0, 0)$ to $(l, h)$, with the box's center ($x, y)$ constrained to the rectangle defined by $[\frac{j}{2}, l - \frac{j}{2}], \times [\frac{k}{2}, h - \frac{k}{2}]$, we can instead treat the \emph{center} of that $[\frac{j}{2}, l - \frac{j}{2}], \times [\frac{k}{2}, h - \frac{k}{2}]$ box as a point like in problem 1.  

It is then clear that we can use problem 1's main insight to the center point as opposed to the small rectangle: that the small rectangle, say, $\frac{j}{2}$ left of the right border of one frame will take an trajectory eqjuivalent to that of a rectangle $\frac{j}{2}$ right of the left border of the adjoining right frame.

So, restate the problem as $j, k = 0, x \rightarrow x - \frac{j}{2}, y \rightarrow \frac{k}{2}$ and solve as in problem 1.

\begin{figure}[!htb]
\centering
\includegraphics[scale=.4]{problem2}
\caption{The center of a $j \times k$ rectangle is a point bouncing in the $(l - j) \times (h - k)$ box}
\end{figure}

\subsection{Problem 3 solution}

Here, we're looking at the set of solutions where $(x, y) \neq (0, 0)$.

For simplicity, we'll consider only a zero-dimensional rectangle ($j = k = 0$) here; otherwise, it can be reduced as in the solution to problem 2 before coming back here.

It's clear that there are many non-solutions here.  Consider, on a grid $h = l = 1$, starting at point $(\frac{1}{2}, 0)$ with a slope $m = 1$.  The point will clearly bounce around between $(1, \frac{1}{2}), (\frac{1}{2}, 1), (0, \frac{1}{2})$ and $ (\frac{1}{2}, 0)$ forever without meeting a corner (Figure 7).

\begin{figure}[!htb]
\centering
\includegraphics[scale=.4]{bounceforever}
\caption{$(x, y) = (\frac{1}{2}, 0), m = 1$ bounces forever}
\end{figure}

First, find the intercept with the next x boundary as in Figure 8.

\begin{figure}[!htb]
\centering
\includegraphics[scale=.4]{intercept}
\caption{Finding the intercept $y^* = y + m(l-x)$}
\end{figure}

If we're going to end in a corner, then adding this intercept $y^*$ this to a series of $a$ `runs' of length $l$ should meet at a lattice point of $b$ `rises' of height $h$, according to slope $m$.  Note that $y^*$ may be greater than $h$ (so it could be many ``squares up''), and also note that if $y^* = 0$, then we're already found our corner.  

Given all of these quantities $y^*, m, h, l$, we're looking to discover nonnegative integers $a$ and $b$ that allow us to solve this equation:

\begin{align}
y^* + a\cdot m \cdot l = b \cdot h, a, b \in \mathbb{N} \\ 
y^* = b \cdot h -  a\cdot m \cdot l \\
1 = b \cdot \frac{h}{y^*} -  a\cdot \frac{m \cdot l}{y^*} 
\end{align}

There are three (not entirely separate) cases here:

\begin{itemize}
\item $\frac{h}{y^*}, \frac{m \cdot l}{y^*}$ are both integers. In this case, we can use Euler's algorithm to find the greatest common denominator of $\frac{h}{y^*}, \frac{m \cdot l}{y^*}$, as well as $b$, $a$.  If that GCD equals one, the algorithm will give us $b$ and $a$.  Otherwise, there is no solution (we never hit a corner).
\item $\frac{h}{y^*}, \frac{m \cdot l}{y^*}$ are rational but there is at least one non-integer among them.  In this case, multiply out the denominators by the necessary factor $d$ to make them integers, and reduce to the first case looking for GCD $d$.  If our GCD is $d$ or divides $d$, we will hit a corner, otherwise we will bounce infinitely.  Note that Figure 7's scenario ($y^* = \frac{3}{2}, m = h = l = 1$) reduces to $1 = b \cdot \frac{2}{3} - a  \cdot \frac{2}{3} \Rightarrow 3 = 2a-2b$, which has no integer solution pair.
\item Exactly one of $\frac{h}{y^*}, \frac{m \cdot l}{y^*}$ is irrational.  Then there is clearly no solution without $a$ or $b$ equaling zero (which we've eliminated).
\item Both $\frac{h}{y^*}, \frac{m \cdot l}{y^*}$ are irrational.  This reduces to:
\begin{align}
\frac{y^*}{h} = b - a\frac{m \cdot l}{h} \\
-\frac{y^*}{h} = a\frac{m \cdot l}{h} \mod 1 
\end{align}
Though we look close to a solution, this rearranging hasn't gotten us much farther since we can't multiply by $\frac{h}{m \cdot l}$ while retaining modulus 1.  For example, try corner-bouncer $y^* = 1+\sqrt{2}, l = 1, h = \frac{1}{2} + \sqrt{2}, m = \sqrt{2}$, yielding $a = 1, b = 2$; equation (8) is satisfied but multiplying by $\frac{h}{m \cdot l}$ doesn't work.  The best we can hope for is that $-y^*$ and $m \cdot l$ have an obvious relationship.

\end{itemize}

As an aside, the likelihood that a randomly chosen set of reals $y^*, m, h, l$ will produce a corner bump is zero.   To see this, let's take a set $h, l, m$ and replace $m \rightarrow m \frac{l}{h}$ so we can consider $h, l = 1$.  

Let's say our intercept happens one block to the right.  This fixes our $y^*$ to a single, determined value.  Same with two blocks, three, etc.  Assuming there is an intercept somewhere down the line, there are a countable number of $y^*$ selections which will make this system work, one for each integer greater than zero.  However there are certainly an \emph{uncountable} number of $y^*$ selections with $y^* \in (0, 1)$!







\end{document}
